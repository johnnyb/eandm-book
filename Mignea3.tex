\eandmchapter{The Simplest Self-Replicator, Part~3}{Developing Insights into the Design of the Simplest Self-Replicator and Its Complexity: Part~3---The Metaphysics of an Artificial SSR and the Origin of Life Problem}{Arminius Mignea}{The Lone Pine Software}

\begin{abstract}
This paper is the last in a three-part series investigating the internals 
of the simplest possible self replicator (SSR).  
The previous two papers
investigated the necessary design and possible physical implementation of 
such a self-replicator.
This paper compares potential man-made self-replication
to the existing natural self-replicators on Earth, present in the 
structured hierarchy of ecosystems throughout the world.
The insights offered by this series of papers are used to
reflect upon possible scenarios for the origin of life and their implications.
\end{abstract}


\section[Insights into the Design]{The Insights into the Design of the SSR, Its Complexity, and the Origin of Life}

By way of review, the following are some conclusions that can be made 
from Part 1 and Part 2 of this study of the design of the
Simplest Self-Replicator (SSR), the complexity of the SSR, and the feasibility of construction of
an artificial, autonomous SSR.

\subsection{The SSR has an overly complex design}

\index{self-replication!design|(}
The analyses from Part~1 and Part~2 demonstrated that even the Simplest Self-Replicator (SSR)
has an strikingly complex design. 
Beginning with the goal to design an SSR that can accurately reproduce itself and using logical, empirical, and systematic methods, it was shown that the artificial 
SSR must have a rich set of fully integrated advanced
capabilities (functions).  Any project or attempt to construct an
artificial SSR requires the employment of the most advanced engineering
techniques using the latest technologies.
\index{self-replication!design|)}

\subsection{There are many unknowns about the cell and its mechanisms}

\index{self-replication!challenges|(}
Scientists working in molecular biology, genetics, biotechnology,
bioinformatics, and related disciplines have made significant progress
in understanding the mechanisms of the cell and the information that
drives some of its activities. However, it is in the author's estimation
that scientists and engineers are still
at the beginning of a lengthy road to discover many of the remaining
mechanisms, information repositories and processes in living cells and
organisms. Here are some areas that are so far still hidden (at least
partially) from human knowledge:

\begin{itemize}
\item What are the mechanisms for information communication in the cell?  Some have been identified, but many still remain to be discovered.
\item Where in the cell is the information 
stored for building the  the cell's body plan, including the type, number, and orientation of the cell's organelles?  Additionally, will these organelles be linked to each other and other cellular structures? Finally, what is the specific, biochemical composition of each type of organelle?
\item How is the “supply-chain” function achieved in a cell, supplying the needed organic material building blocks for
protein and organelle construction at the right time?
\item What is the nature of the computations performed within the cell? Is
it based on protein/enzyme interactions only? Are there any other
forms of computations?
\item What are the inner mechanisms or control centers that drive cell
growth and cell division?
\end{itemize}
\index{self-replication!challenges|)}

Many of the necessary functions for self-replication
elucidated in Part~1 of this study have not been identified within
the cell yet, or have only been understood superficially.

\subsection{The material basis for an artificial SSR}

\index{self-replication!biological self-replication}
The most recommended approach for the design and construction of an
artificial SSR should choose a biochemical material basis and a
cell-like construction scale. However, the lack of knowledge of so many
cellular functions, and the technological limits to
operating at the cellular level of scale makes this approach
impractical and condemned to failure. The goal is to construct
an artificial SSR from inert chemicals
using a biological material basis, with full understanding and control
of all elements and mechanisms involved in such a construction. An
accomplishment like Craig Venter’s synthetic bacterial cell, although
remarkable, it is not at all at the level of achievement this would
imply, since Venter's project utilized an existing, living organism
as the scaffolding for his project.

\index{self-replication!macro-scale self-replication}
The alternative approach for building an artificial SSR is to take a
macro-scale approach using a
common manufacturing/engineering material basis.
Compared to what is evidently possible on a biochemical basis, even
the smallest miniaturized devices such as smart phones or miniature
motors are clunky by comparison.  In the analyses in Part~1 and
Part~2 of what is involved in the design and construction of even
a clunky artificial SSR, design requirements such as the energy
closure, material, closure, and information closure make the author
skeptical that even the most advanced labs in the world would be
able to design and construct a fully-functional SSR on a macro-scale.

\subsection{Summary of our findings}

\begin{enumerate}
\item There are many single celled living organisms that
\memph{are fully autonomous} and have a \memph{genuine ability to
self-replicate} achieving the energy closure, material
closure, and information closure requirements for self-replication.
\item Scientists are still at the beginning of the process of
fully understanding the design, information architecture,
and biochemical mechanisms of the living world, including
the simplest self-replicating single-celled organisms.  
\item It can be reasonably estimated that scientists and
engineers are not able with the current knowledge and technology to
create from scratch an artificial SSR with a biological
material basis both because of the lack of understanding of current single-celled
replicating organisms and because of the lack of current investigative,
operational, and constructive methods for manipulating and fabricating 
biological-scale artifacts.
\item It can reasonably be estimated that scientists and
engineers will encounter enormous difficulties in the
design and construction of a macro-scale autonomous SSR with a
non-biological material basis because of the difficulty
in satisfying the closure conditions at a reasonable scale.  No engineering
artifact with a complexity approaching that of the investigated SSR, with
its level of autonomy and complete automation, has ever been constructed.
\item The prior analyses revealed that the SSR's
ability to self-replicate is founded on a full assortment of
\memph{highly structured} information resident in the SSR and carried over
accurately to each descendent SSR.  The information stored in the SSR
is highly structured for the following reasons:

\begin{enumerate}
\item Each \memph{abstract concept} used by the SSR's design (raw
material, part, procedure, construction plan, etc.) is represented by a
\memph{catalog of entries}, each entry describing an instantiation of
that (abstract) category.
\item Each entry in a particular catalog (i.e. for a particular
abstraction) has a well-defined \memph{set of properties} describing
that type of object (entry).
\item There are many complex relationships between entries of
different catalogs, i.e., between represented abstractions. For example,
a part can be made from a particular material; a construction plan is
made from a sequence of procedures.
\end{enumerate}
\end{enumerate}

\index{origin of life}
The functional model of the SSR developed here has shown
that the SSR must be composed from a large number of well-defined
capabilities and mechanisms whose behavior must be integrated, synchronized, and
coordinated in their behaviors. An SSR cannot grow and duplicate if 
all these functions are not in place and fully functional. The SSR growth
and replication processes cannot be achieved with only one or a subset of the
required functions.  All must be in place from the beginning.
For example, it is not enough to have a fabrication function (RNA) if it
is missing the fabrication plan catalog (DNA). Or, even if it has both
the fabrication function and the fabrication plan catalog (RNA + DNA)
but is missing the input flow control function (membrane controlled
pores) and the division control function, the ``primitive SSR'' will not
be able to replicate.

The level of sophistication, autonomy,
self-sufficiency, and complexity of the simplest single
celled organisms is beyond the level of technological
sophistication achieved by humans thus far. 
The laws of nature cannot generate highly structured information, because
the kind of information needed for the catalogs needed by the SSR are
not repetitive or law-like.  Likewise, chance cannot be responsible for
such information because the catalogs must be finely tuned to deliver
accurate information, with small deviations leading to the possibility of
catastrophic consequences.
Given the rational structure and plan of the cell's internal
arrangement, its information, and its sophisticated
mechanisms, whose sophistication has only been partially
understood by humans because of their enormous complexity, could be the
result of natural processes---the laws of nature or random events.

The belief that the laws of
nature and any sequence of natural events and circumstances
could have created a self-replicating cell does note have a rational
foundation.  This author contends that it is purely a statement of faith without a defensible
scientific or empirical basis.
These considerations lead the author to the conclusion that a naturalistic
explanation of the origin of life is impossible.
It is unreasonable to conclude that while the smartest
human scientists and engineers are currently unable to design and
construct an artificial SSR from scratch because of its supreme
complexity, some random sequence of natural events could have produced
such a sophisticated self-replicator.

\section[From the Physics to the Metaphysics]{From the Physics to the Metaphysics of the SSRs}

\index{Paley, William}
In the last two hundred years, the human understanding of both technology and
biology was in its infancy.  Even then, however, William Paley had this
to say about what was known about biological mechanisms at the time:

\begin{quote}
SUPPOSE, in the next place, that the person who found the
watch, should, after some time, discover that, in addition to all the
properties which he had hitherto observed in it, it possessed the
unexpected property of producing, in the course of its movement,
another watch like itself (the thing is conceivable); that it contained
within it a mechanism, a system of parts, a mould for instance, or a
complex adjustment of lathes, files, and other tools, evidently and
separately calculated for this purpose; let us inquire, what effect
ought such a discovery to have upon his former conclusion.

The first effect would be to increase his admiration of the
contrivance, and his conviction of the consummate skill of the
contriver. Whether he regarded the object of the contrivance, the
distinct apparatus, the intricate, yet in many parts intelligible
mechanism, by which it was carried on, he would perceive, in this new
observation, nothing but an additional reason for doing what he had
already done,---for referring the construction of the watch to design,
and to supreme art.

William Paley, \textit{Natural Theology: or, Evidences of the Existence
and Attributes of the Deity}. Beginning of chapter II. \textit{State of
Argument Continued}\textstyleauthori{, 1809}%% FIXME - convert citation
\end{quote}

\index{origin of life}
\index{natural law}
\index{materialism}
\index{naturalism}
These analyses have shown that there are strong reasons to be skeptical that
humans are able, at this time, to design and build a fully autonomous
self-replicator. Furthermore, there are stronger reasons to be skeptical that
the self-replicators found on Earth are the results of
natural laws combined with random natural events or circumstances.

Scientists and engineers at NASA created
detailed plans and projects for creating artificial self-replicators to
be realized either as self-replicating moon factories or as
self-replicating inter-stellar probes.  The time horizon for the
implementation of these projects is either sometime during the
21\textsuperscript{st} century or in the distant future. 
These ambitious projects emphasize the technical hurdles that seem impossible to solve with current technology.

On the other hand, there is a vast assortment of
self-replicators populating the planet Earth. The estimated number of
organisms on Earth is between 10\textsuperscript{20} and
10\textsuperscript{30}, with an estimated 9.7 million varieties of
organisms on Earth. Among those varieties are bacteria, microbes,
fungi, plants, algae, grass, shrubs, trees, insects, mollusks, fish,
birds, and mammals. Some live in the seas---in 
lakes,  in rivers, and miles below the ocean's surface. Other
organisms live in ice, in the Earth’s crust, or on the Earth’s surface.
Some smaller organisms live in other organisms. 
All of these organisms constitute varying positions in the Earth's ecosystem, 
many providing energy for other biological SSRs.
Ocean
plankton serves as food for smaller fish and ocean creatures that, in
turn serve as food for larger fish or ocean mammals. On Earth, grass
and plants make up the food for rodents, animals and birds. It appears
that for each type of a self-replicator there is a particular food
niche suited for it. 
The three closure requirements--energy, material, and information---which 
are so difficult to achieve for an artificial SSR are routinely satisfied 
by the internal design and construction of all of these organisms.

\index{self-replication!advanced self-replicators|(}
Many of these organisms are not simple self-replicators. Many are significantly more complex than the artificial SSR  that was proposed in the prior papers. They are comprised of multiple cells and cell types. Additionally, while cell replication does occur, the organism, which is a structured hierarchy of systems of cells, tissues, and organs, replicates in a more sophisticated way at the whole organism level. This study did not consider the SSR’s mobility; however, most self-replicating organisms are mobile, which significantly facilitates their ability to feed and replicate. Finally, many self-replicating organisms are endowed with a wide spectrum of sensory organs that allows them to sense the environment.

How can one make sense of the presence of this plethora of
autonomous self-replicators? How can one explain the
existence of these self-replicators in tandem with a rational, hierarchical
structuring of their food chain and a harmonious integration into the
Earth’s environmental conditions?
\index{self-replication!advanced self-replicators|)}

This study has pointed to the immense amount of design and coordination \memph{required}
to produce even the simplest self-replicator.  This means that, at least from the origin of 
biological life, the Earth has had amazingly intricate machines.  As the explanation of
the material closure requirement shows, the environment for life to exist must 
contain suitable materials in a suitable state for use.  As the discussion of the 
information closure requirement reveals, the information to perform self-replication
had to exist from the very beginning.  As the exposition of the energy closure 
requirement discloses, each piece of the working self-replicator must be built to be 
tied in to the power distribution system.  This indicates that, rather than being built up
over time, life has been infused with design from the beginning.  The fact that these
intricate interconnections go beyond individual organisms and extend to entire ecosystems
indicates that the ecosystem itself, with self-replicators ranging from the smallest
single-celled organisms to such multicellular marvels as birds and mammals, operates
within the original plan.

\textit{Homo sapiens} occupies a unique place in this continuum of creatures, being
not only a product of the original design, but also a designer himself, with a mind
tuned to building
things like houses, roads, bridges, engines, cars, airplanes, and planetary exploration
vehicles.  
The mind of \textit{Homo sapiens}
looks in amazement to the Earth, at the Sun and planets, and at the galaxy
and beyond---it is a mind that dreams of conquering the galaxy.
This mind allows \textit{Homo sapiens} to not only operate within the grand design,
but also to examine and reflect on it himself, and become a builder and designer 
within the greater design.
With this mind he explores the plants, insects, birds, and animals in
the environment, and studies their make and behavior.  Further, this mind
allows him to look
into how these living organisms are constructed, investigating their inner workings.
With this mind \textit{Homo sapiens} has learned and is learning about the amazing
complexity of living organisms, and this author suggests that he
should remain in awe of the magnificent power that created the design at the beginning.
