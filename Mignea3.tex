\eandmchapter{The Simplest Self-Replicator, Part 3}{Developing Insights into the Design of the Simplest Self-Replicator and its Complexity: Part 3---The Metaphysics of an Artificial SSR and the Origin of Life Problem}{Arminius Mignea}{The Lone Pine Software}

\begin{abstract}
This paper is the last in a three-part series investigating the internals 
of the simplest possible self replicator (SSR).  
The previous two papers
investigate the necessary design and possible physical implementation of 
such a self-replicator.
This paper compares potential man-made self-replication
to the existing natural self-replicators on Earth, present in the 
structured hierarchy of ecosystems throughout the world.
The insights offered by this series of papers are used to
reflect upon possible scenarios for the origin of life and their implications.
\end{abstract}


\section[Insights into the Design]{The Insights into the Design of the SSR, Its Complexity, and the Origin of Life}

Here are some conclusions of our investigation into the design of the
SSR, the complexity of the SSR and the feasibility of construction of
an artificial, autonomous SSR as described in Part 1 and Part 2.

\subsection{The SSR has an overly complex design}

\index{self-replication!design|(}
Our analysis determined that even the Simplest Self-Replicator (SSR)
\textbf{has an overly complex design}. From its ability to reproduce
with accuracy we derived in a logical, empirical, systematic manner
that the SSR must have a rich set of fully integrated advanced
capabilities (functions).  Any project or attempt to construct an
artificial SSR requires the employment of the most advanced engineering
techniques and technologies.
\index{self-replication!design|)}

\subsection{Many unknowns about the cell and its mechanisms}

\index{self-replication!challenges|(}
Scientists working in molecular biology, genetics, biotechnology,
bioinformatics and related life disciplines made significant progress
in understanding the mechanisms of the cell and the information that
drives some of its activities. However, in our estimation we are still
at the beginning of a lengthy road to discover many of the remaining
mechanisms, information repositories and processes in living cells and
organisms. Here are some areas that are so far still hidden (at least
partially) from human knowledge:

\begin{itemize}
\item What are the mechanisms for information communication in the cell
(besides those already known)?
\item Where in the cell is the information about the “body plan” of the
cell stored? Information about what cell organelles need to be created
during cell replication? How many of each type? During construction,
where will they be placed in the space of the cell, and will there be
any “linkages” between them and other cell structures? What is the
composition in proteins and other organic components of each type of
organelle?
\item How is the “supply-chain” function achieved in a cell (the
supplying in time of the needed organic material building blocks for
protein and organelle construction)?
\item What is the nature of “computations” performed within the cell? Is
it based on proteins/enzymes interactions only? Are there any other
forms of computations?
\item What are the inner mechanisms/ control centers that drive the cell
growth and cell division?
\end{itemize}
\index{self-replication!challenges|)}

\subsection{A biological material basis for an artificial SSR?}

\index{self-replication!biological self-replication}
The most recommended approach for the design and construction of an
artificial SSR should choose a bio-chemical material basis and a
cell-like construction scale. However, the lack of knowledge of so many
things about the cell, our reduced abilities to investigate and to
operate at the cellular scale level makes the choice of this approach
totally impractical and condemned to a clear failure. We are
considering here the construction of an artificial SSR “from scratch”
using a biological material basis, with full understanding and control
of all elements and mechanisms involved in such a construction. An
accomplishment like Craig Venter’s synthetic bacterial cell, although
remarkable, it is not at all at the level of achievement this would
imply.

\subsection{A non-biological basis for an artificial SSR?}

\index{self-replication!macro scale self-replication}
The alternative approach for building an artificial SSR is to employ a
“macro” scale (it’s true at the minimum scale we can achieve) using a
common manufacturing/engineering material basis that is common for the
artifacts that the engineers know how to construct. Think for example
of electrical motors, metal/plastic robots, microprocessors, silicon
semiconductors, PCBs, plastics, ceramics, etc. This is because we know
how to build mechanical motors (even miniature ones) and NOT biological
motors. We know how to build “clunky” computers (even if they are small
like those in an iPad), but we don’t know how to build biological
computers or, in general biological machinery (here there might be some
insignificant exceptions).

Unfortunately, our analysis of what is involved in  the design and the
construction plans for a clunky artificial SSR, things like the energy
closure, material closure, information closure, the fact that humans
never built any fully automated, completely autonomous machinery (which
“knows” how to fabricate machines), make us quite skeptical that the
most advanced labs in the world, given some appropriate financial or
economic incentives, will be able to design and to construct a “clunky”
 artificial SSR using top-of-the-class technologies.

\subsection{Summary of our findings}

Let’s summarize our findings:

\begin{enumerate}
\item We know that there are many single celled living organisms that
\textbf{are fully autonomous} and have a \textbf{genuine ability to
self-replicate} achieving the \textbf{energy closure}, \textbf{material
closure} and \textbf{information closure} conditions.
\item Scientists \textbf{are still at the beginning of the process of
understanding} in full depth the design, the information architecture,
the bio-chemical mechanisms of the living world and in particular of
the simplest self-replicating single-celled organisms
\item It is estimated on a reasonable basis that scientists and
engineers are not able with the current knowledge and technology to
create from scratch \textit{an artificial SSR with a biological
material basis} for two fundamental reasons:

\begin{enumerate}
\item Because of Finding 2 above (lack of full understanding of the
single celled, self-replicating organisms).
\item Scientists and engineers are lacking significant investigative,
operational and constructive (manufacturing) methods for manipulating
biological matter or fabricating biological scale artifacts.
\end{enumerate}
\item It is estimated on a reasonable basis that scientists and
engineers will encounter enormous (unsolvable) difficulties in the
design and construction of a clunky autonomous SSR \textit{with a
non-biological material basis} for the following reasons:

\index{self-replication!closure requirements}
\begin{enumerate}
\item Extreme difficulty in satisfying the \textbf{energy closure}
condition in the artificial SSR 
\item Extreme difficulty in satisfying the \textbf{material closure}
condition in the artificial SSR 
\item Extreme difficulty in satisfying the \textbf{information closure}
condition in the artificial SSR
\item No engineering artifact with a complexity approaching that of the
investigated SSR, with its level of autonomy and complete automation
has ever been constructed
\end{enumerate}
\item Our study of the design of the SSR revealed that its
\textbf{ability to self-replicate is founded on a full assortment of
highly structured information} resident in the SSR and carried over
accurately to each descendent SSR.  The information stored in the SSR
is highly structured for the following reasons:

\begin{enumerate}
\item Each \textbf{abstract concept} used by the SSR design (raw
material, part, procedure, construction plan) is represented by a
\textbf{catalog of entries}, each entry describing an instantiation of
that (abstract) category
\item Each entry in a particular catalog (i.e. for a particular
abstraction) has a well-defined \textbf{set of properties} describing
that type of object (entry)
\item There are some \textbf{complex relationships between entries} of
different catalogs, i.e. between represented abstractions. For example,
a part can be made from a particular material; a construction plan is
made from a sequence of procedures.
\end{enumerate}
\end{enumerate}

\index{origin of life}
The set of findings above lead to the conclusion that a naturalistic
explanation of the origin of life is impossible for the following
reasons:

\textbf{Reason 1:}  Neither the laws of nature, nor random events can
generate highly structured information which we determined that must be
present in SSR (information catalogs).

\textbf{Reason 2:}  The functional model of the SSR we developed showed
that it must be composed from a well-rounded number of well-defined
capabilities and mechanisms that must be integrated, synchronized and
coordinated in their behaviors. An SSR cannot grow and duplicate if not
all these functions are all in place, fully functional. The SSR growth
and replication cannot be achieved with only one or a subset of the
required functions but rather all must be in place from the beginning.
For example it is not enough to have a fabrication function (RNA) if it
is missing the fabrication plan catalog (DNA). Or, even if it has both
the fabrication function and the fabrication plan catalog (RNA + DNA)
and is missing the input flow control function (membrane controlled
pores) and the division control function the “primitive SSR” will not
be able to replicate.

\textbf{Reason 3:} It is unreasonable to accept that the internal
arrangements, the information in the cell and its sophisticated
mechanisms that make rational sense, but have been only partially
understood by humans because they are overly complicated could be the
result of natural processes, the laws of nature or random events.

\textbf{Reason 4:} It is unreasonable to accept that while the smartest
human scientists and engineers will probably fail to design and
construct an artificial SSR from scratch because of its supreme
complexity, some random sequence of natural events could have produced
such a sophisticated self-replicator.

\textbf{Reason 5:} The level of sophistication, the level of autonomy
and self-sufficiency, the degree of complexity of the simplest single
celled organisms is much beyond the level of technology and engineering
sophistication achieved by humans so far. The belief that the laws of
nature and any sequence of natural events and natural circumstances
could have created a self-replicating cell has absolutely no rational
foundation and is a pure religious belief without a defensible
scientific or empirical basis

\section[Simplifying Assumptions]{Simplifying Assumptions for the Design and Construction of an SSR}

\index{self-replication!design!simplified|(}
Let’s consider some \textbf{significant simplifying assumptions} for the
design and construction of an artificial SSR as listed below:

\begin{enumerate}
\item Eliminate the requirement that the SSR produces its own energy.
The electrical energy (at an appropriate voltage/amperage) will be
supplied to the artificial SSR from outside.
\item Eliminate the requirement that the SSR has the ability to select,
identify and accept through its input gateway appropriate raw
materials. All raw materials will be supplied as stock materials to the
artificial SSR. Additional, optional simplification: all stock
materials are labeled appropriately (with bar codes or RFIDs labels for
example). However (as an illustration) the SSR will still need to use
stock copper fed through the input gateways to fabricate copper wires
of certain gauges, or to use copper in the fabrication of electrical
motor parts.
\item Eliminate the requirement that the SSR fabricate certain highest
(most demanding) technology parts, components and assemblies (like
computer boards, microprocessors, semiconductor chips and memories,
etc.) These high-technology parts/components (named in the
self-replication literature “vitamins”) will be supplied, carefully
labeled from the outside world through the input gateways.
\item Eliminate the requirement that the SSR must fabricate any part,
component, assembly or machinery from scratch. The SSR will be supplied
with already-manufactured parts that are used in the composition of all
its machinery, scaffolding and enclosure. This simplifying assumption
means that now the SSR needs to be designed as a (sophisticated)
self-assembler that achieves self-replication by assembling exact
copies of itself using an exhaustive pool of ALL the parts of all the
machinery/assemblies it is composed of from the supplied elementary
parts coming through its input gateways.
\item Eliminate the requirement that the information repositories
(information catalogs) that drive the functions of the SSR reside
within the SSR. This requirement need to be replaced with requirements
for the SSR to possess reliable, high speed communication capabilities
to access the information catalogs (and possibly part of the software)
residing somewhere outside the SSR. This assumption may simplify
certain elements of the SSR design but at the same time will make other
requirements (communication, availability) more stringent for both the
SSR and for the external information resource. 
\end{enumerate}

Even if the original requirements for the design and construction of an
autonomous, artificial SSR are relaxed and any or a combination of the
above Simplifying Assumptions are used as starting conditions for such
a project, there are still significant hurdles that need to be overcome
in designing and constructing an artificial SSR.
\index{self-replication!design!simplified|)}

\section[From the Physics to the Metaphysics]{From the Physics to the Metaphysics of the SSRs}

\begin{quote}

\index{Paley, William}
\textit{“SUPPOSE, in the next place, that the person who found the
watch, should, after some time, discover that, in addition to all the
properties which he had hitherto observed in it, it possessed the
unexpected property of producing, in the course of its movement,
another watch like itself (the thing is conceivable); that it contained
within it a mechanism, a system of parts, a mould for instance, or a
complex adjustment of lathes, files, and other tools, evidently and
separately calculated for this purpose; let us inquire, what effect
ought such a discovery to have upon his former conclusion.}

\textit{The first effect would be to increase his admiration of the
contrivance, and his conviction of the consummate skill of the
contriver. Whether he regarded the object of the contrivance, the
distinct apparatus, the intricate, yet in many parts intelligible
mechanism, by which it was carried on, he would perceive, in this new
observation, nothing but an additional reason for doing what he had
already done,-{}-for referring the construction of the watch to design,
and to supreme art…”}

William Paley, \textit{Natural Theology: or, Evidences of the Existence
and Attributes of the Deity}. Beginning of chapter II. \textit{State of
Argument Continued}\textstyleauthori{, 1809}%% FIXME - convert citation
\end{quote}

\index{origin of life}
\index{natural law}
\index{materialism}
\index{naturalism}
We have seen that we have very strong reasons to be skeptical that
humans are able at this time to design and build fully autonomous
self-replicators. We have much stronger reasons to be skeptical that
the self replicators that we encounter on Earth are the results of
natural laws combined with random natural events or circumstances.

We have seen that scientists and engineers at NASA or from other
organizations thought about the future of space exploration and created
detailed plans and projects for creating artificial self replicators to
be realized either as self-replicating moon factories or as
self-replicating inter-stellar probes.  The time horizon for the
implementation of these projects is either sometime during the
21\textsuperscript{st} century or farther on in the future. These
projects are very ambitious and detailed enough to emphasize technical
difficulties for their implementations that might be impossible to
solve with the current or near future technologies.

On the other hand we are facing the reality of a vast assortment of
self-replicators populating the planet Earth. The estimated number of
organisms on Earth is between 10\textsuperscript{20} and
10\textsuperscript{30}. There are an estimated 9.7 million varieties of
organisms on Earth. Among those varieties are bacteria, microbes,
fungi, plants, algae, grass, shrubs, trees, insects, mollusks, fish,
birds, mammals. Some live in the waters of the seas, in the water of
the lakes and rivers some live miles deep in the ocean waters. Other
organisms live in ice, in the Earth’s crust or on the Earth’s surface
or some smaller ones in other organisms. We know that the organisms of
the living world constitute layers in the food chain pyramid. Ocean
plankton serves as food for smaller fish and ocean creatures that, in
turn serve as food for larger fish or ocean mammals. On earth, grass
and plants make up the food for rodents, animals and birds. It appears
that for each species of a self-replicator there is a particular food
niche for that species. From our SSR insights we can affirm that the
energy closure, material closure and information closure – so difficult
to achieve for an artificial SSR – are common, ordinary conditions that
are satisfied by the internal design and construction of all these
species of organisms.

\index{self-replication!advanced self-replicators|(}
Many of these organisms are not simple self-replicators. Many are
significantly more complex than the SSR we investigated and profiled.
They are significantly more complex for various reasons:

\begin{itemize}
\item They are made of multiple cells and multiple cell types. 
\item Although the cell still self-replicates, the organism which is a
structured hierarchy of systems of cells, tissues and organs has a more
sophisticated way to replicate sexually at the whole organism level 
\item Many complex self-replicators are mobile and their mobility in
their medium significantly facilitate their ability to feed and
replicate.
\item Many self-replicators are endowed with a wide spectrum of sensor
organs that allow them to sense the environment either visually,
through smells, through taste, through touching, through sensing
movements or through sensing sounds.
\end{itemize}

How can we make sense of the presence of this immense plethora of
autonomous self-replicators? How can we make sense and explain the
existence of these self-replicators, with rational hierarchical
structuring of their food chain and harmonious integration in the
Earth’s environmental conditions?
\index{self-replication!advanced self-replicators|)}

%% FIXME - why is this italicized?
\textit{The author suggests the following explanation for
consideration.}

\index{origin of life}
\index{self-replication!implications}
An extremely advanced civilization visited the planet Earth sometime in
the past. This civilization had a Master Designer.  The Master Designer
knew how to design and build self-replicators. He knew that the
biochemistry is the appropriate material basis with the appropriate
scale for the design and construction of these self replicators. He
knew how to use just a few chemical elements: hydrogen, carbon, oxygen,
nitrogen, phosphorus and a few others as the building blocks of an
amazing variety of molecular machinery. He knew how to codify and store
information in the DNA and other complex organic molecules. The Master
Designer populated the Earth with an immense number of self-replicators
from the smallest to the largest. From those that are at the base of
food chain, to those that are at the top of the food chain. From
self-replicators that are made of a single cell to birds, mammals and
the Homo Sapiens, the crown of the Master Designer creation.  He
endowed the Homo Sapiens with a body and a mind. A mind that the Homo
Sapiens uses to contemplate his surroundings to design his own things
like houses, roads, bridges, engines, cars, airplanes and planetary
exploration vehicles. A mind that he uses to explore the plants,
insects, birds and animals in the environment, to study their make and
their behavior.

A mind that started looking into how these living organisms are
constructed and what are the most inner workings of them. A mind that
looks in amazement to our Earth, to our Sun and planets, to our galaxy
and much beyond. A mind that dreams to conquer the galaxy but cannot
stop to ask what advanced civilization left its indelible creative
marks on planet Earth. A mind that can understand the amazing
complexity of the living organisms and remain in awe in the face of the
magnificent creative power of the Master Designer reflected in the
myriad results of his work.
