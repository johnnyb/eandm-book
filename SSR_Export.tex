% This file was converted to LaTeX by Writer2LaTeX ver. 1.0.2
% see http://writer2latex.sourceforge.net for more info
\documentclass[letterpaper]{article}
\usepackage[utf8]{inputenc}
\usepackage[T1]{fontenc}
\usepackage[english]{babel}
\usepackage{amsmath}
\usepackage{amssymb,amsfonts,textcomp}
\usepackage{color}
\usepackage[top=0.5in,bottom=0.5in,left=0.5in,right=0.5in,includehead,head=0.5in,headsep=0.4602in,includefoot,foot=0.5in,footskip=0.9602in]{geometry}
\usepackage{array}
\usepackage{supertabular}
\usepackage{hhline}
\usepackage{hyperref}
\hypersetup{pdftex, colorlinks=true, linkcolor=blue, citecolor=blue, filecolor=blue, urlcolor=blue, pdftitle=Linear Services Publishing}
\newcommand\textsubscript[1]{\ensuremath{{}_{\text{#1}}}}
% Text styles
\newcommand\textstyleBookTitle[1]{\textbf{\textsc{#1}}}
\newcommand\textstyleTitleChar[1]{\textbf{#1}}
\newcommand\textstyleauthori[1]{#1}
\newcommand\textstylecitation[1]{#1}
\newcommand\textstylepagenumber[1]{#1}
\makeatletter
\newcommand\arraybslash{\let\\\@arraycr}
\makeatother
% Footnote rule
\setlength{\skip\footins}{0.0469in}
\renewcommand\footnoterule{\vspace*{-0.0071in}\setlength\leftskip{0pt}\setlength\rightskip{0pt plus 1fil}\noindent\textcolor{black}{\rule{0.25\columnwidth}{0.0071in}}\vspace*{0.0398in}}
% Pages styles
\makeatletter
\newcommand\ps@Convertedi{
  \renewcommand\@oddhead{\hfill \hfill Design of the Simplest Self Replicator}
  \renewcommand\@evenhead{\@oddhead}
  \renewcommand\@oddfoot{\textbf{\textit{Revision 0.4}}\textit{\hfill Page }\thepage{}\textstylepagenumber{\textit{ of }}?\textstylepagenumber{\textit{\hfill }}April 23, 2013}
  \renewcommand\@evenfoot{\@oddfoot}
  \renewcommand\thepage{\arabic{page}}
}
\newcommand\ps@Standard{
  \renewcommand\@oddhead{}
  \renewcommand\@evenhead{}
  \renewcommand\@oddfoot{\textbf{\textit{Revision 0.4}}\textit{\hfill Page }\thepage{}\textstylepagenumber{\textit{ of }}?\textstylepagenumber{\textit{\hfill }}April 23, 2013}
  \renewcommand\@evenfoot{\@oddfoot}
  \renewcommand\thepage{\arabic{page}}
}
\makeatother
\pagestyle{Standard}
\setlength\tabcolsep{1mm}
\renewcommand\arraystretch{1.3}
\title{Linear Services Publishing}
\begin{document}
\clearpage\setcounter{page}{1}\pagestyle{Standard}

\bigskip


\bigskip


\bigskip


\bigskip


\bigskip


\bigskip


\bigskip


\bigskip


\bigskip


\bigskip


\bigskip

\section[Developing Insights into the Design of the Simplest
Self{}-Replicator (SSR) and its
Complexity]{\textstyleBookTitle{\textup{Developing Insights into the
Design of the Simplest Self-Replicator (SSR) and its Complexity}}}
\hypertarget{RefHeading3030306210128}{}\hypertarget{Toc349682451}{}
\bigskip

\clearpage\setcounter{page}{1}\pagestyle{Convertedi}
{\bfseries\color[rgb]{0.21176471,0.37254903,0.5686275}
Table of Contents}

\setcounter{tocdepth}{3}
\renewcommand\contentsname{}
\tableofcontents

\bigskip

{\bfseries\color[rgb]{0.21176471,0.37254903,0.5686275}
Table of Figures}

\listoffigures
\clearpage{\bfseries
\textstyleTitleChar{Developing Insights into the Design of the Simplest
Self-Replicator (SSR) and its Complexity}}


\bigskip

Arminius Mignea\textsuperscript{1}


\bigskip

\textsuperscript{1}The Lone Pine Software, San Jose, California

{\bfseries\color[rgb]{0.21176471,0.37254903,0.5686275}
Abstract}


\bigskip

This paper investigates the internals of the simplest-possible self
replicator (SSR). The SSR is defined as having an enclosure with input
and output gateways and having the ability to create an exact replica
of itself by just ingesting and processing materials from its
environment. The study takes an empirical approach and identifies one
by one the internal capabilities or functions that must operate inside
the SSR to provide its fully autonomous replication behavior. The
analysis considers various significant aspects that confronts the
design and construction of an artificial, concrete SSR: the material
basis of its construction; the effects of the variable geometry of the
SSR during its growth through the cloning and then division phases; the
three closure rules that must be satisfied by the SSR: energy closure,
material closure and the information closure. The highest technical
challenges that need to be faced by the design and construction of the
artificial SSR are discussed. The emerging complexity of the artificial
SSR is depicted using a metaphorical comparison of the replicating SSR
with a full city populated only by automated machinery and robots that
systematically and orderly construct the new city quarters identical
with the old city quarters with no help from outside but only the
construction materials entering through the city gateways. An
evaluation is made if the current level of technology is good enough
for the successful completion of a “design and construct an artificial
autonomous SSR” project either with a nano biochemical basis or a macro
material basis. The insights of the internal SSR design are used to
reflect upon the search for a materialist explanation of the origin of
life. The paper concludes contrasting the futuristic NASA projects for
a self-replicating factory on the Moon and an inter-stellar probe for
population of the galaxy with the presence on planet Earth of a
structured hierarchy of eco systems of self-replicators that provide
feeding niches for millions of varieties of its self-replicators. 


\bigskip

{\bfseries\color[rgb]{0.21176471,0.37254903,0.5686275}
The Topic Coverage in Three Parts}


\bigskip

The topic in the title is covered in three parts that focus on different
aspects of gaining insights on the design and complexity of the
Simplest Self Replicator (SSR):


\bigskip

\begin{enumerate}
\item Developing a Functional Model for the Simplest Self-Replicator.
\end{enumerate}

\bigskip

\begin{enumerate}
\item Evaluating the Complexity of an Exhaustive Design and of a
Concrete Implementation of an Artificial SSR
\end{enumerate}

\bigskip

\begin{enumerate}
\item The Metaphysics of an Artificial SSR and the Origin of Life (OOL)
Problem
\end{enumerate}

\bigskip

\clearpage\section[PART I]{PART I}
\hypertarget{RefHeading3032306210128}{}\section[Developing a Functional
Model for the Simplest Self{}-Replicator]{Developing a Functional Model
for the Simplest Self-Replicator}
\hypertarget{RefHeading3034306210128}{}
\bigskip

{\bfseries
\hypertarget{RefHeading3036306210128}{}Introduction}


\bigskip

One of the most remarkable characteristics of the living organisms is
their ability to self-replicate. There are many forms and
manifestations of the self-replication capability, and these forms vary
from the simplest, single-celled organisms through a wide range of
living forms up to the most complex organisms including humans and
other mammals.


\bigskip

\subsection[In search of the Simplest Self{}-Replicator (SSR)]{In search
of the Simplest Self-Replicator (SSR)}
\hypertarget{RefHeading3038306210128}{}
\bigskip

One of the most intriguing questions that ordinary people, engineers,
scientists and philosophers obsessed over for centuries was how life
originated and how this ability of living organisms to create
descendants that look like their parents appeared on planet Earth. 


\bigskip

\subsection[Efforts to trace a materialistic emergence of life and self
replicators]{Efforts to trace a materialistic emergence of life and
self replicators}
\hypertarget{RefHeading3040306210128}{}Many researchers and scientists
are investing tremendous resources, in trying to identify any plausible
natural means by which the simplest forms of life may have been created
from inanimate matter. They are trying to identify and hopefully
reproduce a lucky set of events and circumstances that somehow put
together the basic elements and parts of the simplest entity that began
to manifest the ability to replicate and thus become a living organism.


\bigskip

{\bfseries
\hypertarget{RefHeading3042306210128}{}Goals, Assumptions and
Requirements}

The goal of this study was to use an engineering approach in developing
insights into the internal design of a simplest possible self
replicator (SSR). The SSR is defined for the purpose of this study as
an autonomous artifact that has the ability to get material input from
its environment, grow and create an exact replica of itself. The
replica should “inherit” the ability from its “mother” SSR to create in
its turn an exact copy of itself.


\bigskip

It is important to observe that this simple definition of the SSR mimics
accurately the characteristic behavior of many single cellular
organisms – at least from the point of view of their ability to
self-reproduce. In particular they are autonomous in regards to:


\bigskip

\begin{itemize}
\item their ability to ingest materials from their environment
\item their ability to use the ingested materials for growth and
production of internal energy
\item their ability to produce an identical copy of themselves usually
through a two-step process of cloning and division.
\end{itemize}
The more specific work hypothesis for this research can be described in
the following terms. A hypothetical research lab is launching a project
to design and build an artificial SSR. The first phase of this project
is to ask a team of engineers to perform a preliminary study with the
following objectives:


\bigskip

\begin{enumerate}
\item  Create a top level design of the SSR that identifies all its
functional components with specific characterization of the role of
each functional component, its responsibilities and its interactions
with the other functional components within the SSR framework.
\end{enumerate}

\bigskip

\begin{enumerate}
\item  Identify the candidate engineering technologies to be employed in
the concrete implementation of the SSR and of all its functional
components.
\end{enumerate}

\bigskip

\begin{enumerate}
\item  Identify the highest level of difficulties in constructing the
concrete artificial SSR
\end{enumerate}

\bigskip

\begin{enumerate}
\item  Conclude with an overall estimate of the complexity of the
project to construct the artificial SSR. For a pragmatic estimation of
SSR construction complexity, it may be compared with selected,
existing, top technology human-made artifacts.
\end{enumerate}
\section[]{\color{black} }

\bigskip

{\bfseries
\hypertarget{RefHeading3044306210128}{}The Two Phases of the Self
Replication}

At the highest level the SSR has the following composition as
illustrated in 1 below.

 1

{\bfseries
\label{bkm:Ref331099850}Figure  The SSR structure}


\bigskip

The SSR replication has two main phases:

\begin{enumerate}
\item The cloning phase is illustrated in  The Cloning phase in SSR
replication
\item The division phase is illustrated in  The Division phase in SSR
replication
\end{enumerate}

\bigskip

{\bfseries
\label{bkm:Ref329890094}Figure  The Cloning phase in SSR replication}


\bigskip


\bigskip

{\bfseries
\label{bkm:Ref329890734}Figure  The Division phase in SSR replication}


\bigskip

The behavior of the SSR can be described succinctly as follows:

\begin{itemize}
\item Input raw materials and raw parts are accepted by input enclosure
gates
\item Input raw materials are processed through material extraction into
good materials for fabrication of parts or for energy generation
\item Energy is generated and made available throughout the SSR
\item Fabrication function starts to fabricate parts, components and
assemblies for: 

\begin{itemize}
\item Cloning (creating copies) of all SSR internal elements
\item Creating scaffolding elements for the growing SSR interior
\item Creating new elements that are added to the growing enclosure
\end{itemize}
\item When the cloning of all original SSR internal parts is completed,
the SSR division starts:

\begin{itemize}
\item The original SSR content is now at (for example) the “north pole”
of the SSR enclosure
\item The cloned SSR content (the “nascent daughter SSR”) is now at the
“south pole” of the SSR enclosure
\item The SSR enclosure and its content now divide at the “equatorial”
plane and the separate “mother” (at North) and daughter (at South) SSR
emerges.
\end{itemize}
\end{itemize}
{\bfseries
\hypertarget{RefHeading3046306210128}{}Identifying SSR capabilities as
specific functions}

\subsection[An empirical approach]{An empirical approach}
\hypertarget{RefHeading3048306210128}{}We will build step by step some
understanding of what must be the elements and the abilities of the
SSR, by conducting an analysis of what must be happening inside and at
the periphery of the SSR so that its growth and replication will occur.


\bigskip

\subsection[The SSR enclosure and its input gateways and output
gateways]{The SSR enclosure and its input gateways and output gateways}
\hypertarget{RefHeading3050306210128}{}We will assume that the SSR is
made by an enclosure that has the role to separate the SSR from its
environment. We will assume that on this enclosure there are some
openings (gateways) that are used to accept inside SSR some good
substances: raw materials and raw parts from the SSR environment. We
will call these openings \textbf{input gateways}. There are also
openings used by the SSR to push out from inside some refuse materials
and parts that result from certain transformation/fabrication processes
inside the SSR. We will name these openings \textbf{output gateways}.


\bigskip

\subsection[The input flow control function]{The input flow control
function}
\hypertarget{RefHeading3052306210128}{}There is one relevant question
for the input gateways. Are all the raw materials and raw parts that
exist or touch the outside of the enclosure good for the SSR processes?
Certainly they are not. SSR and its input gateways must feature some
ability to be selective in accepting or rejecting substances, materials
and parts that are outside and may enter the SSR interior. We will
identify this needed feature of the SSR as its \textbf{input flow
control function.}


\bigskip

\subsection[The raw materials and parts catalog]{The raw materials and
parts catalog}
\hypertarget{RefHeading3054306210128}{}The next question to be answered
is how the SSR “knows” which are “good” raw materials and “good” parts
versus “bad” materials and parts? The answer is that the SSR must
possess a catalog of good raw materials and parts and this will be the
informational basis on which the input gateways will open or stay
closed. Let’s name this catalog the \textbf{raw materials and parts
catalog.}


\bigskip

\subsection[The materials and parts identification function]{The
materials and parts identification function}
\hypertarget{RefHeading3056306210128}{}The next question is how the SSR
will recognize and accurately identify a material or part at an input
gateway as a good or bad one? That is not a trivial ability, but the
SSR must possess it. This ability can be described as a way to
determine the nature of materials and parts its input gateways are
exposed to. This ability may be supported by a set of material probing
procedures and processes. We will name this SSR ability the
\textbf{materials and parts identification function}. This capability
is at least on the order of complexity of certain probes with which the
Martian rover was equipped to analyze the Martian soil in order to
decide if certain compounds are present or not on the surface of that
planet.


\bigskip

\subsection[The systematic labeling/tagging of all raw materials, raw
parts and fabricated parts and components]{The systematic
labeling/tagging of all raw materials, raw parts and fabricated parts
and components}
\hypertarget{RefHeading3058306210128}{}The next issue that requires
clarification is the following. Let’s assume that an input gateway
assisted by the raw materials and parts identification function
determined that a piece of raw material is one of the good materials
recorded in the good raw materials and parts catalog. This piece is
going to be admitted inside the SSR and transported to a certain place
for processing or possibly for temporary storage followed by
processing. It appears that a good practice to be employed by the SSR
would be to tag or label this piece somehow so that its nature, once
determined at the input gateway, is well known and available for
subsequent processing stations or storage stations in the SSR. We will
now assume that any raw material or part accepted inside the SSR, once
its nature is identified, is immediately tagged/labeled using a system
similar to the bar codes or RFIDs (radio-frequency identification)
where the code used is one of the codes in the catalog of raw materials
and parts. This systematic labeling and tagging of all accepted
materials and parts will be considered another responsibility of the
\textbf{materials and parts identification function. }More than that,
we will assume that in the process of SSR growth and during the cloning
phase, the SSR will fabricate new parts, components and assemblies
using either raw materials and raw parts or previously fabricated
parts, components and assemblies. The point is that the SSR and the
\textbf{materials and parts identification function }in
particular\textbf{ }should be responsible to\textbf{ }tag/label not
only raw materials and parts accepted inside SSR but also all
fabricated parts, components and assemblies. This allows us to affirm
that all elements present inside the SSR and all SSR parts should bear
a permanent identification/tag.


\bigskip

\subsection[The catalog of fabricated parts, components and
assemblies]{The catalog of fabricated parts, components and assemblies}
\hypertarget{RefHeading3060306210128}{}This raises also another
important aspect for the SSR design. The SSR must possess not only an
exhaustive catalog of all raw materials and raw parts, but also a
\textbf{catalog of all fabricated parts, components and assemblies
}with a unique identifier for each \textbf{type} of such element. 


\bigskip

\subsection[The bill of materials function]{The bill of materials
function}
\hypertarget{RefHeading3062306210128}{}The automated fabrication
processes will need additional informational support. We will name this
capability the \textbf{bill of materials function}. It is supported by
an exhaustive catalog that has entries that specify for each and every
fabricated part, component and assembly what is the list of materials
and parts that is needed to fabricate that part. For each item in this
list the count or quantity of that part or material must be also
specified. For example, for a “power supply enclosure” part with
identifier:  “ID-50712294”, its entry in the bill of materials catalog
may look like the table below:


\bigskip

{\bfseries
Table  Example Entry in the Bill of Material Catalog for the
{\textquotedbl}Power Supply Enclosure{\textquotedbl} part}

The flags: Q=quantity, C=count, D=dimensions specify what properties are
specified for each part name


\bigskip

\begin{flushleft}
\tablehead{}
\begin{supertabular}{|m{1.3094599in}m{1.3573599in}m{0.9837598in}m{0.6705598in}m{0.82335985in}m{1.0323598in}|}
\hline
\textbf{Part name} &
\textbf{Part Identifier} &
\textbf{Flags} &
\textbf{Count} &
\textbf{Quantity} &
\textbf{Dimensions(“)}\\
Sheet metal  1/16 &
~
 &
~
 &
{}- &
~
 &
~
\\
Screws 1/8*2 &
~
 &
~
 &
8 &
~
 &
~
\\
Washers  ¼*2 &
~
 &
~
 &
8 &
~
 &
~
\\\hline
\end{supertabular}
\end{flushleft}

\bigskip

The \textbf{bill of materials} is an \textbf{informational function} of
the SSR. Like all SSR informational functions it has two components:


\bigskip

\begin{itemize}
\item A specific catalog (or database) – in this case the bill of
material catalog
\item A set of information access sub-functions to search, read, write,
update or delete specific entries in the associated catalog. This set
of sub-functions can be accessed by all other SSR functions 
\end{itemize}
\subsection[The fabrication material extraction function ]{The
fabrication material extraction function }
\hypertarget{RefHeading3064306210128}{}Some of the raw materials
admitted inside the SSR cannot be directly used by the SSR fabrication
processes. They need to be transformed into “fabrication materials”
through one or more specific processes. As an example – which may not
need to have application inside the artificial SSR - is the fabrication
of steel (as a “fabrication material”) from iron ore and coal (both of
these are in this case “raw materials”). The ability of the SSR to
extract fabrication materials from sets of raw materials and parts is
the \textbf{fabrication material extraction function}.  Fabrication
materials are registered in the \textbf{catalog of fabrication
materials }while every procedure and technological process that is used
to extract fabrication materials from raw materials and parts is
documented in a \textbf{fabrication material extraction process
catalog}.


\bigskip

\subsection[The supply chain function]{The supply chain function}
\hypertarget{RefHeading3066306210128}{}Let’s now assume that during
cloning phase the SSR must fabricate a component of type A.  The bill
of materials entry for component of type A specifies that its
“fabrication recipe” requires 2 raw parts of type X and 4 raw parts of
type Y.  The SSR will need a capability to coordinate the input
gateways to admit ahead of time the required counts of parts X and Y,
and create some stock in a SSR “stock room” so that the fabrication of
components A can go smoothly and depend a little less on what raw parts
are available at any given moment at any one of the enclosure input
gateways. We are going to name this SSR capability the \textbf{supply
chain function}. This function is thus responsible to interact with
fabrication processes inside SSR, to gather information ahead of
fabrication time, on what raw materials, raw parts or even fabricated
parts are needed and command the input flow control function, the
material and parts identification function to admit, supply and stock
those elements inside the SSR.


\bigskip

\subsection[The energy generation and distribution function]{The energy
generation and distribution function}
\hypertarget{RefHeading3068306210128}{}All the machines inside SSR need
a source of energy to perform their work. Thus we must have inside the
SSR a capability to produce energy from raw materials and raw parts
that are appropriate for energy generation. This SSR capability will be
named the \textbf{energy generation and distribution }function since it
has the responsibility not only to generate energy but also to manage
and distribute it to all energy consumers inside the SSR. It is
important to note that the catalog of raw materials and raw parts as
well as the catalog of fabricated parts may contain entries that will
be marked as elements used for energy generation and/or distribution.
Also the \textbf{catalog of processes} will contain entries identifying
those material processes (and their detailed description) that are used
to generate energy from the energy-marked materials and parts. The
supply chain function is responsible to manage the timely supply of
materials and parts not only for fabrication but also for energy
generation and transport.


\bigskip

\subsection[The transport function]{The transport function}
\hypertarget{RefHeading3070306210128}{}The functioning SSR features
multiple “sites”, i.e. places where specific actions happen. These
sites will be distributed spatially inside the SSR, on its enclosure
and have well established positions relative to the elements that
maintain the three dimensional (and growing) structure of the SSR –
named scaffolding elements. For example, there are input gateway sites,
there are possibly some stock room sites, there are fabrication sites
and there are assembly sites where the elements of the growing clone
inside the SSR are being put together by some machinery. The SSR must
feature a capability to carry various elements between sites. This
capability is named the \textbf{transport function}. It may employ
specific means of transport, like using conduits, avenues, conveyors,
etc. that are adequate for the nature of elements being transported and
the placement of the sites inside SSR.  We will see later that an
important aspect of SSR activity is the transport of information
between “producers” and “consumers” inside the SSR. For this reason the
transport function is responsible to \textbf{transport also
information} between SSR sites (at least for the provision of the
physical, lower layers of the transport of information).


\bigskip

\subsection[The manipulation function]{The manipulation function}
\hypertarget{RefHeading3072306210128}{}Another important capability that
is needed inside the SSR is the \textbf{manipulation function}. This
function consists of the ability to handle, grab or manipulate raw
materials, raw parts, fabricated parts, fabricated components and
fabricated assemblies. For example this manipulation ability is needed
to take a raw material or raw part admitted at an input gateway and
place it on a conveyor that has as destination a stock room or a
fabrication site. There another “manipulator” will grab the material or
part and place it in a specified position in the stock room or place it
on a fabrication bench or machinery. Manipulation examples abound,
since no matter what elements are processed, transported, fabricated,
assembled or pushed out of an output gateway there is a need to
adequately handle those elements.


\bigskip

\subsection[The fabrication function]{The fabrication function}
\hypertarget{RefHeading3074306210128}{}Since the SSR must be able to
clone its core elements, its enclosure and its scaffolding elements,
there is an absolute need for the SSR to have a \textbf{fabrication
function}. This function is the ability to fabricate exact copies of
all parts, components and assemblies that exist inside a “mature” SSR
or on its enclosure. In other words the fabrication function must be
able to fabricate all machinery inside the SSR, including fabrication
machinery. Since all the elements that reside inside the SSR must be
copied (cloned) and various types of information and software elements
(as we will see later) also reside inside that “mature” SSR, results by
implication that the fabrication function may have also the ability to
accurately copy information and software (since the information or
software cannot be fabricated “per se” and assuming that the detailed
design of the artificial SSR includes software as the nature of certain
fabricated components of the SSR).


\bigskip

\subsection[The assemblage function]{The assemblage function}
\hypertarget{RefHeading3076306210128}{}Another capability that must
reside inside the SSR is the \textbf{assemblage function}. This is the
ability of the SSR to assemble or put together parts into
increasing-in-complexity components and assemblies. The assemblage
function is strongly related to the fabrication function. These two
functions can be seen as the two faces of the same coin. It makes sense
though to see them as distinct functions where the fabrication function
creates new parts from raw materials and raw parts through special
processes (example metal machining) while the assemblage function puts
together fabricated parts into more and more complex assemblies of
parts and components. The assemblage function may be needed for example
to erect and expand the scaffolding and the enclosure during the SSR
growth besides being used to create assemblies of smaller components
and parts.


\bigskip


\bigskip

{\bfseries
\hypertarget{RefHeading3078306210128}{}Additional SSR functions}

\subsection[The recycling function]{The recycling function}
\hypertarget{RefHeading3080306210128}{}The SSR functions discussed so
far provide specific assistance for the “ingestion” of new raw
materials and parts into the SSR and the SSR growth during the phase of
cloning – based on continuous production of energy and on planned
fabrication of the elements of the “daughter” clone growing inside the
expanding SSR enclosure. As in any process that performs fabrication
and construction of new parts, there will be “residue” raw materials
and raw part fragments.  The SSR must be designed to carefully control
the growth of the inside and enclosure elements of the SSR. It cannot
grow without limits or in an uncontrolled manner. In order to achieve
this objective the SSR must:


\bigskip

\begin{itemize}
\item Re-introduce in the fabrication and growth cycles certain elements
of the “residue” raw materials, raw parts or raw part fragments that
can be reused.
\item Identify and specifically mark as refuse certain residue elements
that cannot be recycled and that are then pushed as refuse through the
output gateways.
\item Provide specific processes to “clean” and “tidy up” the SSR
interior fabrication and transport spaces, such that fabrication and
assembly of the cloned parts is not stopped or affected and the SSR
maintains “proper structure” during the cloning and division phases.
\end{itemize}
Most of the above responsibilities pertain to the \textbf{recycling
function.}


\bigskip

\subsection[The output flow control function]{The output flow control
function}
\hypertarget{RefHeading3082306210128}{}The recycling function controls
the \textbf{output flow control function }which is the ability to
control the enclosur\textbf{e output gateways }for forcing out the SSR
the raw materials, raw parts and raw part fragments \textbf{marked as
refuse} by the recycling function.


\bigskip

\subsection[The construction plan function]{The construction plan
function}
\hypertarget{RefHeading3084306210128}{}We already saw that the bill of
materials function provides for each fabricated part, component or
assembly of the SSR a list of raw materials, parts and sub-components
that are needed for fabrication of that element. Thus an entry in the
bill of materials information catalog is similar in concept with the
“list of ingredients” for cooking a meal. Besides the list of
ingredients the recipe for a meal contains an ordered “list of steps”
needed to prepare the meal. In a similar manner, the SSR must store
descriptive information for all construction steps needed to fabricate
each SSR element. This capability of the SSR to store and make
accessible detailed information about the set of fabrication steps and
processes needed for the fabrication of each SSR element (part,
component, assembly of components, up to including the SSR itself) is
called the \textbf{construction plan function}.


\bigskip

\subsection[The construction plan information catalog]{The construction
plan information catalog}
\hypertarget{RefHeading3086306210128}{}The \textbf{construction plan
function} has an associated \textbf{construction plan information
catalog}. This catalog has an entry for each fabricated element of the
SSR. A fabricated element can be a simple fabricated part (fabricated
from a single good material). A fabricated element can also be a
component which, in this context, refers to an element fabricated
through assemblage of two or more fabricated parts. A fabricated
assembly is even more complex: it is fabricated from multiple simple
parts and one or more components and possibly one or more (sub)
assemblies. The mature SSR (before starting the cloning process) is a
particular case of an assembly. Another example of an assembly (most
complex in this case) is the SSR grown to contain both the mother core
elements and the daughter elements just before division starts. 


\bigskip

Each entry in the construction plan catalog contains the following
information items:


\bigskip

\begin{itemize}
\item A reference to the entry in the bill of materials catalog for the
same element (to access the “ingredients” needed for the element
fabrication)
\item An ordered sequence of fabrication and assembly steps.
\end{itemize}
For each fabrication/assembly step the following information may be
provided:


\bigskip

\begin{itemize}
\item The spatial assembly or placement instruction of parts/materials
involved in the step ( how to spatially place a part/component P
relative to the assembly under construction on the work bench, before a
fabrication/technology step)
\item The type and detailed description of each fabrication/assembly
process executed during the current fabrication step.
\item Technological parameters of fabrication/assembly process (like
ambient temperature, length of process, etc.)
\item List of “residue” parts/materials resulted from the process/step.
\item Part manipulation steps (with x, y, z starting point, x, y, z
ending point, any part rotation, with axis specification or translation
movement).
\item Any fabrication step verification procedure – to determine if the
step completed successfully – within the accepted parameters, or the
fabrication step was a failure.
\item Recovery action list in case a fabrication step fails with a
specific verification error.
\end{itemize}
In short, the rationale for the nature, structure and the extent of
information items stored for each step of a \textbf{fabrication plan
entry} is to provide support for full automation of that element
fabrication. And, as suggested above \textbf{fabrication steps} can be
of a very large variety. The nature of the fabrication process as well
as the nature of fabrication steps depends on the material basis to be
selected for the design and implementation of the artificial SSR. The
alternative material bases that can be realistically considered for
creating an artificial SSR are discussed in the Part II of this study.
To help understand the nature of a fabrication step here are several
examples:


\bigskip

\begin{itemize}
\item A part or component being placed in a particular position relative
to the semi-assembled element – in preparation for the next joining
(welding, screwing) operation or process (diffusion, thermal treatment,
metal machining)
\item An assemblage step. 
\item A metal machining kind of fabrication step
\item A chemical process step
\item A thermal process step
\item An electrolytic process step
\item A nano-technology assemblage step
\item An information file copy step
\item A fabricated component (assembly) test step
\end{itemize}
The next issue that requires a solution is to devise a way to track the
progress of the cloning and division steps. This is provided by the
\textbf{construction status function}. This function uses an
information catalog similar with the construction plan catalog named
\textbf{construction status catalog}. It is similar in the sense that
it has the same list of element entries as the construction plan
catalog describing the same hierarchical composition of each element
(part, component, assembly) in sub-elements.  Each entry in this
catalog has construction status information that reflects the current
construction status of that entry and can have values like:


\bigskip

\begin{itemize}
\item Not-started
\item Started
\item Completed
\end{itemize}
In a similar manner each fabrication step of an entry has a current
construction status field that is also used to mark and keep track of
the fabrication/construction status for that element at the fabrication
step level.

Before discussing the last set of SSR functions – this last set contains
higher level functions – we will focus briefly on the reality that the
SSR must have a variable geometry and the problems that need solutions
due to this variable geometry.


\bigskip

\subsection[The SSR variable geometry]{The SSR variable geometry}
\hypertarget{RefHeading3088306210128}{}The SSR has variable geometry
because:


\bigskip

\begin{itemize}
\item The mature SSR must grow in volume and enclosure surface during
the cloning phase to make space in its interior for the growing clone. 
\item The geometry changes even more radically when the division phase
starts and culminates with the complete division of the original SSR in
two: the “mother” SSR and the “daughter” SSR.
\end{itemize}
The SSR variable geometry has the following relevant aspects that must
be carefully considered by the SSR design and by the hypothetical
implementation of the artificial SSR:


\bigskip

\begin{itemize}
\item The SSR enclosure must have such a structure and composition
(texture) that allows:

\begin{itemize}
\item Growth in surface as the SSR interior grows in volume during the
cloning phase. This may require a design that allows selective
insertion of new enclosure parts/elements in between existing
parts/elements and some kind of “linkage/connection” of each new
element with its neighboring elements.
\item Insertion in the enclosure of new input gateways and output
gateways while the enclosure grows.
\item Division of the enclosure into two separate enclosures (one for
the mother SSR and one for the daughter SSR)  with each enclosure
carrying its separate sets of input and output gateways, its
scaffolding and interior elements.
\end{itemize}
\item Special design provisions must be made for the interior SSR
scaffolding. The SSR scaffolding is made of any kind of structural
elements (pylons, walls, supports, connectors) that are needed to
maintain the three dimensional structure and integrity of the enclosure
and of the SSR interior space(s). The scaffolding design and its
elements need to be conceived such that:

\begin{itemize}
\item The scaffolding elements my change size (grow or shrink) as the
interior of the SSR grows (during cloning) or shrinks (during division)
\item The spacing between scaffolding elements and their connectors may
also grow or shrink (during cloning respective division phases).
\end{itemize}
\item The design of the SSR must also make provisions for the growth,
variable geometry and dynamic restructuring and re-linking of any SSR
transport, conduits, paths or communication lines during the cloning
phase and the division phase.
\end{itemize}
The variable geometry means that the SSR design must make specific,
detailed provisions of all spatial evolution, of all geometrical
definition points or trajectories (in the x, y, z axes) of all variable
elements of the SSR enclosure, scaffolding and interior. These spatial
trajectories need to be harmoniously and coherently coordinated with
all fabrication and assemblage steps of the cloning and division
phases.


\bigskip

\subsection[The Communication and Notification Function]{The
Communication and Notification Function}
\hypertarget{RefHeading3090306210128}{}This function is responsible to
provide and manage the information communication and notification
machinery and mechanisms between the command centers and execution
centers of the SSR. For example the fabrication control function-
acting as a command center – may send a command as a specifically
encoded information “package”  to the fabrication and assemblage
functions to build a particular component of the “daughter” clone. In
this circumstance the fabrication and the assemblage functions operate
as execution centers for the command. When the fabrication of the
requested component is completed by the fabrication function it will
send a specifically formatted notification information package back to
the fabrication control function with the meaning that the specific
command for the fabrication of the specific element was successfully
completed. Within the same example scenario the fabrication function
will send in its turn – this time playing a command role itself – a
command to the supply-chain function (the executor entity) to trigger
the transport of the needed fabrication “ingredients” for the clone
element to the fabrication site. The \textbf{communication and
notification function} need to be deployed ubiquitously throughout the
SSR to allow communications/notifications between various functions and
machinery operating all over the SSR.


\bigskip

\subsection[The higher level SSR functions]{The higher level SSR
functions}
\hypertarget{RefHeading3092306210128}{}The SSR functions that are
described in subsequent sections are the highest level functions of the
SSR. They accomplish their goals by coordinating and choreographing the
lower level functions described in the previous sections.


\bigskip

\subsection[The scaffolding growth function]{The scaffolding growth
function}
\hypertarget{RefHeading3094306210128}{}The \textbf{scaffolding growth
function} is responsible to manage the construction, growth and
position change of the SSR scaffolding elements during the cloning and
the division phases of the SSR replication. This function needs to
manage the variable geometry of the scaffolding elements in
synchronization and coordination with the other spatial changes of the
SSR on its enclosure and in its interior.


\bigskip

\subsection[The enclosure growth function]{The enclosure growth
function}
\hypertarget{RefHeading3096306210128}{}The \textbf{enclosure growth
function} is responsible to manage the construction, growth and shape
change of the SSR enclosure as well as the coordinated addition of
input and output gateways on the enclosure during the cloning and
division phases of the SSR replication. As already alluded to, this
function needs to manage the variable geometry of the enclosure, the
dynamic shifting of the gateways on the enclosure surface, and the
radical shape changes of the enclosure around the division of the SSR
into the mother and daughter descendants.


\bigskip

\subsection[The fabrication control function]{The fabrication control
function}
\hypertarget{RefHeading3098306210128}{}The \textbf{fabrication control
function} is responsible for the construction, assemblage and variable
geometry management of all interior elements – in particular those
related to the cloning part of the SSR during the cloning and the
division phases of the SSR replication.  Like the two preceding growth
functions this function coordinates activities of the fabrication
function, assemblage function, construction plan function, recycling
function and other lower level functions. 


\bigskip

\subsection[The cloning control function]{The cloning control function}
\hypertarget{RefHeading3100306210128}{}The \textbf{cloning control
function} is responsible to coordinate the whole cloning phase of the
growing SSR. It basically does that by coordinating the cloning and the
growth of all involved SSR compartments through a tight control,
synchronization and coordination of the scaffolding growth, enclosure
growth and fabrication control functions. This function is responsible
in particular to start the cloning process, to monitor its development
and to determine accurately when the cloning process is complete. One
particular responsibility of the cloning control function is the
\textbf{cloning of the information} stored into the mother SSR. This
information cloning is performed at the end of the cloning phase when
all internal machinery, internal scaffolding and enclosure elements of
the mature SSR were completely cloned and constructed as part of the
“nascent” daughter SSR. The information is cloned by systematically,
accurately and completely copying all information catalogs from
resident machinery of the mature SSR into the corresponding new
machinery of the clone part.


\bigskip

\subsection[The division control function]{The division control
function}
\hypertarget{RefHeading3102306210128}{}The \textbf{division control
function} has full control of the division phase of the SSR replication
process. It manages the SSR division through specific commands sent to
the scaffolding growth, enclosure growth and fabrication control
functions.  In particular this function is responsible to start the
division process, to “choreograph” its development on all SSR
compartments (enclosure, scaffolding and core) and to accurately
determine when division process is complete. One particular
responsibility of the division control function is to “start the
engines” of the nascent “daughter SSR”. Just before the moment the
division is complete the division control function must send a command
to the “daughter SSR” to start its own machinery and control functions.
In this way when the division completes, the separated daughter SSR
becomes a “mature”, fully functioning autonomous SSR ready to start its
own replication process. The separated “mother SSR” can start in its
turn a new replication cycle.


\bigskip

\subsection[The replication control function]{The replication control
function}
\hypertarget{RefHeading3104306210128}{}The \textbf{replication control
function} is the highest level SSR function. It is responsible to
accomplish the SSR full replication cycle by coordinating and
choreographing its two phases: cloning and division through
corresponding control and coordination of the \textbf{cloning control
function} and the \textbf{division control function}.


\bigskip

 Metaphorically speaking the replication control function implements the
two significant \textbf{SSR designer commandments:}


\bigskip

\begin{itemize}
\item \textbf{Grow} and
\item \textbf{Multiply}
\end{itemize}

\bigskip

\subsection[The SSR Function Dependencies Diagram]{The SSR Function
Dependencies Diagram}
\hypertarget{RefHeading3106306210128}{}An overall diagram illustrating
the identified set of functions present in the SSR and some of the
dependencies between these functions is presented in 4


\bigskip

The main dependencies and interactions between the identified SSR
functions are depicted in this figure. Not all such function
dependencies and interactions are represented in the picture but only
the main ones. 


\bigskip

The Communication and Notification Function is represented separately
since it relates to and is used by almost any other SSR function
because the need for information communication between functions as
well as notification (another form of information exchange) is
ubiquitous through SSR functions.


\bigskip

As already mentioned the relationships and dependencies between the
functions are more complex and richer than depicted in the diagram. For
example the Transport function depends on Energy Generation and
Transport function although this dependency is not depicted in the
diagram.


\bigskip


\bigskip

{\bfseries
\label{bkm:Ref330735805}Figure  SSR functions and their dependencies}


\bigskip


\bigskip

{\bfseries
\hypertarget{RefHeading3108306210128}{}The type and nature of SSR
components}


\bigskip

At this point it makes sense to only consider the general conceptual
categories to which various components of the artificial SSR pertains.
We grouped these components according to a few categories.

Part II of this paper will go into more concrete details about actual
physical components that may make up the artificial SSR.


\bigskip

SSR Component Categories


\bigskip

\begin{itemize}
\item Information storage and access
\item Information processing
\item Information coding and decoding
\item Information transport, communication and notification
\end{itemize}

\bigskip

\begin{itemize}
\item Material identification
\item Material transport and manipulation
\item Material processing
\item Mechanical and chemical transformation of materials
\item Material fabrication and assemblage
\end{itemize}

\bigskip

\begin{itemize}
\item Energy generation
\item Energy transport
\item Energy conversion
\item Energy distribution and management
\end{itemize}

\bigskip

\begin{itemize}
\item Environment sensing
\item Environment (local) control
\end{itemize}

\bigskip

\begin{itemize}
\item SSR construction plan representation
\item SSR dynamic 3D evolution representation
\item SSR construction status representation
\item SSR parts inventory representation
\end{itemize}

\bigskip


\bigskip

{\bfseries
\hypertarget{RefHeading3110306210128}{}The SSR and its information
catalogs}

We already encountered various types of information catalogs (databases
or repositories) that together make up the information base of the SSR.
This section contains a list of all types of information catalogs
identified so far. It is quite probable that as a more in depth
analysis is performed on this topic the need of additional types of
information catalogs will be discovered.


\bigskip

All information describing in detail each element of the SSR during its
full “life-cycle”, all relationships between these elements captured in
construction plans (body plans) and all fabrication and assemblage
procedures and processes need to be thoroughly, systematically and
coherently designed and captured into the SSR information catalogs.


\bigskip

We are making certain simplifying assumptions, just to make the
presentation of ideas easier to understand. We are not discussing and
presenting a very significant aspect of the nature of information that
the SSR design must capture. A real information repository contains
conceptual lists of tables (catalogs) of items of the same nature but
also contain lists of the various relationships that exist between the
items in different tables (catalogs).  For example, there are certain
relationships between the items in the catalog of raw materials and the
items (entries) in the catalog of bill of materials. There are
different sets of relationships between the entries in the catalog of
the construction plans and the entries in the catalog of the bill of
materials and the entries in the catalog of processes.


\bigskip

Here is the list of information catalogs identified (or alluded to) so
far:


\bigskip

\begin{itemize}
\item The catalog of raw materials
\item The catalog of raw parts
\item The catalog of fabrication materials
\item The catalog of raw materials identification procedures and
processes
\item The catalog of raw parts identification procedures and processes
\item The catalog of fabrication materials extraction procedures and
processes
\item The catalog of energy generation procedures and processes
\item The bill of materials catalog
\item The catalog of internal machinery
\item The catalog of construction plans
\item The catalog of fabrication procedures and processes
\item The catalog of assembly procedures and processes
\item The catalog of fabrication verification procedures
\item The catalog of fabrication errors handling procedures
\item The catalog of energy consumptions
\item The catalog of recycling elements and procedures
\item The catalog of construction status
\end{itemize}

\bigskip

\clearpage\section[PART II]{PART II}
\hypertarget{RefHeading3112306210128}{}\section[Evaluating the
Complexity of an Exhaustive Design and of a Concrete Implementation of
an Artificial SSR]{Evaluating the Complexity of an Exhaustive Design
and of a Concrete Implementation of an Artificial SSR}
\hypertarget{RefHeading3114306210128}{}
\bigskip

{\bfseries
\hypertarget{RefHeading3116306210128}{}The Three Closure Requirements as
the Basis of an Autonomous SSR}

SSR must be fully autonomous. This means that it can only get raw
materials and raw parts from its environment and benefit from (or
struggle because) the environmental conditions specific to its
location.

Full autonomy means specifically :


\bigskip

\begin{enumerate}
\item SSR must fabricate all its energy from input materials and the
generated energy must be sufficient for the SSR to produce an exact
replica of itself. This condition is called the \textbf{“energy
closure”}. 
\item SSR must use only materials admitted through its input gateways
and these materials must be sufficient for the SSR to grow and generate
its replica. This condition is called the \textbf{“material closure”}. 
\item SSR must use only information present/stored initially in the
“mature” SSR and this information must be sufficient to produce an
exact replica of the SSR. This condition is called the
\textbf{“information closure”}.
\end{enumerate}

\bigskip

{\bfseries
\hypertarget{RefHeading3118306210128}{}The Core Approach to Cloning}


\bigskip

In this section we try to answer the following important question for
the design of the artificial SSR: \textbf{what is the core mechanism to
be used by the artificial SSR to accurately clone all of its elements?}


\bigskip

Below are two possible answers. And we believe that most any other
imagined answers may very well be similar or equivalent with one of the
two answers below:


\bigskip

\begin{enumerate}
\item Design and use a universal copy machine similar with a key copy
machine (but much more sophisticated) with the goal of achieving the
simplest possible solution – and avoid the “complications” of solution
2 below.
\item Use an exhaustive descriptive, operational and constructional SSR
information database driving an integrated set of specialized, software
and computer-controlled automatons.
\end{enumerate}

\bigskip

\subsection[Why the “super copy machine” approach is not adequate]{Why
the “super copy machine” approach is not adequate}
\hypertarget{RefHeading3120306210128}{}Let’s describe in more detail
what this type of solution would mean. This approach assumes that the
SSR contains a sophisticated machine that can examine and accurately
copy all other pieces and machinery making up the mature SSR. This
implies that this super copy machine can copy itself or, more
realistically a copy of itself. In other words, the SSR contains two
super copy machines, one, Machine A is going to do the actual copy of
all SSR machinery and do the copy of the other super copy machine,
machine B. So the second copy machine, machine B is just used as a
model for the first machine A. Also, the emphasis on this solution is
that these super copy machines A and B will alleviate the need to store
so much information, to have so much software and so much
computer-controlled machinery as you will see in the second approach.


\bigskip

Trying to analyze in depth this first solution we will find out the
following:


\bigskip

\begin{enumerate}
\item There will be a need to have another machinery
M\textsuperscript{disassembler}, to disassemble machine B in all its
constituent parts: b\textsubscript{1}, b\textsubscript{2},
b\textsubscript{3}, …, b\textsubscript{N }so that machine A can copy
each constituent part.
\item The machine A will copy all parts b\textsubscript{1},
b\textsubscript{2}, b\textsubscript{3}, …, b\textsubscript{N} twice (to
create all pieces needed for a Clone machine A: Copy\textsubscript{A}
and a clone machine B: Copy\textsubscript{B}) ca\textsubscript{1},
ca\textsubscript{2}, ca\textsubscript{3}, …, ca\textsubscript{N }(for
Copy\textsubscript{A }machine) and cb\textsubscript{1},
cb\textsubscript{2}, cb\textsubscript{3}, …, cb\textsubscript{N} (for
Copy\textsubscript{B} machine).
\item There will be a need for another machinery
M\textsuperscript{assembler }that will know how to take all the parts
ca\textsubscript{1}, ca\textsubscript{2}, ca\textsubscript{3}, ….,
ca\textsubscript{N} and assemble them together into the
Copy\textsubscript{A} machine and parts cb\textsubscript{1},
cb\textsubscript{2}, cb\textsubscript{3}, …, cb\textsubscript{N }and
assemble them together into the Copy\textsubscript{B }. But how such a
M\textsuperscript{assembler} machine can be constructed, without having
separate, well-structured information of this nature:
\end{enumerate}
\begin{itemize}
\item A catalog of all parts b\textsubscript{1}, b\textsubscript{2},
b\textsubscript{3}, …, b\textsubscript{N  }and for each such part a
unique identifier and possibly physical and geometrical characteristics
(dimensions) of the part
\item A store room location (x, y, z)  from where the
M\textsuperscript{assembler} machine will pick each one of the parts
ca\textsubscript{1}, ca\textsubscript{2}, ca\textsubscript{3}, …,
ca\textsubscript{N }during assembly steps to construct the
Copy\textsubscript{A} machine
\item A \textbf{catalog of assembly instructions} (some geometrical x,
y, z instructions, what kind of assemblage step: like screwing,
inserting, welding, etc. on:

\begin{itemize}
\item how to put together part ca\textsubscript{2} to the assembly made
of parts: (ca\textsubscript{1})
\item how to add part ca\textsubscript{3} to the assembly made of parts
(ca\textsubscript{1}, ca\textsubscript{2})
\item how to add part ca\textsubscript{4} to assembly made of parts
(ca\textsubscript{1}, ca\textsubscript{2}, ca\textsubscript{3}),
\item …….
\item how to add part ca\textsubscript{N} to the assembly made of parts
(ca\textsubscript{1}, ca\textsubscript{2}, ca\textsubscript{3},
ca\textsubscript{4}, …., ca\textsubscript{N-1})
\end{itemize}
\item A manipulator machine (robot that can follow computerized
instructions) to  be controlled by the M\textsuperscript{assembler}
machine in assembling the CopyA machine
\end{itemize}
So, although the original intent for Solution 1 is to avoid mountains of
information and armies of automatons and machinery, our analysis
revealed that this solution cannot do without those ingredients, and
there is no magic copy machine that can do its work, without structured
collections of information and many helper automatons (machinery) that
in turn must be information, software and computer controlled.

The conclusion is that there is no magic “smart copy machine” solution
that is significantly distinguishable from the Solution 2. Another
problem for Solution 1 is that certain components of machine B may not
be fabricated by plain (mechanical) assemblage of parts but rather by
using more demanding assemblage processes (let’s take welding for
example, or an electro-chemical process) that have no exact
dis-assemble counterparts. Let’s then focus on what Solution 2 describe
in the section below.


\bigskip

\subsection[Exhaustive information, integrated systems driving
information{}-controlled automatons]{Exhaustive information, integrated
systems driving information-controlled automatons}
\hypertarget{RefHeading3122306210128}{}\label{bkm:Ref348984751}We cannot
find any other way an artificial SSR would be capable to replicate, but
only if it is thoroughly and carefully designed as a collection of
integrated sub-systems, controlling a large variety of automatons using
a significant collection of integrated information catalogs
(databases).


\bigskip

We already mentioned a number of information catalogs while identifying
specific SSR functions. Any informational SSR function has both an
associated catalog and also a set of access sub-functions that provide
a set of access operations to the information catalog that can be used
by other SSR functions to execute specific action sequences.


\bigskip

{\bfseries
\hypertarget{RefHeading3124306210128}{}The material basis of the SSR}

When approaching the task of the design and implementation of an
artificial SSR a capital question surfaces rather soon. What should be
the material basis for the artificial SSR? We see here two distinct
possibilities.


\bigskip

\begin{enumerate}
\item \textbf{Use a biological basis} for the design and construction of
the artificial SSR and a micro/nano scale common for unicellular
organisms. 
\end{enumerate}
This means that the SSR must be constructed using organic materials, the
same or similar with those used by the cells, tissues and organs of the
living world.

Here are the \textbf{advantages} of this approach:


\bigskip

\begin{itemize}
\item We know that this approach is possible, proof being the presence
of so many organisms and microorganisms in nature. The question here is
if this approach is accessible to our current engineering technologies.
\item There are low levels of energy consumption associated with this
approach and a solution for generation of enough energy for SSR
replication is one of the biggest challenges.
\item The aqueous medium of this approach may facilitate solutions for
the variable geometry problem.
\end{itemize}
Here are the \textbf{problems or barriers} for this approach:


\bigskip

\begin{itemize}
\item The investigative tools and observation means available to us at
this scale are still quite limited
\item The most advanced micro-biology manipulation and fabrication
tools/approaches are still rather primitive and very limited when
considering the tasks that need to be accomplished: fabrication,
assemblage, manipulation at nano scales, computing machinery
fabrication, software execution, information storage and communication.
\item We are just at the beginning of the process of understanding how
cells work. There are many areas and aspects of cell biology that
appear to be beyond our comprehension.
\item Examples of challenges that we cannot solve with current
technology: 

\begin{itemize}
\item building computers on a biological material basis / scale (or
understanding how the cell proteins and other organic cell elements can
be used for computation in a general way)
\item building bio-chemicals manufacturing machines at a biological
scale.
\item building information storage with a biochemistry material /scale
\item having software running on biological type computers.
\item communicating information on a biological material basis /scale.
\end{itemize}
\end{itemize}
The conclusion is that it is quite clear that with the current level of
technology it is impossible to create a design and an implementation
plan for an artificial SSR using biochemistry and a biological material
basis and a molecular scale. We are forced to consider any other
alternative with better chances of success.


\bigskip

\begin{enumerate}
\item \textbf{Use a macro scale and materials and technology common for
fabrication of engineering artifacts.}
\end{enumerate}
This approach means that we will need to consider and use the minimum
dimensional scales for which there are available manufacturing
technologies for most of the parts, components and machinery that make
up the SSR.  In particular this material basis approach implies the
following:


\bigskip

\begin{itemize}
\item Materials used to construct the SSR enclosure, SSR scaffolding and
SSR interior elements are common engineering materials used by current
fabrication technologies: metals, alloys, plastics, ceramics, silicon
or other special materials.
\item The scale of the artifacts to be fabricated as elements of the
artificial SSR must be selected to strike the best compromise between
two opposing tendencies:

\begin{itemize}
\item The smallest possible scale should be used in the design and
implementation of the artificial SSR parts in order to:

\begin{itemize}
\item minimize the energy consumed by the SSR during a replication cycle

\item minimize the size, volume and mass of the artificial SSR and thus
the amount of materials “ingested” into the artificial SSR and used for
fabrication of the clone inside the SSR.
\end{itemize}
\item The scale selected for the artificial SSR must be such that there
are known engineering technologies and machinery to fabricate,
manipulate and assemble all the parts of all the SSR machinery. This
means - for an illustrative example - that if the minimum size of a
semiconductor fabrication equipment that is being manufactured today is
0.5 meters then the designed size of the mature artificial SSR cannot
be smaller than 1 meter. For this example a more realistic artificial
SSR designed dimension should be in the range of at least 10-100
meters.
\end{itemize}
\end{itemize}

\bigskip


\bigskip


\bigskip

{\bfseries
\hypertarget{RefHeading3126306210128}{}The type and nature of SSR
Components}


\bigskip

We concluded in section:   that SSR must be designed and implemented as
a collection of integrated, computer-controlled and software controlled
automatons. This helps us to identify the nature of some of the
elements and components that make up the SSR. The artificial SSR must,
by necessity contain this type of elements:


\bigskip

\begin{itemize}
\item Computing machinery which implies that the following type of
elements must be present inside the artificial SSR:

\begin{itemize}
\item Computers
\item Printed circuit boards (PCB)
\item Microprocessors
\item Highly integrated circuits (Application Specific Integrated
Circuits = ASICs ) – specialized, high density integrated circuits for
specific computing/application tasks: networking, numerical processing,
image processing, etc.
\item Semiconductor memories (solid state memories)
\item Magnetic memory (hard drives)
\item Electric power supplies
\item Computer connectors and wiring.
\end{itemize}
\item Networking Communication Devices:

\begin{itemize}
\item Routers (wired/wireless)
\item Switches
\item Modems
\end{itemize}
\item Software
\item Robots
\item Energy generation and distribution machinery

\begin{itemize}
\item Generators
\item Transformers
\item Converters
\item Wiring
\end{itemize}
\item Batteries
\item Fabrication machinery
\item Metal machining machinery
\end{itemize}
{\bfseries
\hypertarget{RefHeading3128306210128}{}Derived Design Requirements}

In this section we are presenting a list of design and implementation
requirements for the artificial SSR that emerged from the previous
analysis and from the inferences presented so far. These requirements
were only implied during our discussion so far but now they are made
explicit and described in some detail.


\bigskip

\subsection[Each SSR Machine is Power{}-Driven]{Each SSR Machine is
Power-Driven}
\hypertarget{RefHeading3130306210128}{}This condition has the following
significant consequences for the SSR design.


\bigskip

\begin{itemize}
\item The SSR must have a power distribution network (electrical
distribution network) that must reach each of the SSR machinery. The
design of the layout and geometry of the power network must consider
the variable geometry of the SSR enclosure, scaffolding and its
interior space and structure – in particular in the zones affected by
growth and shape changes.
\item Each SSR machine must be designed to use and consume power
(electricity) at a level adequate for its nature and the actions it
performs. 
\item Most (all) machinery is computer driven, which means that either
there is a parallel SSR power network for an energy level (voltage)
adequate for computing devices, or each machine must have some adequate
power converters (electrical power supplies, batteries).
\item The SSR machines that provide mechanical work or movement must be
provided with motors (rotational, linear) adequate for their nature.
\item Mobile machinery (transporters, moving robots) must be designed
such that their mobility is not inhibited or constrained while they
are/remain plugged into the SSR power network(s).  Designing all mobile
machinery with rechargeable batteries may solve or significantly
simplify the connectivity constraints but will require additional
provisions for battery fabrication processes and fabrication and
provision of battery charging stations.
\item The design of each machine must provide specification of average
power consumption on all power networks (normal power level and
computer power level) to which the machine is connected.
\end{itemize}
\subsection[Each SSR Machinery is Computer{}-Driven and
Software{}-Driven]{Each SSR Machinery is Computer-Driven and
Software-Driven}
\hypertarget{RefHeading3132306210128}{}These are the consequences for
the SSR design derived from this condition:


\bigskip

\begin{itemize}
\item Most SSR machines must host at least one internal computer (with
the possible exception of some simpler (electro-mechanical) machines
that can be remotely controlled).
\item The SSR must have high technology machinery and processes needed
to fabricate computers and all their building blocks (parts)
\item Each SSR machine that hosts computers must be “networked”:
connected by wire or wirelessly to other machines and “control centers”
in the SSR.
\item The SSR must have machinery that has the ability to not only
fabricate computers but also to install them into other SSR machines,
to plug them to that machine power network and to the SSR communication
network.
\item The SSR must have machinery that is able to download and copy
software into any computer installed into a SSR machine and to start
(boot) that software on that machine and to monitor its availability
and expected behavior.
\item THE SSR must have the capability to test each of its machinery, to
detect malfunctions (errors) in computer and software installations as
well as in the machine hardware, to detect malfunction in the computer
and software execution and to have adequate procedures to diagnose and
repair the identified problems based sometimes on the availability of
fabricated spare parts.
\item The software that drives each particular machinery must be
designed and written with a full understanding of the physical and
cinematic capabilities and constraints/limitations of that machinery.
It must take into account all possible use cases of the machine and all
its components behaviors and interactions with external objects and
events and handle them correctly.
\item The software that drives fabrication and assemblage machinery and
materials and fabrication processes needs to be based on a thorough
design of the machines to be built, their cinematic capabilities and
their specified power and energy consumption.
\item The software written for various SSR functions must carefully and
accurately choreograph, coordinate and synchronize the activities of
multiple SSR machines (fabrication machines, material process machines,
manipulation and transport robots/arms, assemblage and construction
machines) providing a continuous monitoring of the 3D spaces occupied
by each machine and its mobile parts, to avoid collisions and to ensure
cooperative progress with both lower level and higher level tasks of
the growing SSR.
\end{itemize}
\subsection[Each SSR Machinery is Information Communication
Capable]{Each SSR Machinery is Information Communication Capable}
\hypertarget{RefHeading3134306210128}{}These are the consequences for
the SSR design derived from this condition:


\bigskip

\begin{itemize}
\item The artificial SSR is a collection of automated machines and
robots. Their cooperation and coordination for achieving tasks from the
simplest (fabricating a part, or manipulating a part in a sequence of
steps for an assemblage operation) to the complex ones (like the
fabrication, the assemblage of computing hardware and software
installation for a new fabrication machinery) requires extensive,
continuous, multipoint and multi-level (hierarchical) communication of
information between machines, control functions and software
components.
\item The SSR must have a comprehensive physical layer communication
network for information transport (either wire-based and/or wireless)
with access points located on each (or most) SSR machines/robots and
sometimes in between the subsystems of the same SSR machine.
\item The SSR might need to have adequate networking devices (like
routers, switches, modems, codecs) to implement needed communication
patterns and topologies.
\item The SSR machines and software components engaged in communication
will need adequate networking/communication protocols with appropriate
characteristics for the needed communication bandwidth, handling of
errors and retransmissions, reliable routing and end point addressing.
\item The SSR should have capabilities to deploy software on newly
constructed machines, network nodes and be able to bring up the
network, verify it as part of starting up the “daughter” SSR system
(including its underlying communication network) as a preparatory step
in the SSR division phase.
\end{itemize}
{\bfseries
\hypertarget{RefHeading3136306210128}{}The Most Significant Challenges
for the Design and Implementation of an Artificial SSR}

In this section we are going to enumerate and discuss the most difficult
challenges and hurdles that the design and implementation of an
artificial SSR faces.


\bigskip

\subsection[The energy generation and the energy closure challenge]{The
energy generation and the energy closure challenge}
\hypertarget{RefHeading3138306210128}{}This challenge presents multiple
aspects discussed below.


\bigskip

\begin{itemize}
\item The \textbf{selection of an adequate basis for energy generation}.
Basically this depends on what natural materials that are available in
the SSR environment can be used for energy generation
\item We enumerate below some of the candidate material basis for energy
generation that might be considered for the design and implementation
of an artificial SSR

\begin{itemize}
\item Biochemical or organic (like vegetation used for energy
generation), biogas, biodiesel
\item Coal
\item Oil/petroleum
\item Natural gas
\item Methanol
\item Hydrogen (micropower, fuel cells)
\item Solar
\item Wind
\item Nuclear
\end{itemize}
\item The \textbf{energy closure challenge} means that the amount of
energy generated by the SSR from the primary energy producing materials
accepted/extracted from the SSR environment must be sufficient to power
all machinery (fabrication, assemblage, construction, transport,
manipulators, robots, computers, and networking gear) that equip the
SSR.
\item The SSR must be designed with the ability to slow down or even
completely shut down during the periods when the input of energy
producing materials is reduced or null.
\item The SSR ability to provide means to store energy (with batteries,
accumulators or stocking energy producing materials) may smooth out or
eliminate the need for transition to “hibernation” or shutdown states
of the SSR.
\item Designing SSR machines with local sources of energy (like
rechargeable batteries, accumulators, fuel reservoirs or fuel cells)
may provide all these machines with true, unconstrained mobility and
will significantly simplify the SSR design and implementation
difficulties related to keeping all mobile machines hooked to flexible
power wiring or network wiring.
\item Burning of fuels or preparing the materials for energy generation
(chemicals) increases the concerns that the design and SSR
implementation need to consider. Examples of such concerns are: 

\begin{itemize}
\item preserving SSR internal environmental parameters: temperature,
humidity; 
\item avoiding hazardous materials or avoiding fires;
\item providing storage and transportation containers for non-solid
(liquid or gaseous) materials.
\end{itemize}
\end{itemize}
\subsection[The material closure challenge]{The material closure
challenge}
\hypertarget{RefHeading3140306210128}{}The material closure challenge
for the artificial SSR design and implementation can be formulated this
way: all fabrication materials that are needed for fabrication of all
parts of the SSR machinery must be available in the SSR environment or
must be extractable from raw materials available in the SSR
environment.


\bigskip

This challenge may not appear to be so daunting. However if we just
think that the artificial SSR will have fabrication machinery for metal
machining, computers with semiconductor microprocessors and memories,
plastics, ceramics, we realize that there is potentially an extremely
large list of materials needed for SSR fabrication. Only a very short
subset of materials needed for the “clunky” (macro scale) artificial
SSR can give us an idea of how much a challenge this aspect can become:


\bigskip

\begin{itemize}
\item Iron,
\item Steel (of various varieties)
\item Copper,
\item Aluminum
\item Metal alloys (of different varieties)
\item Silver,
\item Gold,
\item Ceramics,
\item Plastics
\item Silicon
\item Polytetrafluoroethylene (Teflon) for Printed Circuit Boards
(PCBs),
\item Tin,
\item Nickel,
\item Germanium
\end{itemize}
Even if the list above is only partial it appears that there is a very
small probability that the environment where the artificial SSR may be
placed may feature such a large diversity of immediately available
materials or materials from which the materials in the list can be
somehow extracted. This makes the \textbf{material closure requirement
to appear unsolvable} and thus any project to design and implement an
artificial “clunky” replicator may be condemned to failure. 


\bigskip

\subsubsection[The match between the SSR design and the design of the
SSR environment]{The match between the SSR design and the design of the
SSR environment}
\hypertarget{RefHeading3142306210128}{}
\bigskip

Another way to formulate the challenge of the material closure is that a
successful selection of materials used for energy generation and for
fabrication of SSR internal parts, components and machinery must be
based on a thorough knowledge of the environment in which the designed
SSR is projected to exits, the nature of raw materials and parts in
such environment and if there are realistic material extraction paths
and processes starting with those materials in the environment. For us
there is no exaggeration to state that the chances of a successful
design and implementation of an artificial SSR depends on the design of
its projected environment. In other words the success requires a
perfect design both of the SSR and of its environment.


\bigskip

\subsection[The Fabrication Challenge]{The Fabrication Challenge}
\hypertarget{RefHeading3144306210128}{}The fabrication challenge can be
stated as the requirement that the artificial SSR must be able to
fabricate and assemble any type of parts, components and machines that
are part of the “mature” SSR which implies that all fabrication and
assemblage machines should be able to fabricate exact copies of
themselves.


\bigskip

While the material closure challenge focuses on the difficulty of having
available a very wide spectrum of fabrication materials, the
fabrication challenge also raises a wide spectrum of concerns:


\bigskip

\begin{itemize}
\item Since the artificial SSR will have a lot of machinery made with
metals (fabrication machinery, construction and assemblage machines,
robots, manipulator arms, networking gear,  wires, power supplies,
conduits, scaffolding), the SSR must have metal machining machinery –
probably of a large diversity. 
\item The SSR must be able to fabricate machinery and enclosures for
energy generation.
\item The SSR must be able to fabricate machinery and enclosures
(recipients) to control and host an undefined set of processes
(material extraction, energy generation, possible chemical reaction
processes, electrolytic processes, PCB etching chemical processes)
\item The SSR must be able to fabricate computers and computer parts
including microprocessors, integrated circuits, application specific
integrated circuits (ASICs), signal processing integrated circuits,
controller integrated circuits, printed circuit boards (PCBs), power
supplies, cabling, semiconductor memories, magnetic memories and media
(hard drives, solid state drives). This also implies that the SSR must
feature highly demanding “clean room” spaces with robotic manipulation
of materials and parts and semiconductor fabrication equipment.
\end{itemize}
\subsection[The Information Closure Challenge and the Hardware/Software
Completeness Challenge]{The Information Closure Challenge and the
Hardware/Software Completeness Challenge}
\hypertarget{RefHeading3146306210128}{}The information closure for the
SSR is the requirement that the information resident on the SSR is
sufficient to drive its successful replication without any additional
information coming from outside the SSR. The hardware and software
completeness requirement further extends the information closure
requirement with the demand that the computing hardware and software
present in the SSR together with the information resident in the SSR
are sufficient to drive, control and successfully complete the cloning
and division phases of the replication of the SSR. The SSR hardware and
software must provide full automation of the control, fabrication,
assemblage and the handling of special situations like error detection,
error repair and recovery after error.


\bigskip

The \textbf{design of the information} resident in the SSR must be
appropriate for its self-replication. It must be:


\bigskip

\begin{itemize}
\item \textbf{Complete (exhaustive):} it must cover all relevant aspects
(materials, parts, processes, procedures, plans, spatial structures,
error and recovery handling, etc.) that intervene during replication.
Completeness means also that the information designed and stored in the
SSR is correctly correlated with the design of the SSR environment.
That means, for example that the SSR design should be based on an
accurate and exhaustive list of raw materials and raw parts that exist
in the SSR environment together with the material identification
procedures and material processing/extraction procedures for those
materials.
\item \textbf{Adequate:} must cover all descriptive details of all
entries in the information catalogs, with all relevant properties for
these entries, with correct representation of various relationships
between the entries in the information catalogs.
\end{itemize}
The \textbf{computing hardware and software completeness requirement}
means the following:


\bigskip

\begin{itemize}
\item The designed computing hardware and software for each machine is
complete, sufficient and adequate to control, drive and monitor that
machine and to answer commands from the SSR control centers and to
properly communicate and exchange information, status and control
commands with other machines as needed to accomplish the higher level
functions of the SSR..
\item The hardware and software that are used by various SSR functions
and control centers are also complete, sufficient and adequate: they
cover all possible use cases including all possible errors and
incidents.
\end{itemize}
\subsection[The Highest Challenge: the SSR Design Challenge]{The Highest
Challenge: the SSR Design Challenge}
\hypertarget{RefHeading3148306210128}{}The SSR design challenge simply
means that the design of the SSR and the design of all its subsystems
(reviewed in the previous sections) are adequate for accomplishing a
successful self-replication of the fully autonomous SSR with
preservation and no degradation of the self-replication capability
passed to all generations of daughter SSRs.


\bigskip

There are several specific aspects of the design challenge enumerated
below.


\bigskip

\begin{itemize}
\item The design of the SSR must be fully coordinated with the design
(and thus the knowledge) of the environment in which the SSR will be
placed. This means in particular that the SSR design need to be fully
informed about the nature, characteristics and environmental conditions
(temperature, pressure, humidity, aggregation status) of the medium
where it will exist, including the nature of raw materials and raw
parts that are present in this medium.
\item The analysis and investigation conducted so far revealed that the
design and construction of a fully autonomous self-replicating SSR are
\textbf{extremely demanding}. The success of such a design and
construction appear to be heavily determined by appropriate choices
listed below and how these choices harmonize or not with the SSR
environment and its nature:

\begin{itemize}
\item The material basis of the SSR. Is it a carbon-based chemistry
(organic material basis) or not?  Is it a macro (“clunky”) material
basis or not?
\item The overall aggregation status of the SSR components: liquid,
solid (compact or with embedded spaces), aqueous, colloidal.
\item The scale of the mature SSR. Is it in the nanometer, micrometer,
millimeter, meter, kilometer scale?
\item The availability of energy generation materials and processes in
the material basis of choice and at the scale of choice.
\item The availability of well mastered techniques for the SSR material
basis of choice, scale of choice of fundamental engineering techniques
like:

\begin{itemize}
\item Energy generation and transport
\item Fabrication
\item Assemblage and construction
\item Transport and mobility
\item Manipulation
\item Computation
\item Information communication
\item Sensing
\end{itemize}
\end{itemize}
\end{itemize}
{\bfseries
\hypertarget{RefHeading3150306210128}{}The Emerging Image of the
Artificial SSR}

An artificial SSR is very similar to a modern city enclosed in a
dome-like structure that communicates with the outside world by
well-guarded gates used by robots to bring in construction materials
from outside the city. This modern city has two quarters: the “old
city” with its infrastructure in place and fully functional with
buildings, plants and avenues. The “new city” quarters are initially a
small empty terrain. As the new city is being constructed, its area
extends together with gradual extension of the dome covering both the
old and the growing new city. Both the old city and new city quarters
are pulsating with construction activity: automated machines (robots)
carry new materials, parts and components that are used to construct
the infrastructure of the new city quarters, to continuously extend the
dome on top of it and to construct in the new quarters an exact replica
of the old city.


\bigskip

Before the construction of the new city quarters even began, here is
what can be seen in the old city:


\bigskip

\begin{itemize}
\item material mining sub-units 
\item metallurgic plants
\item chemical plants
\item power plants
\item an electricity distribution network
\item a library with information for all city construction plans in
electronic form that is made available on the city web and used by the
city control centers, its machinery and robots to achieve their tasks. 
\item a network of avenues, alleys and conduits for robotized
transportation
\item fully automated and robotized manufacturing plants specialized in
fabrication of all parts, components assemblies and machines that are
present in the old city.
\item a fully automated semiconductor manufacturing plant with clean
rooms for fabrication of microprocessors, ASICs, memories and other
highly integrated semiconductor circuits and controllers.
\item a computer manufacturing plant
\item a network equipment manufacturing plant
\item an extended communication network connecting by wire or rather
wirelessly all plants and robots
\item a software manufacturing plant and software distribution and
installation robotized agents.
\item warehouses and stockrooms to store raw and fabricated materials,
parts, components, assemblies and software on some storage media.
\item a materials and parts recycling and refuse management plant
\item an army of intelligent robots for transportation, manipulation,
fabrication and assemblage.
\item an army of recycling robots that maintain clean avenues and
terrains in the city, collect debris from various plants and
reintroduce the recyclable materials and parts in the fabrication and
construction while the bad parts are taken out of the city gates.
\item control, command and monitor centers that coordinate the supply of
materials, the fabrication and the construction of an identical copy of
the original plants, avenues, factories and stockrooms
\item a highly sophisticated, distributed, multi-layered software system
that controls all plants, robots and communications in a cohesive
manner.
\end{itemize}
Each one of the transport carriers, robots, manipulators and
construction machinery is active and does its work without impeding the
movement of any other machine. Everything appears to be moving
seamlessly, orderly and the construction of the new city is making
visible progress under the growing pylons of the bolting dome.


\bigskip

When the new city quarters are completed and they are looking exactly
like the old city, you can see some activity and the machinery and
robots of the new city coming out “to life”. Something starts happening
as well: machinery and robots from both the old and the new city start
a re-modeling of the supporting pylons and of the arching dome. Little
by little their plans start making sense and it can be seen now that
what used to be a single, super-arching dome is changing shape into two
separate domes one for each of the city quarters.


\bigskip

The last robot activity is as coordinated as it ever was: you see the
two teams of the robots, finishing the re-shaping of the domes and the
complete separation of the old city and the new city with their own
round shaped roof. The separating wall is completed and what was once –
not long time ago - the dome hosting the two quarters is now two,
totally separate “old cities” starting their own, separate destinies.


\bigskip

{\bfseries
\hypertarget{RefHeading3152306210128}{}A Brief Survey of Attempts to
Build Artificial Self Replicators}

No successful attempt has been made so far for building a real
autonomous artificial SSR from scratch. W. M. Stevens summarizes the
situation in the abstract of his PhD thesis:


\bigskip

“Research into autonomous constructing systems capable of constructing
duplicates of

themselves has focused either on highly abstract logical models, such as
cellular automata,

or on physical systems that are deliberately simplified so as to make
the problem more tractable “


\bigskip

In his PhD thesis “Self-Replication, Construction and Computation” W. M.
Stevens reviews some of the attempts made at building physical or
abstract self-replicating machines :


\bigskip

\begin{itemize}
\item \textbf{Von Neumann’s kinematic model.} The system is made up of a
control unit governing the actions of a constructing unit, capable of
producing any automaton according to a description provided to it on a
linear tape-like memory structure. The constructing unit picks up the
parts it needs from an unlimited pool of parts and assembles them into
the desired automaton. The project was far from being finished and
remained an abstract model when von Neumann died.
\end{itemize}

\bigskip

\begin{itemize}
\item \textbf{Moses{\textquotesingle} programmable constructor
}\textbf{.} Matt Moses developed a physical constructor designed to be
capable of constructing a replica of itself under the control of a
human operator. The system is made of only 11 different types of
tailor-made plastic blocks.
\end{itemize}

\bigskip

\begin{itemize}
\item \textbf{Self-replicating modular robots.  }Zykov, Mytilinaios,
Adams and Lipson built a modular robotic system in which a
configuration of four modules can construct a replica configuration
when provided with a supply of additional modules in a location known
to the robot.
\end{itemize}

\bigskip

\begin{itemize}
\item \textbf{The RepRap project and 3D printing}. Bowyer et al.  have
developed a rapid prototyping system based around a 3D printer that is
capable of being programmed to manufacture arbitrary 3D objects. Many
of the parts of the printer that is assumed to be self-reproducing
cannot be manufactured by the system and those parts happen to be the
most complex ones (the computer controller for example)
\end{itemize}

\bigskip

\begin{itemize}
\item \textbf{Drexler’s assembler.} In Engines of Creation , K. Eric
Drexler describes a molecular assembler that is capable of operating at
the atomic scale. The molecular machine has a programmable computer, a
mobile constructing head and a set of interchangeable reaction tips
that will trigger chemical reactions controlled to construct any object
at molecular scale. This proposal stirred quite a controversy with some
scientists being skeptical of the feasibility of such a project . 
\end{itemize}

\bigskip

\begin{itemize}
\item \textbf{Craig Venter’s synthetic bacterial cell and synthetic
biology}. Craig Venter and the scientists at
\textcolor[rgb]{0.2,0.2,0.2}{J. Craig Venter Institute in Rockville, MD
reported in the May 20, 2010 issue of journal
}\textit{\textcolor[rgb]{0.2,0.2,0.2}{Science}}\textcolor[rgb]{0.2,0.2,0.2}{
that they created a “new species -{}- dubbed Mycoplasma mycoides
JCVI-syn1.0 – that is similar to one found in nature, except that the
chromosome that controls each cell was created from scratch”
}\textcolor[rgb]{0.2,0.2,0.2}{. In the same ABC news report,
}\textcolor[rgb]{0.2,0.2,0.2}{Mark Bedau, professor of Philosophy and
Humanities at Reed College in Portland, Ore., also writing in the
Nature commentary, called the new species {\textquotedbl}a normal
bacterium with a prosthetic genome.{\textquotedbl} }
\end{itemize}

\bigskip

\begin{itemize}
\item \textbf{\textcolor[rgb]{0.2,0.2,0.2}{Synthetic
biology}}\textcolor[rgb]{0.2,0.2,0.2}{ }is a new area of biological
research and technology that combines science and engineering. It
encompasses a variety of different approaches, methodologies, and
disciplines with a variety of definitions. The common goal is the
design and construction of new biological functions and systems not
found in nature . There are interesting advances in this field with
development of various techniques in domains like synthetic chemistry,
biotechnology, nanotechnology and gene synthesis. However these are far
from constituting a complete, coherent and effective set of techniques
that will allow the construction and synthesis of the large diversity
of machinery and functions that we identified in the preceding text as
the “portrait” of the artificial SSR.
\end{itemize}

\bigskip

\begin{itemize}
\item \textbf{Micro-electro-mechanical systems (MEMS)} is the technology
of very small devices; MEMS are also referred to as
\textit{micromachines} (in Japan), or \textit{micro systems technology}
– \textit{MST} (in Europe). MEMS are made up of components between 1 to
100 micrometres in size (i.e. 0.001 to 0.1~mm), and MEMS devices
generally range in size from 20 micrometres (20 millionths of a meter)
to a millimeter (i.e. 0.02 to 1.0~mm) . The technology made significant
advances with several types of MEMS being currently used in modern
equipment:  accelerometers, MEMS gyroscopes, MEMS microphones, pressure
sensors (used in car tire pressure sensors), disposable blood pressure
sensors and micropower devices. This is probably one of the most
promising types of technology for implementing small scale artificial
SSRs. However, there are still significant hurdles: implementation of
mobile elements (mini robots) and the common difficulty of fabricating
the machinery that fabricates MEMS and microprocessors.
\end{itemize}

\bigskip

\begin{itemize}
\item \textbf{NASA Advanced Automation for Space Missions 1980 Project}
. This study is one of the most realistic exploration of the design of
an artificial “macro” self-replicator and is briefly analyzed in its
own section below as is ”\textbf{A Self-Reproducing Interstellar
Probe}” (REPRO) study 
\end{itemize}

\bigskip

\subsection[NASA Advanced Automation for Space Missions Project]{NASA
Advanced Automation for Space Missions Project}
\hypertarget{RefHeading3154306210128}{}One of the missions of this
project is described in its Chapter 1 :

\textcolor{black}{“Mission IV - Self-Replicating Lunar Factory - an
automated unmanned (or nearly so) manufacturing facility consisting of
perhaps 100 tons of the proper set of machines, tools, and teleoperated
mechanisms to permit both production of useful output and reproduction
to make more factories. “}

\textcolor{black}{Then, later, it is described in more details:}

\textcolor{black}{“(d) Replicating Systems Concepts Team. The
Replicating Systems Concepts Team proposed the design and construction
of an automated, multiproduct, remotely controlled or autonomous, and
reprogrammable lunar manufacturing facility able to construct
duplicates (in addition to productive output) that would be capable of
further replication. The team reviewed the extensive theoretical basis
for self-reproducing automata and examined the engineering feasibility
of replicating systems generally. The mission scenarios presented in
chapter 5 include designs that illustrate two distinct approaches - a
replication model and a growth model - with representative numerical
values for critical subsystem parameters. Possible development and
demonstration programs are suggested, the complex issue of closure
discussed, and the many applications and implications of replicating
systems are considered at length.” }

 below is reproduced from the NASA study and, among other things it
depicts \textit{“}…In the lower left corner, \textit{a lunar
manufacturing facility rises from the surface of the Moon. Someday,
such a factory might replicate itself, or at least produce most of its
own components, so that the number of facilities could grow very
rapidly from a single seed” }


\bigskip

{\bfseries
\label{bkm:Ref330929827}Figure  NASA - the spirit of space missions.
Created by Rick Guidice}


\bigskip

Among other things the project specifies:


\bigskip

\begin{itemize}
\item The seed of the lunar factory – transported from Earth – would
weigh 100 tons
\item Not all machinery could be built on the Moon; computer boards will
be brought up from earth as “vitamins”, since parts closure will not be
achieved
\item It was estimated that the project will be feasible in the
21\textsuperscript{st} century
\end{itemize}
 below illustrates the depth the project reached in considering
fabrication facilities on the Moon.


\bigskip

{\bfseries
\label{bkm:Ref330932474}Figure  LMF parts fabrication sector:
Operations. (Fig 5.17 in the NASA study)}


\bigskip

Being quite thorough the NASA study represents a realistic evaluation of
the extent, the problems and the difficulties that need to be addressed
by the design of a macro (kilometers scale in this case) self
replicator.


\bigskip

\subsection[The Self{}-Reproducing Interstellar Probe (REPRO) Study]{The
Self-Reproducing Interstellar Probe (REPRO) Study}
\hypertarget{RefHeading3156306210128}{}In 1980 Robert A. Freitas
publishes  “A Self-Reproducing Interstellar Probe” (REPRO) study  in
the \textit{Journal of the British Interplanetary Society}. Some of the
main goals of the project are summarized below.


\bigskip

\begin{itemize}
\item REPRO was a mammoth self-reproducing spacecraft to be built in
orbit around Jupiter.
\item REPRO was a vast and ambitious project, equipped with numerous
smaller probes for planetary exploration, but its key purpose was to
reproduce. Each REPRO probe would create an automated factory that
would build a new probe every 500 years. Probe by probe, star by star,
the galaxy would be explored 
\item The total fueled mass of REPRO was projected to be 10**10 Kg = 10
**7 tons = 10 million tons for a probe mass of 100,000 tons.
\item It takes 500 years for REPRO to create a replica of itself in the
relatively hospitable environment of a far-away planet
\item The estimated exploration time of the galaxy was 1– 10 million
years
\end{itemize}
\clearpage\section[PART III]{PART III}
\hypertarget{RefHeading3158306210128}{}\section[The Metaphysics of the
SSR and the Origin Of Life Problem]{The Metaphysics of the SSR and the
Origin Of Life Problem}
\hypertarget{RefHeading3160306210128}{}
\bigskip

{\bfseries
\hypertarget{RefHeading3162306210128}{}The Insights into the Design of
the SSR, its Complexity and the Origin of Life}

Here are some conclusions of our investigation into the design of the
SSR, the complexity of the SSR and the feasibility of construction of
an artificial, autonomous SSR.


\bigskip

\textbf{The SSR has an overly complex design}


\bigskip

Our analysis determined that even the Simplest Self-Replicator (SSR)
\textbf{has an overly complex design}. From its ability to reproduce
with accuracy we derived in a logical, empirical, systematic manner
that the SSR must have a rich set of fully integrated advanced
capabilities (functions).  Any project or attempt to construct an
artificial SSR requires the employment of the most advanced engineering
techniques and technologies.


\bigskip

\textbf{Many unknowns about the cell and its mechanisms}


\bigskip

Scientists working in molecular biology, genetics, biotechnology,
bioinformatics and related life disciplines made significant progress
in understanding the mechanisms of the cell and the information that
drives some of its activities. However, in our estimation we are still
at the beginning of a lengthy road to discover many of the remaining
mechanisms, information repositories and processes in living cells and
organisms. Here are some areas that are so far still hidden (at least
partially) from human knowledge:


\bigskip

\begin{itemize}
\item What are the mechanisms for information communication in the cell
(besides those already known)?
\item Where in the cell is the information about the “body plan” of the
cell stored? Information about what cell organelles need to be created
during cell replication? How many of each type? During construction,
where will they be placed in the space of the cell, and will there be
any “linkages” between them and other cell structures? What is the
composition in proteins and other organic components of each type of
organelle?
\item How is the “supply-chain” function achieved in a cell (the
supplying in time of the needed organic material building blocks for
protein and organelle construction)?
\item What is the nature of “computations” performed within the cell? Is
it based on proteins/enzymes interactions only? Are there any other
forms of computations?
\item What are the inner mechanisms/ control centers that drive the cell
growth and cell division?
\end{itemize}

\bigskip

\textbf{A biological material basis for an artificial SSR?}


\bigskip

The most recommended approach for the design and construction of an
artificial SSR should choose a bio-chemical material basis and a
cell-like construction scale. However, the lack of knowledge of so many
things about the cell, our reduced abilities to investigate and to
operate at the cellular scale level makes the choice of this approach
totally impractical and condemned to a clear failure. We are
considering here the construction of an artificial SSR “from scratch”
using a biological material basis, with full understanding and control
of all elements and mechanisms involved in such a construction. An
accomplishment like Craig Venter’s synthetic bacterial cell, although
remarkable, it is not at all at the level of achievement this would
imply.


\bigskip

\textbf{A non-biological basis for an artificial SSR?}


\bigskip

The alternative approach for building an artificial SSR is to employ a
“macro” scale (it’s true at the minimum scale we can achieve) using a
common manufacturing/engineering material basis that is common for the
artifacts that the engineers know how to construct. Think for example
of electrical motors, metal/plastic robots, microprocessors, silicon
semiconductors, PCBs, plastics, ceramics, etc. This is because we know
how to build mechanical motors (even miniature ones) and NOT biological
motors. We know how to build “clunky” computers (even if they are small
like those in an iPad), but we don’t know how to build biological
computers or, in general biological machinery (here there might be some
insignificant exceptions).


\bigskip

Unfortunately, our analysis of what is involved in  the design and the
construction plans for a clunky artificial SSR, things like the energy
closure, material closure, information closure, the fact that humans
never built any fully automated, completely autonomous machinery (which
“knows” how to fabricate machines), make us quite skeptical that the
most advanced labs in the world, given some appropriate financial or
economic incentives, will be able to design and to construct a “clunky”
 artificial SSR using top-of-the-class technologies.


\bigskip

\textbf{Summary of our findings}


\bigskip

Let’s summarize our findings:


\bigskip

\begin{enumerate}
\item We know that there are many single celled living organisms that
\textbf{are fully autonomous} and have a \textbf{genuine ability to
self-replicate} achieving the \textbf{energy closure}, \textbf{material
closure} and \textbf{information closure} conditions.
\end{enumerate}

\bigskip

\begin{enumerate}
\item Scientists \textbf{are still at the beginning of the process of
understanding} in full depth the design, the information architecture,
the bio-chemical mechanisms of the living world and in particular of
the simplest self-replicating single-celled organisms
\end{enumerate}

\bigskip

\begin{enumerate}
\item It is estimated on a reasonable basis that scientists and
engineers are not able with the current knowledge and technology to
create from scratch \textit{an artificial SSR with a biological
material basis} for two fundamental reasons:

\begin{enumerate}
\item Because of Finding 2 above (lack of full understanding of the
single celled, self-replicating organisms).
\item Scientists and engineers are lacking significant investigative,
operational and constructive (manufacturing) methods for manipulating
biological matter or fabricating biological scale artifacts.
\end{enumerate}
\end{enumerate}

\bigskip

\begin{enumerate}
\item It is estimated on a reasonable basis that scientists and
engineers will encounter enormous (unsolvable) difficulties in the
design and construction of a clunky autonomous SSR \textit{with a
non-biological material basis} for the following reasons:

\begin{enumerate}
\item Extreme difficulty in satisfying the \textbf{energy closure}
condition in the artificial SSR 
\item Extreme difficulty in satisfying the \textbf{material closure}
condition in the artificial SSR 
\item Extreme difficulty in satisfying the \textbf{information closure}
condition in the artificial SSR
\item No engineering artifact with a complexity approaching that of the
investigated SSR, with its level of autonomy and complete automation
has ever been constructed
\end{enumerate}
\end{enumerate}

\bigskip

\begin{enumerate}
\item Our study of the design of the SSR revealed that its
\textbf{ability to self-replicate is founded on a full assortment of
highly structured information} resident in the SSR and carried over
accurately to each descendent SSR.  The information stored in the SSR
is highly structured for the following reasons:

\begin{enumerate}
\item Each \textbf{abstract concept} used by the SSR design (raw
material, part, procedure, construction plan) is represented by a
\textbf{catalog of entries}, each entry describing an instantiation of
that (abstract) category
\item Each entry in a particular catalog (i.e. for a particular
abstraction) has a well-defined \textbf{set of properties} describing
that type of object (entry)
\item There are some \textbf{complex relationships between entries} of
different catalogs, i.e. between represented abstractions. For example,
a part can be made from a particular material; a construction plan is
made from a sequence of procedures.
\end{enumerate}
\end{enumerate}
The set of findings above lead to the conclusion that a naturalistic
explanation of the origin of life is impossible for the following
reasons:


\bigskip

\textbf{Reason 1:}  Neither the laws of nature, nor random events can
generate highly structured information which we determined that must be
present in SSR (information catalogs).


\bigskip

\textbf{Reason 2:}  The functional model of the SSR we developed showed
that it must be composed from a well-rounded number of well-defined
capabilities and mechanisms that must be integrated, synchronized and
coordinated in their behaviors. An SSR cannot grow and duplicate if not
all these functions are all in place, fully functional. The SSR growth
and replication cannot be achieved with only one or a subset of the
required functions but rather all must be in place from the beginning.
For example it is not enough to have a fabrication function (RNA) if it
is missing the fabrication plan catalog (DNA). Or, even if it has both
the fabrication function and the fabrication plan catalog (RNA + DNA)
and is missing the input flow control function (membrane controlled
pores) and the division control function the “primitive SSR” will not
be able to replicate.


\bigskip

\textbf{Reason 3: } It is unreasonable to accept that the internal
arrangements, the information in the cell and its sophisticated
mechanisms that make rational sense, but have been only partially
understood by humans because they are overly complicated could be the
result of natural processes, the laws of nature or random events.


\bigskip

\textbf{Reason 4:  }It is unreasonable to accept that while the smartest
human scientists and engineers will probably fail to design and
construct an artificial SSR from scratch because of its supreme
complexity, some random sequence of natural events could have produced
such a sophisticated self-replicator.


\bigskip

\textbf{Reason 5:} The level of sophistication, the level of autonomy
and self-sufficiency, the degree of complexity of the simplest single
celled organisms is much beyond the level of technology and engineering
sophistication achieved by humans so far. The belief that the laws of
nature and any sequence of natural events and natural circumstances
could have created a self-replicating cell has absolutely no rational
foundation and is a pure religious belief without a defensible
scientific or empirical basis


\bigskip

{\bfseries
\hypertarget{RefHeading3164306210128}{}Simplifying Assumptions for the
Design and Construction of an SSR}

Let’s consider some \textbf{significant simplifying assumptions} for the
design and construction of an artificial SSR as listed below:


\bigskip

\begin{enumerate}
\item Eliminate the requirement that the SSR produces its own energy.
The electrical energy (at an appropriate voltage/amperage) will be
supplied to the artificial SSR from outside.
\item Eliminate the requirement that the SSR has the ability to select,
identify and accept through its input gateway appropriate raw
materials. All raw materials will be supplied as stock materials to the
artificial SSR. Additional, optional simplification: all stock
materials are labeled appropriately (with bar codes or RFIDs labels for
example). However (as an illustration) the SSR will still need to use
stock copper fed through the input gateways to fabricate copper wires
of certain gauges, or to use copper in the fabrication of electrical
motor parts.
\item Eliminate the requirement that the SSR fabricate certain highest
(most demanding) technology parts, components and assemblies (like
computer boards, microprocessors, semiconductor chips and memories,
etc.) These high-technology parts/components (named in the
self-replication literature “vitamins”) will be supplied, carefully
labeled from the outside world through the input gateways.
\item Eliminate the requirement that the SSR must fabricate any part,
component, assembly or machinery from scratch. The SSR will be supplied
with already-manufactured parts that are used in the composition of all
its machinery, scaffolding and enclosure. This simplifying assumption
means that now the SSR needs to be designed as a (sophisticated)
self-assembler that achieves self-replication by assembling exact
copies of itself using an exhaustive pool of ALL the parts of all the
machinery/assemblies it is composed of from the supplied elementary
parts coming through its input gateways.
\item Eliminate the requirement that the information repositories
(information catalogs) that drive the functions of the SSR reside
within the SSR. This requirement need to be replaced with requirements
for the SSR to possess reliable, high speed communication capabilities
to access the information catalogs (and possibly part of the software)
residing somewhere outside the SSR. This assumption may simplify
certain elements of the SSR design but at the same time will make other
requirements (communication, availability) more stringent for both the
SSR and for the external information resource. 
\end{enumerate}
Even if the original requirements for the design and construction of an
autonomous, artificial SSR are relaxed and any or a combination of the
above Simplifying Assumptions are used as starting conditions for such
a project, there are still significant hurdles that need to be overcome
in designing and constructing an artificial SSR.


\bigskip

{\bfseries
\hypertarget{RefHeading3166306210128}{}From The Physics to the
Metaphysics of the SSRs}

\textit{“SUPPOSE, in the next place, that the person who found the
watch, should, after some time, discover that, in addition to all the
properties which he had hitherto observed in it, it possessed the
unexpected property of producing, in the course of its movement,
another watch like itself (the thing is conceivable); that it contained
within it a mechanism, a system of parts, a mould for instance, or a
complex adjustment of lathes, files, and other tools, evidently and
separately calculated for this purpose; let us inquire, what effect
ought such a discovery to have upon his former conclusion.}

\textit{The first effect would be to increase his admiration of the
contrivance, and his conviction of the consummate skill of the
contriver. Whether he regarded the object of the contrivance, the
distinct apparatus, the intricate, yet in many parts intelligible
mechanism, by which it was carried on, he would perceive, in this new
observation, nothing but an additional reason for doing what he had
already done,-{}-for referring the construction of the watch to design,
and to supreme art…”}

William Paley, \textit{Natural Theology: or, Evidences of the Existence
and Attributes of the Deity}. Beginning of chapter II. \textit{State of
Argument Continued}\textstyleauthori{, 1809}


\bigskip


\bigskip

We have seen that we have very strong reasons to be skeptical that
humans are able at this time to design and build fully autonomous
self-replicators. We have much stronger reasons to be skeptical that
the self replicators that we encounter on Earth are the results of
natural laws combined with random natural events or circumstances.


\bigskip

We have seen that scientists and engineers at NASA or from other
organizations thought about the future of space exploration and created
detailed plans and projects for creating artificial self replicators to
be realized either as self-replicating moon factories or as
self-replicating inter-stellar probes.  The time horizon for the
implementation of these projects is either sometime during the
21\textsuperscript{st} century or farther on in the future. These
projects are very ambitious and detailed enough to emphasize technical
difficulties for their implementations that might be impossible to
solve with the current or near future technologies.


\bigskip

On the other hand we are facing the reality of a vast assortment of
self-replicators populating the planet Earth. The estimated number of
organisms on Earth is between 10\textsuperscript{20} and
10\textsuperscript{30}. There are an estimated 9.7 million varieties of
organisms on Earth. Among those varieties are bacteria, microbes,
fungi, plants, algae, grass, shrubs, trees, insects, mollusks, fish,
birds, mammals. Some live in the waters of the seas, in the water of
the lakes and rivers some live miles deep in the ocean waters. Other
organisms live in ice, in the Earth’s crust or on the Earth’s surface
or some smaller ones in other organisms. We know that the organisms of
the living world constitute layers in the food chain pyramid. Ocean
plankton serves as food for smaller fish and ocean creatures that, in
turn serve as food for larger fish or ocean mammals. On earth, grass
and plants make up the food for rodents, animals and birds. It appears
that for each species of a self-replicator there is a particular food
niche for that species. From our SSR insights we can affirm that the
energy closure, material closure and information closure – so difficult
to achieve for an artificial SSR – are common, ordinary conditions that
are satisfied by the internal design and construction of all these
species of organisms.


\bigskip

Many of these organisms are not simple self-replicators. Many are
significantly more complex than the SSR we investigated and profiled.
They are significantly more complex for various reasons:


\bigskip

\begin{itemize}
\item They are made of multiple cells and multiple cell types. 
\item Although the cell still self-replicates, the organism which is a
structured hierarchy of systems of cells, tissues and organs has a more
sophisticated way to replicate sexually at the whole organism level 
\item Many complex self-replicators are mobile and their mobility in
their medium significantly facilitate their ability to feed and
replicate.
\item Many self-replicators are endowed with a wide spectrum of sensor
organs that allow them to sense the environment either visually,
through smells, through taste, through touching, through sensing
movements or through sensing sounds.
\end{itemize}
How can we make sense of the presence of this immense plethora of
autonomous self-replicators? How can we make sense and explain the
existence of these self-replicators, with rational hierarchical
structuring of their food chain and harmonious integration in the
Earth’s environmental conditions?


\bigskip

\textit{The author suggests the following explanation for
consideration.}


\bigskip

An extremely advanced civilization visited the planet Earth sometime in
the past. This civilization had a Master Designer.  The Master Designer
knew how to design and build self-replicators. He knew that the
biochemistry is the appropriate material basis with the appropriate
scale for the design and construction of these self replicators. He
knew how to use just a few chemical elements: hydrogen, carbon, oxygen,
nitrogen, phosphorus and a few others as the building blocks of an
amazing variety of molecular machinery. He knew how to codify and store
information in the DNA and other complex organic molecules. The Master
Designer populated the Earth with an immense number of self-replicators
from the smallest to the largest. From those that are at the base of
food chain, to those that are at the top of the food chain. From
self-replicators that are made of a single cell to birds, mammals and
the Homo Sapiens, the crown of the Master Designer creation.  He
endowed the Homo Sapiens with a body and a mind. A mind that the Homo
Sapiens uses to contemplate his surroundings to design his own things
like houses, roads, bridges, engines, cars, airplanes and planetary
exploration vehicles. A mind that he uses to explore the plants,
insects, birds and animals in the environment, to study their make and
their behavior .

A mind that started looking into how these living organisms are
constructed and what are the most inner workings of them. A mind that
looks in amazement to our Earth, to our Sun and planets, to our galaxy
and much beyond. A mind that dreams to conquer the galaxy but cannot
stop to ask what advanced civilization left its indelible creative
marks on planet Earth. A mind that can understand the amazing
complexity of the living organisms and remain in awe in the face of the
magnificent creative power of the Master Designer reflected in the
myriad results of his work.


\bigskip

{\bfseries
\hypertarget{RefHeading3168306210128}{}References}

\begin{enumerate}
\item \label{bkm:Ref330755466}Freitas, R.A. Jr., Merkle, R., Kinematic
Self-Replicating Machines,
\url{http://www.molecularassembler.com/KSRM/5.6.htm}, Landes
Bioscience, Georgetown, TX, 2004.
\item \label{bkm:Ref330757844}Stevens, W.M., \textit{Self-Replication
Construction and Computation}, PhD Thesis, The Open University,
\url{http://www.srm.org.uk/thesis/WillStevens-Thesis.pdf}, page iii,
2009
\item \label{bkm:Ref330750347}Stevens, W.M., \textit{Self-Replication
Construction and Computation}, PhD Thesis, The Open University,
\url{http://www.srm.org.uk/thesis/WillStevens-Thesis.pdf}, pages 24-32,
2009
\item \label{bkm:Ref330751664}M Moses. \textit{A physical prototype of a
self-replicating universal constructor}.  Master’s thesis, Department
of Mechanical Engineering, University of New Mexico, 2001.
\item \label{bkm:Ref330752277}Victor Zykov, Efstathios Mytilinaios,
Bryant Adams, and Hod Lipson. \textit{Self-reproducing}\textit{
machines}. \textit{Nature}, 435:163–164, 2005.
\item \label{bkm:Ref330753103}Adrian Bowyer.  \textit{The
self-replicating rapid prototyper - Manufacturing for the masses}. In
\textit{Proceedings of the 8th National Conference on Rapid Design,
Prototyping and} \textit{Manufacturing}. Rapid Prototyping and
Manufacturing Association, 2007.
\item \label{bkm:Ref330756364}K E Drexler. \textit{Engines of Creation:
The Coming Era of Nanotechnology}. Anchor Press/Doubleday, New York,
1986.
\item \label{bkm:Ref331270777}\textstylecitation{Smalley, Richard E.
(September 2001).
}\href{http://www.sciamdigital.com/index.cfm?fa=Products.ViewIssuePreview&ARTICLEID_CHAR=F90C4210-C153-4B2F-83A1-28F2012B637}{{\textquotedbl}Of
Chemistry, Love and Nanobots{\textquotedbl}}\textstylecitation{.
}\textstylecitation{\textit{Scientific American}}\textstylecitation{
}\textstylecitation{\textbf{285}}\textstylecitation{ (3): 76–7}
\item \label{bkm:Ref330845705}Smith, M.,
\textcolor[rgb]{0.2,0.2,0.2}{Scientists Create First
{\textquotesingle}Synthetic{\textquotesingle} Cells, ABC News, May 21,
2010,
}\url{http://abcnews.go.com/Health/Wellness/scientists-create-synthetic-cells/story?id=10708502}
\item \label{bkm:Ref330927503}Synthetic biology in Wikipedia,
\url{http://en.wikipedia.org/wiki/Synthetic_biology}, 2012
\item \label{bkm:Ref330927519}Microelectromechanical systems on
Wikipedia,
\url{http://en.wikipedia.org/wiki/Microelectromechanical_system}  
\item \label{bkm:Ref330927779}Freitas, A.R. Jr., Gilbreath, W.P.,
Editors\textit{, Advanced Automation for Space Missions},
\textit{Proceedings of the 1980 NASA/ASEE Summer Study},
\url{http://www.islandone.org/MMSG/aasm/} 
\item \label{bkm:Ref339395464}\label{bkm:Ref330928919}Freitas, A.R.,
Jr., \textit{A Self-Reproducing Interstellar Probe}, \textit{Journal of
the British Interplanetary Society}, Vol. \textbf{33}, 1980 Gilster,
P., \textit{Via }\textit{Nanotechnology to the Stars} at
http://www.centauri-dreams.org/?p=96
\end{enumerate}
\end{document}
