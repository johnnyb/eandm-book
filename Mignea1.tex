%% FIXME - note which part has the bibliography

\newcommand\migneafigurewide[3]{
	\begin{figure}
		\centering
		\includegraphics[width=6in]{#1}
		\caption{{#2}}
		\label{#3}
	\end{figure}
}
\newcommand\migneafigure[3]{
	\begin{figure}
		\centering
		\includegraphics[width=3in]{#1}
		\caption{{#2}}
		\label{#3}
	\end{figure}
}
\newcommand\mterm[1]{\emph{#1}}
\newcommand\memph[1]{\emph{#1}}

\eandmchapter{The Simplest Self-Replicator, Part~1}{Developing Insights into the Design of the Simplest Self-Replicator and its Complexity: Part~1---Developing a Functional Model for the Simplest Self-Replicator}{Arminius Mignea}{The Lone Pine Software}

\begin{abstract}
\index{self-replication}
\index{biology!cell}
This is the first in a three-part series investigating the internals 
of the simplest possible self-replicator (SSR). 
The SSR is defined as having an enclosure with input
and output gateways and having the ability to create an exact replica
of itself by ingesting and processing materials from its
environment. This first part takes an analytical approach and identifies, one
by one, the internal functions that must operate inside
the SSR to be a fully autonomous replicator.
\end{abstract}

\section{Introduction}

One of the most remarkable characteristics of living organisms is
their ability to self-replicate. There are many forms and
manifestations of self-replication. These forms vary
from the simplest, single-celled organisms to a wide range of
living forms to the most complex organisms, including humans and
other mammals.

\index{origin of life}
One of the most intriguing questions that ordinary people, engineers,
scientists, and philosophers have obsessed over for centuries is how life on Earth
originated and is able
to create
descendants that look like their parents.
Many researchers and scientists
have invested tremendous resources in trying to identify a plausible
natural means by which the simplest forms of life may have been created
from inanimate matter. They have tried to identify, and hopefully
reproduce, a set of events and circumstances that somehow puts
together the basic elements of the simplest entity 
to replicate and thus become a living organism.

\index{self-replication!design}
The goal of this study is to use an engineering approach to develop
insights into the internal design of a simplest possible self-replicator (SSR). 
The SSR is defined for the purpose of this study as
an autonomous artifact that has the ability to obtain material input from
its environment, grow, and create an exact replica of itself. The
replica should ``inherit'' the ability from its ``mother'' SSR to create,
in its turn, an exact copy of itself.

\index{cellular metabolism}
It is important to observe that this simple definition of the SSR accurately mimics the characteristic behavior of many single-celled organisms, at least from the perspective of their ability to self-reproduce. In particular, they are autonomous in regards to their ability to ingest materials from their environment, to use the ingested materials for growth and production of internal energy, and to produce an identical copy of themselves, usually through a two-step process of cloning and division. 

Part~1 and Part~2 of this paper will analyze the process of self-replication as though it is a preliminary study for a research lab tasked to design and build an artificial SSR from scratch. The objectives of this study are as follows:

\begin{enumerate}
\item  Create a top level design of the SSR that identifies all its
functional components with specific characterization of the role of
each functional component, its responsibilities and its interactions
with the other functional components within the SSR framework.
\item  Identify the candidate engineering technologies that could be employed in
the concrete implementation of the SSR and of its functional
components.
\item  Identify the most difficult tasks in constructing the
concrete artificial SSR.
\item  Conclude with an overall estimate of the complexity 
in constructing
the artificial SSR. For a pragmatic estimation of
SSR construction complexity, it may be compared with 
existing, high technology, human-made artifacts.
\end{enumerate}

Because this is a thought experiment on what elements would be required to construct a simple self-replicator and makes use of general knowledge in the field of engineering, references have not been included at the end of part one.

\section{The Two Phases of Self-Replication}

\index{self-replication!phases}
At the highest level, the SSR has the composition 
illustrated in figure \ref{fig:ssr_structure}.

\migneafigure{MigneaSsrStructure}{The SSR Structure}{fig:ssr_structure}

\index{cloning phase|see{self-replication, cloning phase}}
\index{self-replication!cloning phase}
\index{division phase|see{self-replication, division phase}}
The SSR replication process has two main phases---the cloning phase (illustrated in figure \ref{fig:cloning_phase}) and the division phase (illustrated in figure \ref{fig:division_phase}).

\migneafigure{MigneaSsrCloningPhase}{The Cloning Phase in SSR Replication}{fig:cloning_phase}

\migneafigure{MigneaSsrDivisionPhase}{The Division Phase in SSR Replication}{fig:division_phase}

The behavior of the SSR, including basic support functions and the two replication phases, can be outlined as follows:

\index{self-replication!inputs}
\index{self-replication!material extraction}
\index{self-replication!fabrication}
\index{fabrication|see{self-replication, fabrication}}
\index{materials fabrication|see{self-replication, fabrication}}
\index{energy generation|see{self-replication, energy}}
\index{energy distribution|see{self-replication, energy}}
\index{self-replication!energy}
\begin{itemize}
\item Input raw materials and raw parts are accepted by input enclosure
gates.
\item Input raw materials are processed through material extraction into
good materials for fabrication of parts or for energy generation.
\item Energy is generated and made available throughout the SSR.
\index{self-replication!functions}
\item Processed materials are passed to the fabrication function which fabricate parts, components and
assemblies for cloning of all SSR internal elements, creating scaffolding elements for the growing SSR interior, and creating new elements that are added to the growing enclosure.
\item When the cloning of all original SSR internal parts is completed,
the SSR division starts:

\index{self-replication!replication process}
\begin{itemize}
\item The original SSR contents are now at, for example, the north pole
of the SSR enclosure.
\item The cloned SSR contents (the nascent daughter SSR) are now at the
“south pole” of the SSR enclosure.
\item The SSR enclosure and its content now divide at the equatorial
plane and the separate mother SSR (at north) and daughter SSR (at south)
emerge.
\end{itemize}
\end{itemize}

\section[Identifying SSR Capabilities]{Identifying SSR Capabilities as Specific Functions}

By conducting a step-by-step analysis of what must be happening inside and at
the periphery of the SSR, its growth and replication abilities can be characterized.

\subsection[The SSR enclosure and its input and output gateways]{The SSR enclosure and its input and output gateways}

\index{self-replication!inputs}
Two reasonable assumptions will be made about the SSR. The first assumption is
that the SSR is
comprised of an enclosure that has the role of separating the SSR from its
environment. Secondly, the surface of this enclosure has
openings that are used to accept good
substances into the SSR---raw materials and raw parts from the SSR environment. 
These openings will be called \mterm{input gateways}. There are also
openings used by the SSR to expel from inside the SSR refuse materials
and parts that result from certain transformation/fabrication processes.
These openings will be called \mterm{output gateways}.

\subsection[The input flow control function]{The input flow control
function}

\index{self-replication!flow control}
\index{flow control|see{self-replication, flow control}}
There is one primary question
for the input gateways: Are all the raw materials and raw parts that
exist or touch the outside of the enclosure good for the SSR processes?
Certainly, they are not. The SSR and its input gateways must feature some
ability to select or reject substances, materials,
and parts that are outside the SSR interior and determine whether they should enter. 
For this paper, this feature of the SSR will be called the \mterm{input flow control function}.

\subsection[The raw materials and parts catalog]{The raw materials and
parts catalog}

\index{self-replication!material extraction}
The next question to be answered
is, how does the SSR knows which are good raw materials and parts
and which are bad materials and parts? The SSR must
possess a catalog of good raw materials and parts that will be the
informational basis on which the input gateways will open or stay
closed. This catalog will be referred to as the \mterm{raw materials and parts catalog}.

\subsection[The materials and parts identification function]{The
materials and parts identification function}

\index{self-replication!parts identification|(}
The next question is how the SSR
will recognize and accurately identify a material or part at an input
gateway as good or bad? That is not a trivial ability. 
It is a way to
determine the nature of the materials and parts its input gateways are
exposed to. This ability may be supported by a set of material probing
procedures and processes. This SSR ability will be called the
\mterm{materials and parts identification function}. 
The complexity of this capability can be compared to the probes on the Martian rover that were used to analyze soil samples for particular compounds.

\subsection[The systematic labeling/tagging of all raw materials, raw parts, and fabricated parts and components]{The systematic labeling/tagging of all raw materials, raw parts, and fabricated parts
and components}

An input gateway,
assisted by the raw materials and parts identification function,
determines that a piece of raw material is one of the good materials
recorded in the good raw materials and parts catalog. This piece is
going to be admitted into the SSR and transported to a particular place
for processing or possibly to a temporary storage location followed by
processing.  In order to do this successfully, the SSR needs
to tag, or label, this piece so that its nature, once
determined at the input gateway, is available for
subsequent processing stations or storage stations in the SSR. Therefore,
any raw material or part that is allowed to enter the SSR, once
its nature is identified, is immediately tagged or labeled using a system
similar to the bar codes or RFIDs (radio-frequency identification)
where the code used is one of the codes in the catalog of raw materials
and parts. This systematic labeling and tagging of all accepted
materials and parts will be considered another responsibility of the
\mterm{materials and parts identification function}.  Additionally,
during the SSR growth and cloning phases, the SSR will fabricate new parts, components, and assemblies
using either raw materials and raw parts or previously fabricated
parts, components, and assemblies. The point is that the 
\mterm{materials and parts identification function}
will be responsible for tagging or labeling not
only raw materials and parts accepted inside SSR, but also all
fabricated parts, components, and assemblies. Therefore,
all elements inside the SSR and all SSR parts should bear
a permanent identification tag.
\index{self-replication!parts identification|)}

\subsection[The catalog of fabricated parts, components and assemblies]{The catalog of fabricated parts, components and assemblies}

\index{self-replication!fabrication|(}
This raises another
important aspect for the SSR design: The SSR must possess not only an
exhaustive catalog of all raw materials and raw parts, but also a
\mterm{catalog of all fabricated parts, components and assemblies}
with a unique identifier for each \memph{type} of such elements. 

\subsection[The bill of materials function]{The bill of materials function}

The automated fabrication
processes will need additional informational support. 
This capability can be called the \mterm{bill of materials function}. It is supported by
an exhaustive catalog with entries that specify  the list of materials required to fabricate
each and every 
fabricated part, component,and assembly.
For each item in this
list, the quantity of those materials must be also
specified. The \mterm{bill of materials} is an \memph{informational function} of
the SSR. Like all SSR information functions, it has two components: a specific catalog (or database),
 and a set of information access sub-functions to search, read, write, update, or delete specific entries in 
the associated catalog that can be accessed by all other SSR functions.  

In this case, the catalog
is the bill of materials catalog.  
Below is an example of what an abbreviated
entry for a “power supply enclosure” part may look like in such a catalog:

%\begin{supertabular}{|m{1.3094599in}m{1.3573599in}m{0.9837598in}m{0.6705598in}m{0.82335985in}m{1.0323598in}|}
\begin{table}[h]
\caption{Example Entries in the Bill of Materials Catalog for the ``Power Supply Enclosure'' part}
\begin{center}
\begin{tabular}{| l l l l l l |}
\hline
\textbf{Part name} &
\textbf{Part ID} &
\textbf{Flags} &
\textbf{Count} &
\textbf{Qty} &
\textbf{Dimensions}\\
Sheet metal $\frac{1}{16}$ &
ID-02409
 &
$Q + D$
 &
- &
1.2
 &
16x24
\\
Screws $\frac{1}{8}*2$ &
ID-01670
 &
$C$
 &
8 &
-
 &
-
\\
Washers $\frac{1}{4}*2$ &
ID-05629
 &
$C$
 &
8 &
-
 &
-
\\\hline
%\end{supertabular}
\end{tabular}
\end{center}
\caption*{The \textit{flags} field tells what properties are specified for the part.  Q=quantity, C=count, D=dimensions}
\end{table}

\subsection[The fabrication material extraction function]{The fabrication material extraction function}

Some of the raw materials
admitted inside the SSR cannot be used directly by the SSR fabrication
processes. They need to be transformed into fabrication materials
through one or more specific processes. An example, which is not necessarily
related to our artificial SSR, is the fabrication
of steel (fabrication material) from iron ore and coal (raw materials). The SSR's ability to
extract fabrication materials from sets of raw materials and parts is
the \mterm{fabrication material extraction function}.  Fabrication
materials are registered in the \mterm{catalog of fabrication materials} 
while every process that is used
to extract fabrication materials from raw materials and parts is
documented in a \mterm{fabrication material extraction process catalog}.
\index{self-replication!fabrication|)}

\subsection[The supply chain function]{The supply chain function}

\index{self-replication!supply chain}
One can now consider the case when during the
cloning phase the SSR must fabricate a component of type A.  The bill
of materials entry for a component of type A specifies that its
“fabrication recipe” requires 2 raw parts of type X and 4 raw parts of
type Y.  The SSR will need to be able to coordinate the input
gateways to admit prior to the cloning process required quantities of parts X and Y,
creating stock in a SSR stock room so that the fabrication of
component A can go smoothly.  This will allow the fabrication of the component to depend less 
on the available raw parts at any given moment at the enclosure input
gateways. This SSR ability will be referred to as the \mterm{supply chain function}. 
This function is responsible for interacting with
fabrication processes within the SSR to gather information prior to
fabrication regarding what raw materials, raw parts, or fabricated
parts are needed. It will then command other functions (i.e., the input flow control function 
and the material and parts identification function) to admit, supply, and stock
those elements within the SSR.

\subsection[The energy generation and distribution function]{The energy generation and distribution function}

\index{self-replication!energy distribution}
All the machines inside the SSR need
an energy source to perform their work. Thus, the SSR must have the ability
to produce energy from the appropriate raw materials and raw parts.
This ability will be
named the \mterm{energy generation and distribution function} since it
has the responsibility not only to generate energy, but also to manage
and distribute energy to the energy consumers within the SSR. Importantly,
the catalog of raw materials and raw parts, as
well as the catalog of fabricated parts, may contain entries that are
marked as elements used for energy generation and/or distribution.
Also the \mterm{catalog of processes} will contain entries detailing the
material processes that are used
to generate energy from the energy-marked materials and parts. The
supply chain function is responsible for managing the timely supply of
materials and parts not only for fabrication but also for energy
generation and transport.

\subsection[The transport function]{The transport function}

\index{self-replication!material transport|(}
The functioning SSR features
multiple sites where specific actions happen. These
sites will be distributed spatially inside the SSR or on its enclosure
and will have well-established positions relative to the elements that
maintain the three dimensional structure of the SSR as it grows 
(named \mterm{scaffolding elements}). For example, there are input gateway sites,
possibly some stock room sites, fabrication sites,
and assembly sites where the elements of the growing clone
inside the SSR are being put together by some machinery. The SSR must
have the ability to carry various elements between sites. This
ability is named the \mterm{transport function}. It may employ
specific means of transport, such as conduits, avenues, conveyors,
etc., that are adequate for the particular elements that are being transported and
the particular sites within the SSR.  An
important aspect of SSR activity is the transport of information
between producers and consumers within the SSR. For this reason, the
transport function is also responsible for \memph{transporting information} 
between the SSR sites, at least for the provision of the
physical, lower layers of the transport of information.

\subsection[The manipulation function]{The manipulation function}
Another important ability that the SSR needs is the
\mterm{manipulation function}. This
function consists of the ability to handle, grab, or manipulate raw
materials, raw parts, fabricated parts, fabricated components, and
fabricated assemblies. For example, this ability is needed
to take a raw material or raw part admitted at an input gateway and
place it on a conveyor that goes to a stock room or a
fabrication site. Once there, another manipulator will grab the material or
part and place it in a specified position in the stock room or place it
on a fabrication bench or machinery. Manipulation examples abound,
since no matter what elements are processed, transported, fabricated,
assembled or pushed out of an output gateway, there is a need to
adequately handle those elements.
\index{self-replication!material transport|)}

\subsection[The fabrication function]{The fabrication function}

\index{self-replication!fabrication|(}
Since the SSR must be able to
clone its core elements, its enclosure, and its scaffolding elements,
it is absolutely necessary for the SSR to have a \mterm{fabrication function}. 
This function is the ability to fabricate exact copies of
all parts, components, and assemblies that exist inside a mature SSR
or on its enclosure. In other words the fabrication function must be
able to fabricate all machinery inside the SSR, including fabrication
machinery. Since all the elements that reside inside the SSR must be
copied (cloned) and various types of information and software elements
also reside inside the mature SSR, 
the fabrication function must also have the ability to
accurately copy information and software.

\subsection[The assemblage function]{The assemblage function}

Another capability that must
reside within the SSR is the \mterm{assemblage function}. This function allows
the SSR to assemble parts into
components and assemblies that increase in complexity. The assemblage
function is strongly related to the fabrication function. While these two
functions can be seen as two sides of the same coin, it makes sense
to see them as distinct functions where the fabrication function
creates new parts from raw materials and raw parts through special
processes (e.g., metal machining) while the assemblage function puts
fabricated parts together into more and more complex assemblies.
The assemblage function may be needed, for example,
to erect and expand the scaffolding and the enclosure during SSR
growth along with creating assemblies of smaller components
and parts.
\index{self-replication!fabrication|)}

\section{Additional SSR Functions}

\subsection[The recycling function]{The recycling function}

\index{self-replication!recycling}
The SSR functions discussed so
far provide specific assistance for the ingestion of new raw
materials and parts into the SSR and SSR growth during the cloning phase
based on continuous production of energy and planned
fabrication of the elements of the daughter clone growing inside the
expanding SSR enclosure. As in any process that performs fabrication
and construction of new parts, there will be residual raw materials
and raw part fragments.  The SSR must be designed to carefully control
the growth of the internal and enclosure elements. It cannot
grow without limits or in an uncontrolled manner. In order to achieve
this objective, the SSR must:

\begin{itemize}
\item Re-introduce in the fabrication and growth cycles certain elements
of the residual raw materials, raw parts, or raw part fragments that
can be reused.
\item Identify and specifically mark as refuse certain residual elements
that cannot be recycled and then expel them as refuse through the
output gateways.
\item Provide specific processes to clean the SSR
interior fabrication and transport spaces such that fabrication and
assembly of the cloned parts is not affected and the SSR
maintains the proper structure during the cloning and division phases.
\end{itemize}

Most of the above responsibilities pertain to the \mterm{recycling function}.

\subsection[The output flow control function]{The output flow control
function}

\index{self-replication!flow control}
The recycling function controls
the \mterm{output flow control function}, which is the ability to
control the enclosure \memph{output gateways} that force out of the SSR
the raw materials, raw parts, and raw part fragments \memph{marked as refuse} 
by the recycling function.

\subsection[The construction plan function]{The construction plan
function}

\index{self-replication!fabrication|(}
It has already been noted that the bill of
materials function provides for each fabricated part, component, or
assembly of the SSR a list of raw materials, parts, and sub-components
that are needed for fabrication of that element. Thus, an entry in the
bill of materials information catalog is similar in concept to the
list of ingredients for cooking a meal. Besides the list of
ingredients, the recipe for a meal contains an ordered list of steps
needed to prepare the meal. In a similar manner, the SSR must store
descriptive information for all construction steps needed to fabricate
each SSR element. The SSR's ability to store and make
accessible detailed information about the set of fabrication steps and
processes needed for the fabrication of each SSR element (part,
component, assembly of components, up to including the SSR itself) is
called the \mterm{construction plan function}.

\subsection[The construction plan information catalog]{The construction
plan information catalog}

The \mterm{construction plan function} has an associated \mterm{construction plan information catalog}. 
This catalog has an entry for each fabricated element of the
SSR. A fabricated element can be a simple fabricated part (i.e., fabricated
from a single good material). A fabricated element can also be a
component which, in this context, refers to an element fabricated
through the assemblage of two or more fabricated parts. A fabricated
assembly is even more complex: it is fabricated from multiple simple
parts and one or more components and possibly one or more (sub)
assemblies. The mature SSR (before starting the cloning process) is a
particular case of a fabricated assembly. 
Another example of a fabricated assembly, and the most complex example in this case, is the SSR that contains both the mother core elements and the daughter elements just before division begins.


Each entry in the construction plan catalog contains a reference to the entry in the bill of materials catalog for the same element (to access the parts and components needed for the element fabrication) and an ordered sequence of fabrication and assembly steps.

For each fabrication/assembly step the following information may be
provided:

\begin{itemize}
\item The spatial assembly or placement instruction of parts and/or materials
involved in the step (i.e., how to spatially place a part/component, P,
relative to the assembly under construction on the work bench before a
fabrication step)
\item The type and detailed description of each fabrication/assembly
process executed during the current fabrication step
\item Technological parameters of the fabrication/assembly process (e.g.,
ambient temperature, length of process, etc.)
\item List of residual parts/materials resulting from the process or particular step
\item Part manipulation steps (This includes x, y, z starting point, x, y, z
ending point, part rotation, axis specification, or translation movements.)
\item Any fabrication step verification procedure to determine if the
step completed successfully within accepted parameters or if the
fabrication step was a failure.
\item Recovery action list in case a fabrication step fails with a
specific verification error
\end{itemize}

In short, the rationale for the nature, structure, and extent of
information items stored for each step of a \mterm{fabrication plan entry} 
is to provide support for full automation of that element of
fabrication. And, as suggested above, \mterm{fabrication steps} can be
of a very large variety. The nature of the fabrication process, as well
as the nature of fabrication steps, depends on the material basis that will be
selected for the design and implementation of the artificial SSR. The
alternative material bases that can be realistically considered for
creating an artificial SSR are discussed in Part~2 of this study.

Here are several examples of the nature of a fabrication step:

\begin{itemize}
\item A positioning step (i.e., a part or component being placed in a particular position relative
to the semi-assembled element in preparation of another step)
\item An assemblage step
\item A metal machining step
\item A chemical process step
\item A thermal process step
\item An electrolytic process step
\item A nanotechnology assemblage step
\item An information file copy step
\item A fabricated component (assembly) test step
\end{itemize}

The next issue to solve requires devising
a way to track the
progress of the cloning and division steps. This is provided by the
\mterm{construction status function}. This function uses an
information catalog that is similar to the construction plan catalog named the
\mterm{construction status catalog}. It has 
the same list of element entries as the construction plan
catalog and describes the same hierarchical composition of each element
(part, component, assembly) in sub-elements.  Each entry in this
catalog has construction status information that reflects the current
construction status of that entry and can have values like ``not-started,'' 
``started,'' or ``completed.''

In a similar manner each fabrication step of an entry has a current
construction status field that is also used to mark and keep track of
the fabrication/construction status for that element at the fabrication
step level.
\index{self-replication!fabrication|)}

\subsection[The SSR variable geometry]{The SSR variable geometry}

\index{self-replication!growth and development|(}
Another issue in developing the artificial SSR is that the SSR has variable
geometry.  The SSR has variable geometry because, first, the mature SSR must 
grow in volume and enclosure surface during the cloning phase to make space 
in its interior for the growing clone. Second, the geometry changes even more 
radically when the division phase
starts and culminates with the complete division of the original SSR in
both the mother SSR and the daughter SSR.

The SSR variable geometry presents many considerations and challenges that must
be carefully considered by the SSR design and by the hypothetical
implementation of the artificial SSR.  
The first consideration is the structure and composition (texture) of the SSR.  This
must be designed to allow the following:
\begin{itemize}
\item Surface area growth as the SSR interior grows in volume during the
cloning phase. This may require a design that allows selective
insertion of new enclosure parts/elements in between existing
parts/elements and a way to link or connect each new
element with its neighboring elements.
\item Insertion of new input gateways and output gateways while the enclosure grows.
\item Division of the enclosure into two separate enclosures (one for
the mother SSR and one for the daughter SSR)  with each enclosure
carrying its separate sets of input and output gateways, 
scaffolding, and interior elements.
\end{itemize}

Next, special design provisions must be made for the interior SSR
scaffolding. The SSR scaffolding is made of structural
elements (e.g., pylons, walls, supports, connectors, etc.) that are needed to
maintain the three dimensional structure and integrity of the enclosure
and of the SSR interior space(s). The scaffolding design and its
elements need to be conceived such that, first, the scaffolding elements may
change size as the
interior of the SSR grows (during cloning) or shrinks (during division);
second, the spacing between scaffolding elements and their connectors may
also grow or shrink (during cloning or division phases).  
Finally, the design of the SSR must also make provisions for the growth,
variable geometry, and dynamic restructuring and re-linking of any SSR
transport, conduits, paths or communication lines during the cloning
and division phases.

The variable geometry means that the SSR design must make specific,
detailed provisions for the entirety of its spatial evolution. This includes all geometrical
definition points or trajectories (in the x, y, and z axes) of all variable
elements of the SSR enclosure, scaffolding and interior. These spatial
trajectories need to be harmoniously and coherently coordinated with
all fabrication and assemblage steps of the cloning and division
phases.
\index{self-replication!growth and development|)}

\subsection[The Communication and Notification Function]{The Communication and Notification Function}

\index{self-replication!signalling and communication}
This function is responsible for
providing and managing the information communication and notification
machinery and mechanisms between the command centers and execution
centers of the SSR. For example, the fabrication control 
function, acting as a command center, may send a command as a specifically
encoded information “package”  to the fabrication and assemblage
functions to build a particular component of the daughter clone. In
this circumstance, the fabrication and the assemblage functions operate
as execution centers for the command. When the fabrication of the
requested component is completed by the fabrication function, it will
send a specifically formatted notification information package back to
the fabrication control function indicating that the specific
command for the fabrication of the specific element was successfully
completed. Within the same example scenario, the fabrication function
will send, in its turn,---this time playing a command role itself---a
command to the supply-chain function (i.e., the executor entity) to trigger
the transport of the needed fabrication ingredients for the clone
element to the fabrication site. The \mterm{communication and notification function} 
will need to be deployed ubiquitously throughout the
SSR to allow communications/notifications between various functions and
machinery operating all over the SSR.

\section[The higher level SSR functions]{The higher level SSR functions}

The SSR functions that are
described in subsequent sections are the highest level functions of the
SSR. They accomplish their goals by coordinating and choreographing the
lower level functions described in the previous sections.

\subsection[The scaffolding growth function]{The scaffolding growth function}

\index{self-replication!growth and development|(}
The \mterm{scaffolding growth function} 
is responsible for managing the construction, growth, and
position change of the SSR scaffolding elements during the cloning and
the division phases of the SSR replication process. This function needs to
manage the variable geometry of the scaffolding elements in
synchronization and coordination with the other spatial changes of the
SSR both on its enclosure and within its interior.

\subsection[The enclosure growth function]{The enclosure growth
function}

The \mterm{enclosure growth function} is responsible for managing
the construction, growth, and shape
change of the SSR enclosure as well as the coordinated addition of
input and output gateways on the enclosure during the cloning and
division phases of the SSR replication process. As mentioned before, this
function needs to manage the variable geometry of the enclosure, the
dynamic shifting of the gateways on the enclosure's surface, and the
enclosure's radical shape changes during the division of the SSR
into the mother and daughter descendants.
\index{self-replication!growth and development|)}

\subsection[The fabrication control function]{The fabrication control
function}

\index{self-replication!fabrication}
The \mterm{fabrication control function} is responsible for the construction, assemblage and variable
geometry management of all interior elements, in particular those
related to the cloning portion of the SSR involved in the cloning and the
division phases.  Like the two preceding growth
functions, this function coordinates activities of the fabrication
function, assemblage function, construction plan function, recycling
function, and other lower level functions. 

\subsection[The cloning control function]{The cloning control function}

\index{self-replication!cloning phase}
\index{self-replication!process control|(}
The \mterm{cloning control function} is responsible for coordinating the whole cloning phase of the
growing SSR. It coordinates the cloning and the
growth of all of the involved SSR compartments through tight control,
synchronization, and coordination of the scaffolding growth, enclosure
growth, and fabrication control functions. In particular, this function is responsible
for starting the cloning process, monitoring its development,
and accurately determining when the cloning process is complete. One
particular responsibility of the cloning control function is the
\memph{cloning of the information} stored in the mother SSR. This
information cloning is performed at the end of the cloning phase when
all internal machinery, internal scaffolding and enclosure elements of
the mature SSR are completely cloned and constructed as part of the
nascent daughter SSR. The information is cloned by systematically,
accurately, and completely copying all information catalogs from
resident machinery of the mature SSR into the corresponding new
machinery of the clone part.

\subsection[The division control function]{The division control function}

\index{self-replication!division phase}
The \mterm{division control function} has full control of the division phase of the SSR replication
process. It manages the SSR division through specific commands sent to
the scaffolding growth, enclosure growth and fabrication control
functions.  In particular, this function is responsible for starting the
division process, for choreographing its development on all SSR
compartments (enclosure, scaffolding and core), and accurately
determining when the division process is complete. One particular
responsibility of the division control function is to “start the
engines” of the nascent daughter SSR. Just before the moment the
division is complete the division control function must send a command
to the daughter SSR to start its own machinery and control functions.
When the division completes, the separated daughter SSR
becomes a mature, fully functioning autonomous SSR ready to start its
own replication process. The separated “mother SSR” can start, in its
turn, a new replication cycle.

\subsection[The replication control function]{The replication control
function}

The \mterm{replication control function} is the highest level SSR function. It is responsible 
for
accomplishing the full SSR replication cycle by coordinating and
choreographing its two phases: cloning and division.  It does this through
corresponding control and coordination of the \mterm{cloning control function} 
and the \mterm{division control function}.

Metaphorically speaking the replication control function implements the
two significant \memph{SSR designer commandments}: grow and multiply.
\index{self-replication!process control|)}

\subsection[The SSR Function Dependencies Diagram]{The SSR Function
Dependencies Diagram}

\index{self-replication!dependencies|(}
Figure \ref{fig:functions_and_dependencies}
provides an overall diagram illustrating
the identified set of functions present in the SSR and some of the
dependencies between these functions.

This figure depicts the main dependencies and interactions between the identified SSR
functions, but is limited to only the primary dependencies and interactions.

The communication and notification function is represented separately
since it relates to and is used by almost all other SSR functions.
Its ubiquity throughout the SSR functions is due to the need for information communication between functions as
well as notification (another form of information exchange).

As already mentioned the relationships and dependencies between the
functions are more complex than depicted in the diagram. For
example, the transport function depends on the energy generation and
transport function although this dependency is not depicted in the
diagram.

\migneafigurewide{MigneaSsrFunctionsAndDependencies}{SSR Functions and their Dependencies}{fig:functions_and_dependencies}
\index{self-replication!dependencies|)}

\section{The Type and Nature of SSR Components}

\index{self-replication!design!components|(}
At this point it makes sense to break down the previous discussions
into the general conceptual categories of the artificial SSR's components
to provide an overview of what needs to be accomplished for its creation.
Part~2 of this paper will go into more concrete details about the
physical components that may make up the artificial SSR.  The categories
themselves are grouped by functional area.

\subsection{Information Processing Component Categories}
\begin{itemize}
\item Information storage and access
\item Information processing
\item Information coding and decoding
\item Information transport, communication and notification
\end{itemize}

\subsection{Materials Component Categories}
\begin{itemize}
\item Material identification
\item Material transport and manipulation
\item Material processing
\item Mechanical and chemical transformation of materials
\item Material fabrication and assemblage
\end{itemize}

\subsection{Energy Component Categories}
\begin{itemize}
\item Energy generation
\item Energy transport
\item Energy conversion
\item Energy distribution and management
\end{itemize}

\subsection{Environmental Component Categories}
\begin{itemize}
\item Environment sensing
\item Environment (local) control
\end{itemize}

\subsection{Construction Component Categories}
\begin{itemize}
\item SSR construction plan representation
\item SSR dynamic 3D evolution representation
\item SSR construction status representation
\item SSR parts inventory representation
\end{itemize}
\index{self-replication!design!components|)}

\section{The SSR and Its Information Catalogs}

\index{self-replication!design!information catalogs|(}
Various types of information catalogs (databases
or repositories) that together make up the information base of the SSR have already been presented.
This section contains a list of all of the types of information catalogs
identified thus far. It is quite probable that after a more in-depth analysis of the topic, there may be the need for additional types of information catalogs.

The SSR information catalogs need to thoroughly, systematically, and coherently capture 
all information describing in detail each element of the SSR during its
full life-cycle, all relationships between these elements captured in
construction plans (body plans), and all fabrication and assemblage
procedures and processes.

This requires making certain simplifying assumptions in order to make the presentation of ideas easier to understand. This means that every piece of data or information that the SSR design must capture cannot be fully dissected here.
A real information repository not only contains
conceptual lists of tables (catalogs) of items of the same nature but
also lists of the various relationships that exist between the
items in different tables (catalogs).  For example, there are certain
relationships between the items in the catalog of raw materials and the
items in the catalog of the bill of materials. There are
different sets of relationships between the entries in the catalog of
the construction plans and the entries in the catalog of the bill of
materials and the entries in the catalog of processes.  For now, this
is just a list of the core catalogs that have been identified so far:

\begin{itemize}
\item The catalog of raw materials
\item The catalog of raw parts
\item The catalog of fabrication materials
\item The catalog of raw materials identification procedures and
processes
\item The catalog of raw parts identification procedures and processes
\item The catalog of fabrication materials extraction procedures and
processes
\item The catalog of energy generation procedures and processes
\item The bill of materials catalog
\item The catalog of internal machinery
\item The catalog of construction plans
\item The catalog of fabrication procedures and processes
\item The catalog of assembly procedures and processes
\item The catalog of fabrication verification procedures
\item The catalog of fabrication errors handling procedures
\item The catalog of energy consumptions
\item The catalog of recycling elements and procedures
\item The catalog of construction statuses
\end{itemize}
\index{self-replication!design!information catalogs|)}

\section{Conclusion}

This part provides an abstract description of the minimal core components,
processes, information stores, and structural requirements of an artificial 
self-replicating system.  Part~2 will cover physical considerations for
implementing such a design, and Part~3 will cover speculative ideas for
what the existence of self-replicative processes in nature indicates on the
larger scale.
