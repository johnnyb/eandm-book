%% FIXME - note which part has the bibliography

\chapter[The Simplest Self-Replicator, Part 1]{Developing Insights into the Design of the Simplest Self-Replicator and its Complexity: Part 1---Developing a Functional Model for the Simplest Self-Replicator}

\chapterauthor{Arminius Mignea}
\chapteraffiliation{The Lone Pine Software}

\begin{abstract}
This paper investigates the internals of the simplest-possible self
replicator (SSR). The SSR is defined as having an enclosure with input
and output gateways and having the ability to create an exact replica
of itself by just ingesting and processing materials from its
environment. The study takes an empirical approach and identifies one
by one the internal capabilities or functions that must operate inside
the SSR to provide its fully autonomous replication behavior. The
analysis considers various significant aspects that confronts the
design and construction of an artificial, concrete SSR: the material
basis of its construction; the effects of the variable geometry of the
SSR during its growth through the cloning and then division phases; the
three closure rules that must be satisfied by the SSR: energy closure,
material closure and the information closure. The highest technical
challenges that need to be faced by the design and construction of the
artificial SSR are discussed. The emerging complexity of the artificial
SSR is depicted using a metaphorical comparison of the replicating SSR
with a full city populated only by automated machinery and robots that
systematically and orderly construct the new city quarters identical
with the old city quarters with no help from outside but only the
construction materials entering through the city gateways. An
evaluation is made if the current level of technology is good enough
for the successful completion of a “design and construct an artificial
autonomous SSR” project either with a nano biochemical basis or a macro
material basis. The insights of the internal SSR design are used to
reflect upon the search for a materialist explanation of the origin of
life. The paper concludes contrasting the futuristic NASA projects for
a self-replicating factory on the Moon and an inter-stellar probe for
population of the galaxy with the presence on planet Earth of a
structured hierarchy of eco systems of self-replicators that provide
feeding niches for millions of varieties of its self-replicators. 
\end{abstract}

\section{Introduction}

One of the most remarkable characteristics of the living organisms is
their ability to self-replicate. There are many forms and
manifestations of the self-replication capability, and these forms vary
from the simplest, single-celled organisms through a wide range of
living forms up to the most complex organisms including humans and
other mammals.

\subsection[In search of the Simplest Self{}-Replicator (SSR)]{In search
of the Simplest Self-Replicator (SSR)}

One of the most intriguing questions that ordinary people, engineers,
scientists and philosophers obsessed over for centuries was how life
originated and how this ability of living organisms to create
descendants that look like their parents appeared on planet Earth. 

\subsection[Efforts to trace a materialistic emergence of life and self
replicators]{Efforts to trace a materialistic emergence of life and
self replicators}

Many researchers and scientists
are investing tremendous resources, in trying to identify any plausible
natural means by which the simplest forms of life may have been created
from inanimate matter. They are trying to identify and hopefully
reproduce a lucky set of events and circumstances that somehow put
together the basic elements and parts of the simplest entity that began
to manifest the ability to replicate and thus become a living organism.

\section{Goals, Assumptions and Requirements}

The goal of this study was to use an engineering approach in developing
insights into the internal design of a simplest possible self
replicator (SSR). The SSR is defined for the purpose of this study as
an autonomous artifact that has the ability to get material input from
its environment, grow and create an exact replica of itself. The
replica should “inherit” the ability from its “mother” SSR to create in
its turn an exact copy of itself.

It is important to observe that this simple definition of the SSR mimics
accurately the characteristic behavior of many single cellular
organisms – at least from the point of view of their ability to
self-reproduce. In particular they are autonomous in regards to:

\begin{itemize}
\item their ability to ingest materials from their environment
\item their ability to use the ingested materials for growth and
production of internal energy
\item their ability to produce an identical copy of themselves usually
through a two-step process of cloning and division.
\end{itemize}

The more specific work hypothesis for this research can be described in
the following terms. A hypothetical research lab is launching a project
to design and build an artificial SSR. The first phase of this project
is to ask a team of engineers to perform a preliminary study with the
following objectives:

\begin{enumerate}
\item  Create a top level design of the SSR that identifies all its
functional components with specific characterization of the role of
each functional component, its responsibilities and its interactions
with the other functional components within the SSR framework.
\item  Identify the candidate engineering technologies to be employed in
the concrete implementation of the SSR and of all its functional
components.
\item  Identify the highest level of difficulties in constructing the
concrete artificial SSR
\item  Conclude with an overall estimate of the complexity of the
project to construct the artificial SSR. For a pragmatic estimation of
SSR construction complexity, it may be compared with selected,
existing, top technology human-made artifacts.
\end{enumerate}

\section{The Two Phases of the Self Replication}

At the highest level the SSR has the following composition as
illustrated in Figure \ref{fig:ssr_structure}.


\begin{figure}
	%% FIXME - include graphics
	\caption{The SSR Structure}
	\label{fig:ssr_structure}
\end{figure}

The SSR replication has two main phases:

\begin{enumerate}
\item The cloning phase is illustrated in  The Cloning phase in SSR
replication
\item The division phase is illustrated in  The Division phase in SSR
replication
\end{enumerate}


\begin{figure}
	% FIXME - need image
	\caption{The Cloning Phase in SSR Replication}
	\label{ref:cloning_phase}
\end{figure}

\begin{figure}
	% FIXME - need image
	\caption{The Division Phase in SSR Replication}
	\label{ref:division_phase}
\end{figure}

The behavior of the SSR can be described succinctly as follows:

\begin{itemize}
\item Input raw materials and raw parts are accepted by input enclosure
gates
\item Input raw materials are processed through material extraction into
good materials for fabrication of parts or for energy generation
\item Energy is generated and made available throughout the SSR
\item Fabrication function starts to fabricate parts, components and
assemblies for: 

\begin{itemize}
\item Cloning (creating copies) of all SSR internal elements
\item Creating scaffolding elements for the growing SSR interior
\item Creating new elements that are added to the growing enclosure
\end{itemize}
\item When the cloning of all original SSR internal parts is completed,
the SSR division starts:

\begin{itemize}
\item The original SSR content is now at (for example) the “north pole”
of the SSR enclosure
\item The cloned SSR content (the “nascent daughter SSR”) is now at the
“south pole” of the SSR enclosure
\item The SSR enclosure and its content now divide at the “equatorial”
plane and the separate “mother” (at North) and daughter (at South) SSR
emerges.
\end{itemize}
\end{itemize}

\section{Identifying SSR Capabilities as Specific Functions}

\subsection[An empirical approach]{An empirical approach}

We will build step by step some
understanding of what must be the elements and the abilities of the
SSR, by conducting an analysis of what must be happening inside and at
the periphery of the SSR so that its growth and replication will occur.


\subsection[The SSR enclosure and its input gateways and output
gateways]{The SSR enclosure and its input gateways and output gateways}

\hypertarget{RefHeading3050306210128}{}We will assume that the SSR is
made by an enclosure that has the role to separate the SSR from its
environment. We will assume that on this enclosure there are some
openings (gateways) that are used to accept inside SSR some good
substances: raw materials and raw parts from the SSR environment. We
will call these openings \textbf{input gateways}. There are also
openings used by the SSR to push out from inside some refuse materials
and parts that result from certain transformation/fabrication processes
inside the SSR. We will name these openings \textbf{output gateways}.

\subsection[The input flow control function]{The input flow control
function}

\hypertarget{RefHeading3052306210128}{}There is one relevant question
for the input gateways. Are all the raw materials and raw parts that
exist or touch the outside of the enclosure good for the SSR processes?
Certainly they are not. SSR and its input gateways must feature some
ability to be selective in accepting or rejecting substances, materials
and parts that are outside and may enter the SSR interior. We will
identify this needed feature of the SSR as its \textbf{input flow
control function.}

\subsection[The raw materials and parts catalog]{The raw materials and
parts catalog}

\hypertarget{RefHeading3054306210128}{}The next question to be answered
is how the SSR “knows” which are “good” raw materials and “good” parts
versus “bad” materials and parts? The answer is that the SSR must
possess a catalog of good raw materials and parts and this will be the
informational basis on which the input gateways will open or stay
closed. Let’s name this catalog the \textbf{raw materials and parts
catalog.}

\subsection[The materials and parts identification function]{The
materials and parts identification function}

\hypertarget{RefHeading3056306210128}{}The next question is how the SSR
will recognize and accurately identify a material or part at an input
gateway as a good or bad one? That is not a trivial ability, but the
SSR must possess it. This ability can be described as a way to
determine the nature of materials and parts its input gateways are
exposed to. This ability may be supported by a set of material probing
procedures and processes. We will name this SSR ability the
\textbf{materials and parts identification function}. This capability
is at least on the order of complexity of certain probes with which the
Martian rover was equipped to analyze the Martian soil in order to
decide if certain compounds are present or not on the surface of that
planet.

\subsection[The systematic labeling/tagging of all raw materials, raw
parts and fabricated parts and components]{The systematic
labeling/tagging of all raw materials, raw parts and fabricated parts
and components}

\hypertarget{RefHeading3058306210128}{}The next issue that requires
clarification is the following. Let’s assume that an input gateway
assisted by the raw materials and parts identification function
determined that a piece of raw material is one of the good materials
recorded in the good raw materials and parts catalog. This piece is
going to be admitted inside the SSR and transported to a certain place
for processing or possibly for temporary storage followed by
processing. It appears that a good practice to be employed by the SSR
would be to tag or label this piece somehow so that its nature, once
determined at the input gateway, is well known and available for
subsequent processing stations or storage stations in the SSR. We will
now assume that any raw material or part accepted inside the SSR, once
its nature is identified, is immediately tagged/labeled using a system
similar to the bar codes or RFIDs (radio-frequency identification)
where the code used is one of the codes in the catalog of raw materials
and parts. This systematic labeling and tagging of all accepted
materials and parts will be considered another responsibility of the
\textbf{materials and parts identification function. }More than that,
we will assume that in the process of SSR growth and during the cloning
phase, the SSR will fabricate new parts, components and assemblies
using either raw materials and raw parts or previously fabricated
parts, components and assemblies. The point is that the SSR and the
\textbf{materials and parts identification function} in
particular should be responsible to tag/label not
only raw materials and parts accepted inside SSR but also all
fabricated parts, components and assemblies. This allows us to affirm
that all elements present inside the SSR and all SSR parts should bear
a permanent identification/tag.

\subsection[The catalog of fabricated parts, components and
assemblies]{The catalog of fabricated parts, components and assemblies}

\hypertarget{RefHeading3060306210128}{}This raises also another
important aspect for the SSR design. The SSR must possess not only an
exhaustive catalog of all raw materials and raw parts, but also a
\textbf{catalog of all fabricated parts, components and assemblies}
with a unique identifier for each \textbf{type} of such element. 

\subsection[The bill of materials function]{The bill of materials
function}

\hypertarget{RefHeading3062306210128}{}The automated fabrication
processes will need additional informational support. We will name this
capability the \textbf{bill of materials function}. It is supported by
an exhaustive catalog that has entries that specify for each and every
fabricated part, component and assembly what is the list of materials
and parts that is needed to fabricate that part. For each item in this
list the count or quantity of that part or material must be also
specified. For example, for a “power supply enclosure” part with
identifier:  “ID-50712294”, its entry in the bill of materials catalog
may look like the table below:

%% FIXME - fix table
{\bfseries
Table  Example Entry in the Bill of Material Catalog for the
``Power Supply Enclosure'' part}

The flags: Q=quantity, C=count, D=dimensions specify what properties are
specified for each part name


\bigskip

\begin{flushleft}
\tablehead{}
\begin{supertabular}{|m{1.3094599in}m{1.3573599in}m{0.9837598in}m{0.6705598in}m{0.82335985in}m{1.0323598in}|}
\hline
\textbf{Part name} &
\textbf{Part Identifier} &
\textbf{Flags} &
\textbf{Count} &
\textbf{Quantity} &
\textbf{Dimensions(“)}\\
Sheet metal  1/16 &
~
 &
~
 &
{}- &
~
 &
~
\\
Screws 1/8*2 &
~
 &
~
 &
8 &
~
 &
~
\\
Washers  ¼*2 &
~
 &
~
 &
8 &
~
 &
~
\\\hline
\end{supertabular}
\end{flushleft}

The \textbf{bill of materials} is an \textbf{informational function} of
the SSR. Like all SSR informational functions it has two components:

\begin{itemize}
\item A specific catalog (or database) – in this case the bill of
material catalog
\item A set of information access sub-functions to search, read, write,
update or delete specific entries in the associated catalog. This set
of sub-functions can be accessed by all other SSR functions 
\end{itemize}

\subsection[The fabrication material extraction function ]{The
fabrication material extraction function }

\hypertarget{RefHeading3064306210128}{}Some of the raw materials
admitted inside the SSR cannot be directly used by the SSR fabrication
processes. They need to be transformed into “fabrication materials”
through one or more specific processes. As an example – which may not
need to have application inside the artificial SSR - is the fabrication
of steel (as a “fabrication material”) from iron ore and coal (both of
these are in this case “raw materials”). The ability of the SSR to
extract fabrication materials from sets of raw materials and parts is
the \textbf{fabrication material extraction function}.  Fabrication
materials are registered in the \textbf{catalog of fabrication
materials }while every procedure and technological process that is used
to extract fabrication materials from raw materials and parts is
documented in a \textbf{fabrication material extraction process
catalog}.

\subsection[The supply chain function]{The supply chain function}

\hypertarget{RefHeading3066306210128}{}Let’s now assume that during
cloning phase the SSR must fabricate a component of type A.  The bill
of materials entry for component of type A specifies that its
“fabrication recipe” requires 2 raw parts of type X and 4 raw parts of
type Y.  The SSR will need a capability to coordinate the input
gateways to admit ahead of time the required counts of parts X and Y,
and create some stock in a SSR “stock room” so that the fabrication of
components A can go smoothly and depend a little less on what raw parts
are available at any given moment at any one of the enclosure input
gateways. We are going to name this SSR capability the \textbf{supply
chain function}. This function is thus responsible to interact with
fabrication processes inside SSR, to gather information ahead of
fabrication time, on what raw materials, raw parts or even fabricated
parts are needed and command the input flow control function, the
material and parts identification function to admit, supply and stock
those elements inside the SSR.

\subsection[The energy generation and distribution function]{The energy
generation and distribution function}

\hypertarget{RefHeading3068306210128}{}All the machines inside SSR need
a source of energy to perform their work. Thus we must have inside the
SSR a capability to produce energy from raw materials and raw parts
that are appropriate for energy generation. This SSR capability will be
named the \textbf{energy generation and distribution }function since it
has the responsibility not only to generate energy but also to manage
and distribute it to all energy consumers inside the SSR. It is
important to note that the catalog of raw materials and raw parts as
well as the catalog of fabricated parts may contain entries that will
be marked as elements used for energy generation and/or distribution.
Also the \textbf{catalog of processes} will contain entries identifying
those material processes (and their detailed description) that are used
to generate energy from the energy-marked materials and parts. The
supply chain function is responsible to manage the timely supply of
materials and parts not only for fabrication but also for energy
generation and transport.

\subsection[The transport function]{The transport function}

\hypertarget{RefHeading3070306210128}{}The functioning SSR features
multiple “sites”, i.e. places where specific actions happen. These
sites will be distributed spatially inside the SSR, on its enclosure
and have well established positions relative to the elements that
maintain the three dimensional (and growing) structure of the SSR –
named scaffolding elements. For example, there are input gateway sites,
there are possibly some stock room sites, there are fabrication sites
and there are assembly sites where the elements of the growing clone
inside the SSR are being put together by some machinery. The SSR must
feature a capability to carry various elements between sites. This
capability is named the \textbf{transport function}. It may employ
specific means of transport, like using conduits, avenues, conveyors,
etc. that are adequate for the nature of elements being transported and
the placement of the sites inside SSR.  We will see later that an
important aspect of SSR activity is the transport of information
between “producers” and “consumers” inside the SSR. For this reason the
transport function is responsible to \textbf{transport also
information} between SSR sites (at least for the provision of the
physical, lower layers of the transport of information).

\subsection[The manipulation function]{The manipulation function}
\hypertarget{RefHeading3072306210128}{}Another important capability that
is needed inside the SSR is the \textbf{manipulation function}. This
function consists of the ability to handle, grab or manipulate raw
materials, raw parts, fabricated parts, fabricated components and
fabricated assemblies. For example this manipulation ability is needed
to take a raw material or raw part admitted at an input gateway and
place it on a conveyor that has as destination a stock room or a
fabrication site. There another “manipulator” will grab the material or
part and place it in a specified position in the stock room or place it
on a fabrication bench or machinery. Manipulation examples abound,
since no matter what elements are processed, transported, fabricated,
assembled or pushed out of an output gateway there is a need to
adequately handle those elements.

\subsection[The fabrication function]{The fabrication function}

\hypertarget{RefHeading3074306210128}{}Since the SSR must be able to
clone its core elements, its enclosure and its scaffolding elements,
there is an absolute need for the SSR to have a \textbf{fabrication
function}. This function is the ability to fabricate exact copies of
all parts, components and assemblies that exist inside a “mature” SSR
or on its enclosure. In other words the fabrication function must be
able to fabricate all machinery inside the SSR, including fabrication
machinery. Since all the elements that reside inside the SSR must be
copied (cloned) and various types of information and software elements
(as we will see later) also reside inside that “mature” SSR, results by
implication that the fabrication function may have also the ability to
accurately copy information and software (since the information or
software cannot be fabricated “per se” and assuming that the detailed
design of the artificial SSR includes software as the nature of certain
fabricated components of the SSR).

\subsection[The assemblage function]{The assemblage function}

\hypertarget{RefHeading3076306210128}{}Another capability that must
reside inside the SSR is the \textbf{assemblage function}. This is the
ability of the SSR to assemble or put together parts into
increasing-in-complexity components and assemblies. The assemblage
function is strongly related to the fabrication function. These two
functions can be seen as the two faces of the same coin. It makes sense
though to see them as distinct functions where the fabrication function
creates new parts from raw materials and raw parts through special
processes (example metal machining) while the assemblage function puts
together fabricated parts into more and more complex assemblies of
parts and components. The assemblage function may be needed for example
to erect and expand the scaffolding and the enclosure during the SSR
growth besides being used to create assemblies of smaller components
and parts.

\section{Additional SSR Functions}

\subsection[The recycling function]{The recycling function}

\hypertarget{RefHeading3080306210128}{}The SSR functions discussed so
far provide specific assistance for the “ingestion” of new raw
materials and parts into the SSR and the SSR growth during the phase of
cloning – based on continuous production of energy and on planned
fabrication of the elements of the “daughter” clone growing inside the
expanding SSR enclosure. As in any process that performs fabrication
and construction of new parts, there will be “residue” raw materials
and raw part fragments.  The SSR must be designed to carefully control
the growth of the inside and enclosure elements of the SSR. It cannot
grow without limits or in an uncontrolled manner. In order to achieve
this objective the SSR must:

\begin{itemize}
\item Re-introduce in the fabrication and growth cycles certain elements
of the “residue” raw materials, raw parts or raw part fragments that
can be reused.
\item Identify and specifically mark as refuse certain residue elements
that cannot be recycled and that are then pushed as refuse through the
output gateways.
\item Provide specific processes to “clean” and “tidy up” the SSR
interior fabrication and transport spaces, such that fabrication and
assembly of the cloned parts is not stopped or affected and the SSR
maintains “proper structure” during the cloning and division phases.
\end{itemize}

Most of the above responsibilities pertain to the \textbf{recycling
function.}

\subsection[The output flow control function]{The output flow control
function}

\hypertarget{RefHeading3082306210128}{}The recycling function controls
the \textbf{output flow control function }which is the ability to
control the enclosur\textbf{e output gateways }for forcing out the SSR
the raw materials, raw parts and raw part fragments \textbf{marked as
refuse} by the recycling function.

\subsection[The construction plan function]{The construction plan
function}

\hypertarget{RefHeading3084306210128}{}We already saw that the bill of
materials function provides for each fabricated part, component or
assembly of the SSR a list of raw materials, parts and sub-components
that are needed for fabrication of that element. Thus an entry in the
bill of materials information catalog is similar in concept with the
“list of ingredients” for cooking a meal. Besides the list of
ingredients the recipe for a meal contains an ordered “list of steps”
needed to prepare the meal. In a similar manner, the SSR must store
descriptive information for all construction steps needed to fabricate
each SSR element. This capability of the SSR to store and make
accessible detailed information about the set of fabrication steps and
processes needed for the fabrication of each SSR element (part,
component, assembly of components, up to including the SSR itself) is
called the \textbf{construction plan function}.

\subsection[The construction plan information catalog]{The construction
plan information catalog}

\hypertarget{RefHeading3086306210128}{}The \textbf{construction plan
function} has an associated \textbf{construction plan information
catalog}. This catalog has an entry for each fabricated element of the
SSR. A fabricated element can be a simple fabricated part (fabricated
from a single good material). A fabricated element can also be a
component which, in this context, refers to an element fabricated
through assemblage of two or more fabricated parts. A fabricated
assembly is even more complex: it is fabricated from multiple simple
parts and one or more components and possibly one or more (sub)
assemblies. The mature SSR (before starting the cloning process) is a
particular case of an assembly. Another example of an assembly (most
complex in this case) is the SSR grown to contain both the mother core
elements and the daughter elements just before division starts. 

Each entry in the construction plan catalog contains the following
information items:

\begin{itemize}
\item A reference to the entry in the bill of materials catalog for the
same element (to access the “ingredients” needed for the element
fabrication)
\item An ordered sequence of fabrication and assembly steps.
\end{itemize}

For each fabrication/assembly step the following information may be
provided:

\begin{itemize}
\item The spatial assembly or placement instruction of parts/materials
involved in the step ( how to spatially place a part/component P
relative to the assembly under construction on the work bench, before a
fabrication/technology step)
\item The type and detailed description of each fabrication/assembly
process executed during the current fabrication step.
\item Technological parameters of fabrication/assembly process (like
ambient temperature, length of process, etc.)
\item List of “residue” parts/materials resulted from the process/step.
\item Part manipulation steps (with x, y, z starting point, x, y, z
ending point, any part rotation, with axis specification or translation
movement).
\item Any fabrication step verification procedure – to determine if the
step completed successfully – within the accepted parameters, or the
fabrication step was a failure.
\item Recovery action list in case a fabrication step fails with a
specific verification error.
\end{itemize}

In short, the rationale for the nature, structure and the extent of
information items stored for each step of a \textbf{fabrication plan
entry} is to provide support for full automation of that element
fabrication. And, as suggested above \textbf{fabrication steps} can be
of a very large variety. The nature of the fabrication process as well
as the nature of fabrication steps depends on the material basis to be
selected for the design and implementation of the artificial SSR. The
alternative material bases that can be realistically considered for
creating an artificial SSR are discussed in the Part II of this study.
To help understand the nature of a fabrication step here are several
examples:

\begin{itemize}
\item A part or component being placed in a particular position relative
to the semi-assembled element – in preparation for the next joining
(welding, screwing) operation or process (diffusion, thermal treatment,
metal machining)
\item An assemblage step. 
\item A metal machining kind of fabrication step
\item A chemical process step
\item A thermal process step
\item An electrolytic process step
\item A nano-technology assemblage step
\item An information file copy step
\item A fabricated component (assembly) test step
\end{itemize}

The next issue that requires a solution is to devise a way to track the
progress of the cloning and division steps. This is provided by the
\textbf{construction status function}. This function uses an
information catalog similar with the construction plan catalog named
\textbf{construction status catalog}. It is similar in the sense that
it has the same list of element entries as the construction plan
catalog describing the same hierarchical composition of each element
(part, component, assembly) in sub-elements.  Each entry in this
catalog has construction status information that reflects the current
construction status of that entry and can have values like:

\begin{itemize}
\item Not-started
\item Started
\item Completed
\end{itemize}

In a similar manner each fabrication step of an entry has a current
construction status field that is also used to mark and keep track of
the fabrication/construction status for that element at the fabrication
step level.

Before discussing the last set of SSR functions – this last set contains
higher level functions – we will focus briefly on the reality that the
SSR must have a variable geometry and the problems that need solutions
due to this variable geometry.

\subsection[The SSR variable geometry]{The SSR variable geometry}

\hypertarget{RefHeading3088306210128}{}The SSR has variable geometry
because:

\begin{itemize}
\item The mature SSR must grow in volume and enclosure surface during
the cloning phase to make space in its interior for the growing clone. 
\item The geometry changes even more radically when the division phase
starts and culminates with the complete division of the original SSR in
two: the “mother” SSR and the “daughter” SSR.
\end{itemize}

The SSR variable geometry has the following relevant aspects that must
be carefully considered by the SSR design and by the hypothetical
implementation of the artificial SSR:

\begin{itemize}
\item The SSR enclosure must have such a structure and composition
(texture) that allows:

\begin{itemize}
\item Growth in surface as the SSR interior grows in volume during the
cloning phase. This may require a design that allows selective
insertion of new enclosure parts/elements in between existing
parts/elements and some kind of “linkage/connection” of each new
element with its neighboring elements.
\item Insertion in the enclosure of new input gateways and output
gateways while the enclosure grows.
\item Division of the enclosure into two separate enclosures (one for
the mother SSR and one for the daughter SSR)  with each enclosure
carrying its separate sets of input and output gateways, its
scaffolding and interior elements.
\end{itemize}
\item Special design provisions must be made for the interior SSR
scaffolding. The SSR scaffolding is made of any kind of structural
elements (pylons, walls, supports, connectors) that are needed to
maintain the three dimensional structure and integrity of the enclosure
and of the SSR interior space(s). The scaffolding design and its
elements need to be conceived such that:

\begin{itemize}
\item The scaffolding elements my change size (grow or shrink) as the
interior of the SSR grows (during cloning) or shrinks (during division)
\item The spacing between scaffolding elements and their connectors may
also grow or shrink (during cloning respective division phases).
\end{itemize}
\item The design of the SSR must also make provisions for the growth,
variable geometry and dynamic restructuring and re-linking of any SSR
transport, conduits, paths or communication lines during the cloning
phase and the division phase.
\end{itemize}

The variable geometry means that the SSR design must make specific,
detailed provisions of all spatial evolution, of all geometrical
definition points or trajectories (in the x, y, z axes) of all variable
elements of the SSR enclosure, scaffolding and interior. These spatial
trajectories need to be harmoniously and coherently coordinated with
all fabrication and assemblage steps of the cloning and division
phases.

\subsection[The Communication and Notification Function]{The
Communication and Notification Function}

\hypertarget{RefHeading3090306210128}{}This function is responsible to
provide and manage the information communication and notification
machinery and mechanisms between the command centers and execution
centers of the SSR. For example the fabrication control function-
acting as a command center – may send a command as a specifically
encoded information “package”  to the fabrication and assemblage
functions to build a particular component of the “daughter” clone. In
this circumstance the fabrication and the assemblage functions operate
as execution centers for the command. When the fabrication of the
requested component is completed by the fabrication function it will
send a specifically formatted notification information package back to
the fabrication control function with the meaning that the specific
command for the fabrication of the specific element was successfully
completed. Within the same example scenario the fabrication function
will send in its turn – this time playing a command role itself – a
command to the supply-chain function (the executor entity) to trigger
the transport of the needed fabrication “ingredients” for the clone
element to the fabrication site. The \textbf{communication and
notification function} need to be deployed ubiquitously throughout the
SSR to allow communications/notifications between various functions and
machinery operating all over the SSR.

\subsection[The higher level SSR functions]{The higher level SSR
functions}

\hypertarget{RefHeading3092306210128}{}The SSR functions that are
described in subsequent sections are the highest level functions of the
SSR. They accomplish their goals by coordinating and choreographing the
lower level functions described in the previous sections.

\subsection[The scaffolding growth function]{The scaffolding growth
function}

\hypertarget{RefHeading3094306210128}{}The \textbf{scaffolding growth
function} is responsible to manage the construction, growth and
position change of the SSR scaffolding elements during the cloning and
the division phases of the SSR replication. This function needs to
manage the variable geometry of the scaffolding elements in
synchronization and coordination with the other spatial changes of the
SSR on its enclosure and in its interior.

\subsection[The enclosure growth function]{The enclosure growth
function}

\hypertarget{RefHeading3096306210128}{}The \textbf{enclosure growth
function} is responsible to manage the construction, growth and shape
change of the SSR enclosure as well as the coordinated addition of
input and output gateways on the enclosure during the cloning and
division phases of the SSR replication. As already alluded to, this
function needs to manage the variable geometry of the enclosure, the
dynamic shifting of the gateways on the enclosure surface, and the
radical shape changes of the enclosure around the division of the SSR
into the mother and daughter descendants.

\subsection[The fabrication control function]{The fabrication control
function}

\hypertarget{RefHeading3098306210128}{}The \textbf{fabrication control
function} is responsible for the construction, assemblage and variable
geometry management of all interior elements – in particular those
related to the cloning part of the SSR during the cloning and the
division phases of the SSR replication.  Like the two preceding growth
functions this function coordinates activities of the fabrication
function, assemblage function, construction plan function, recycling
function and other lower level functions. 

\subsection[The cloning control function]{The cloning control function}

\hypertarget{RefHeading3100306210128}{}The \textbf{cloning control
function} is responsible to coordinate the whole cloning phase of the
growing SSR. It basically does that by coordinating the cloning and the
growth of all involved SSR compartments through a tight control,
synchronization and coordination of the scaffolding growth, enclosure
growth and fabrication control functions. This function is responsible
in particular to start the cloning process, to monitor its development
and to determine accurately when the cloning process is complete. One
particular responsibility of the cloning control function is the
\textbf{cloning of the information} stored into the mother SSR. This
information cloning is performed at the end of the cloning phase when
all internal machinery, internal scaffolding and enclosure elements of
the mature SSR were completely cloned and constructed as part of the
“nascent” daughter SSR. The information is cloned by systematically,
accurately and completely copying all information catalogs from
resident machinery of the mature SSR into the corresponding new
machinery of the clone part.

\subsection[The division control function]{The division control
function}

\hypertarget{RefHeading3102306210128}{}The \textbf{division control
function} has full control of the division phase of the SSR replication
process. It manages the SSR division through specific commands sent to
the scaffolding growth, enclosure growth and fabrication control
functions.  In particular this function is responsible to start the
division process, to “choreograph” its development on all SSR
compartments (enclosure, scaffolding and core) and to accurately
determine when division process is complete. One particular
responsibility of the division control function is to “start the
engines” of the nascent “daughter SSR”. Just before the moment the
division is complete the division control function must send a command
to the “daughter SSR” to start its own machinery and control functions.
In this way when the division completes, the separated daughter SSR
becomes a “mature”, fully functioning autonomous SSR ready to start its
own replication process. The separated “mother SSR” can start in its
turn a new replication cycle.

\subsection[The replication control function]{The replication control
function}

\hypertarget{RefHeading3104306210128}{}The \textbf{replication control
function} is the highest level SSR function. It is responsible to
accomplish the SSR full replication cycle by coordinating and
choreographing its two phases: cloning and division through
corresponding control and coordination of the \textbf{cloning control
function} and the \textbf{division control function}.

Metaphorically speaking the replication control function implements the
two significant \textbf{SSR designer commandments:}

\begin{itemize}
\item \textbf{Grow} and
\item \textbf{Multiply}
\end{itemize}

\subsection[The SSR Function Dependencies Diagram]{The SSR Function
Dependencies Diagram}

\hypertarget{RefHeading3106306210128}{}An overall diagram illustrating
the identified set of functions present in the SSR and some of the
dependencies between these functions is presented in Figure \ref{fig:functions_and_dependencies}.

The main dependencies and interactions between the identified SSR
functions are depicted in this figure. Not all such function
dependencies and interactions are represented in the picture but only
the main ones. 

The Communication and Notification Function is represented separately
since it relates to and is used by almost any other SSR function
because the need for information communication between functions as
well as notification (another form of information exchange) is
ubiquitous through SSR functions.

As already mentioned the relationships and dependencies between the
functions are more complex and richer than depicted in the diagram. For
example the Transport function depends on Energy Generation and
Transport function although this dependency is not depicted in the
diagram.

\begin{figure}
	%% FIXME - need actual figure
	\label{fig:functions_and_dependencies}
	\caption{SSR Functions and their Dependencies}
\end{figure}

\section{The Type and Nature of SSR Components}

At this point it makes sense to only consider the general conceptual
categories to which various components of the artificial SSR pertains.
We grouped these components according to a few categories.

Part II of this paper will go into more concrete details about actual
physical components that may make up the artificial SSR.

SSR Component Categories %% FIXME - make a headign or some sort of introduction, and give some description of each list

\begin{itemize}
\item Information storage and access
\item Information processing
\item Information coding and decoding
\item Information transport, communication and notification
\end{itemize}

\bigskip

\begin{itemize}
\item Material identification
\item Material transport and manipulation
\item Material processing
\item Mechanical and chemical transformation of materials
\item Material fabrication and assemblage
\end{itemize}

\bigskip

\begin{itemize}
\item Energy generation
\item Energy transport
\item Energy conversion
\item Energy distribution and management
\end{itemize}

\bigskip

\begin{itemize}
\item Environment sensing
\item Environment (local) control
\end{itemize}

\begin{itemize}
\item SSR construction plan representation
\item SSR dynamic 3D evolution representation
\item SSR construction status representation
\item SSR parts inventory representation
\end{itemize}

\section{The SSR and Its Information Catalogs}

We already encountered various types of information catalogs (databases
or repositories) that together make up the information base of the SSR.
This section contains a list of all types of information catalogs
identified so far. It is quite probable that as a more in depth
analysis is performed on this topic the need of additional types of
information catalogs will be discovered.

All information describing in detail each element of the SSR during its
full “life-cycle”, all relationships between these elements captured in
construction plans (body plans) and all fabrication and assemblage
procedures and processes need to be thoroughly, systematically and
coherently designed and captured into the SSR information catalogs.

We are making certain simplifying assumptions, just to make the
presentation of ideas easier to understand. We are not discussing and
presenting a very significant aspect of the nature of information that
the SSR design must capture. A real information repository contains
conceptual lists of tables (catalogs) of items of the same nature but
also contain lists of the various relationships that exist between the
items in different tables (catalogs).  For example, there are certain
relationships between the items in the catalog of raw materials and the
items (entries) in the catalog of bill of materials. There are
different sets of relationships between the entries in the catalog of
the construction plans and the entries in the catalog of the bill of
materials and the entries in the catalog of processes.

Here is the list of information catalogs identified (or alluded to) so
far:

\begin{itemize}
\item The catalog of raw materials
\item The catalog of raw parts
\item The catalog of fabrication materials
\item The catalog of raw materials identification procedures and
processes
\item The catalog of raw parts identification procedures and processes
\item The catalog of fabrication materials extraction procedures and
processes
\item The catalog of energy generation procedures and processes
\item The bill of materials catalog
\item The catalog of internal machinery
\item The catalog of construction plans
\item The catalog of fabrication procedures and processes
\item The catalog of assembly procedures and processes
\item The catalog of fabrication verification procedures
\item The catalog of fabrication errors handling procedures
\item The catalog of energy consumptions
\item The catalog of recycling elements and procedures
\item The catalog of construction status
\end{itemize}

%% FIXME - need a summary / conclusion
