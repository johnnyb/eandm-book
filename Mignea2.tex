\eandmchapter{The Simplest Self-Replicator, Part~2}{Developing Insights into the Design of the Simplest Self-Replicator and Its Complexity: Part~2---Evaluating the Complexity of a Concrete Implementation of an Artificial SSR}{Arminius Mignea}{The Lone Pine Software}

\begin{abstract}
This is the second in a three-part series investigating the internals 
of the simplest possible self replicator (SSR).  
It builds on the construction of a hypothetical self-replicator devised in 
Part~1, and considers various 
significant aspects about the
design and construction of an artificial, concrete SSR: the material
basis of its construction, the effects of the variable geometry of the
SSR during its growth through the cloning and division phases, and the
three closure rules that must be satisfied by the SSR---energy closure,
material closure, and information closure.

The highest technical
challenges that need to be faced by the design and construction of the
artificial SSR will be considered. The emerging complexity of the artificial
SSR is depicted using a metaphorical comparison of the SSR
with a city fully populated by automated machinery that
systematically constructs a new city that is identical
to the old city without external help but only using the
construction materials that enter through the city gateways. 
The current level of technology is evaluated to determine if it is sufficient for the successful completion of the design and construction of an artificial autonomous SSR project using either a nano-biochemical basis or or a macro-material basis.

\end{abstract}

Part~1 of this series analyzed the basic necessary design elements of the simplest self-replicator (SSR), 
including necessary components, functions, processes, and information.
Having established the minimum requirements for the design, this part will discuss the physical implementation
of the SSR.

\section{The Three Closure Requirements as the Basis of an Autonomous SSR}

\index{self-replication!closure requirements|(}
The SSR must be fully autonomous. This means that it can only obtain raw
materials and raw parts from its environment and benefit from (or
struggle because of) the environmental conditions specific to its
location.

Full autonomy specifically requires the SSR to exhibit the following characteristics \citep{freitasmerkle2004}:

\begin{enumerate}
\item The SSR must fabricate all its energy from input materials, and the
generated energy must be sufficient for the SSR to produce an exact
replica of itself. This condition is called the \mterm{energy closure}. 
\item The SSR must use only materials admitted through its input gateways
and these materials must be sufficient for the SSR to grow and generate
its daughter. This condition is called the \mterm{material closure}. 
\item The SSR must use only information that is initially present or stored in the
mature SSR and this information must be sufficient to produce an
exact replica of the SSR. This condition is called the
\mterm{information closure}.
\end{enumerate}
\index{self-replication!closure requirements|)}

\section{The Core Approach to Cloning}

\index{self-replication!mechanism|(}
This section will try to answer the following important question for
the design of the artificial SSR: What is the core mechanism that
the artificial SSR will use to accurately clone all of its elements?

Below are two possible answers, and it is likely that most any other
imagined answers would be similar or equivalent to one of the
two answers below:

\begin{enumerate}
\item Design and use a \mterm{universal physical copy machine} (similar to a key copy
machine but much more sophisticated) that analyzes each part or assembly and produces
a copy, with the goal of simplifying the design and
avoiding having to maintain such detailed catalogs of information.
\item Use an exhaustive descriptive, operational and constructional SSR
information database directing an integrated set of specialized, software
and computer-controlled automatons.  In other words, have sufficient
data stored about the makeup of the SSR itself to generate a new copy
from that data.
\end{enumerate}

\subsection[Why the “universal physical copy machine” approach is not adequate]{Why
the “universal physical copy machine” approach is not adequate}

This approach assumes that the
SSR contains a sophisticated machine that can examine and accurately
copy all other pieces and machinery comprising the mature SSR. This
implies that this universal physical copy machine can even copy itself or, more
realistically, make a copy of a copy of itself. In other words, the SSR contains two
universal physical copy machines.  One, Machine A, is performing the actual copying of
all SSR machinery and the other universal physical copy machine,
Machine B. So the second copy machine, Machine B, is only used as a
model for the first Machine A. The goal for this solution is
to use these copy machines A and B to alleviate the need to store
as much information about the SSR itself, to have as much software, and to contain as much
computer-controlled machinery as in the second approach.

This first solution leads to the following conclusions:

\begin{enumerate}
\item It will require another machinery
M\textsuperscript{disassembler}, to disassemble Machine B in all its
constituent parts: b\textsubscript{1}, b\textsubscript{2},
b\textsubscript{3}, …, b\textsubscript{N }so that Machine A can copy
each constituent part.
\item Then Machine A will copy all parts b\textsubscript{1},
b\textsubscript{2}, b\textsubscript{3}, …, b\textsubscript{N} twice (to
create all pieces needed for a clone of Machine A: Copy\textsubscript{A}
and a clone of Machine B: Copy\textsubscript{B}) ca\textsubscript{1},
ca\textsubscript{2}, ca\textsubscript{3}, …, ca\textsubscript{N }(for
Copy\textsubscript{A} machine) and cb\textsubscript{1},
cb\textsubscript{2}, cb\textsubscript{3}, …, cb\textsubscript{N} (for
Copy\textsubscript{B} machine).
\item There will be a need for another machinery
M\textsuperscript{assembler} that will know how to take all the parts
ca\textsubscript{1}, ca\textsubscript{2}, ca\textsubscript{3}, ….,
ca\textsubscript{N} and assemble them together into the
Copy\textsubscript{A} machine and parts cb\textsubscript{1},
cb\textsubscript{2}, cb\textsubscript{3}, …, cb\textsubscript{N} and
assemble them together into the Copy\textsubscript{B} machine. 
\end{enumerate}

In order
to properly construct M\textsuperscript{assembler}, a vast amount
of well-structured information of this nature must be first programmed:

\begin{itemize}
\item A catalog of all parts b\textsubscript{1}, b\textsubscript{2},
b\textsubscript{3}, …, b\textsubscript{N} and for each such part, a
unique identifier and possibly physical and geometrical characteristics
(dimensions) of the part
\item A store room location (x, y, z)  from where the
M\textsuperscript{assembler} machine will pick each one of the parts
ca\textsubscript{1}, ca\textsubscript{2}, ca\textsubscript{3}, …,
ca\textsubscript{N} during the assembly steps to construct the
Copy\textsubscript{A} machine
\item A \mterm{catalog of assembly instructions} that contains some geometrical x,
y, z instructions and the type of assemblage step (e.g., screwing,
inserting, welding, etc.).  For example, these assembly instructions may describe processes such as the following:

\begin{itemize}
\item how to put together part ca\textsubscript{2} to the assembly made
of parts: (ca\textsubscript{1});
\item how to add part ca\textsubscript{3} to the assembly made of parts
(ca\textsubscript{1}, ca\textsubscript{2});
\item how to add part ca\textsubscript{4} to assembly made of parts
(ca\textsubscript{1}, ca\textsubscript{2}, ca\textsubscript{3}).
\end{itemize}

Such descriptions would continue for all parts that need to be included in the assembly.

\item A manipulator machine (robot that can follow computerized
instructions) to be controlled by the M\textsuperscript{assembler}
machine in assembling the Copy\textsubscript{A} machine
\end{itemize}

Although the original impetus for Solution 1 is to avoid mountains of
information and armies of automatons and machinery, 
this solution requires those ingredients. 
There is no magic copy machine that can do its work without structured
collections of information and many helper automatons (machinery) that,
in turn, must be information-, software-, and computer-controlled.  
The conclusion is that there is no ``magic universal physical copy machine'' solution
that is significantly distinguishable from the Solution 2. 

Another problem with Solution 1 is that certain components of Machine B may not
be fabricated by plain (mechanical) assemblage of parts but rather by
using more demanding assemblage processes, such as welding or 
electro-chemical processes.  These processes have no precise means of disassembly,
which would prevent the universal physical copy machine from being able to reproduce
them.

\subsection[Exhaustive information, integrated systems driving information{}-controlled automatons]{Exhaustive information, integrated systems driving information-controlled automatons}

Since the universal physical copy machine approach does not work, the only
other replication method designs
the SSR as a collection of
integrated sub-systems, controlling a large variety of automatons using
a significant collection of integrated information catalogs
(databases).

A number of information catalogs have already been mentioned while identifying
specific SSR functions. Any informational SSR function has both an
associated catalog and also a set of access sub-functions that provide
a set of access operations to the information catalog that can be used
by other SSR functions to execute specific action sequences.
\index{self-replication!mechanism|)}

\section{The Material Basis of the SSR}

\index{self-replication!design|(}
When approaching the task of the design and implementation of an
artificial SSR, a capital question surfaces rather quickly: What should be
the material basis for the artificial SSR? There are two distinct
possibilities: Either use a biological basis for the SSR on a micro/nano 
scale or use more common macro scale materials and technology.

\subsection{Using a Biological Basis for the SSR}

\index{self-replication!biological self-replication|(}
Using a biological basis for the SSR means that the SSR must be constructed using organic materials.  These materials would be the
same or similar to those used by the cells, tissues, and organs of the
living world.  There are several advantages to this approach.  First, the
proof in the feasibility of this approach is the presence of the varied organisms
and microorganisms in nature.  A key question is whether this approach is accessible given
current engineering technologies. Second, even though energy generation is one of the biggest challenges for SSR replication, there are known levels of energy consumption based on biological systems. Lastly, biological systems tend to be in aqueous medium, which may facilitate solutions for the variable geometry problem.

The biological approach is not without its setbacks. First, since this approach takes place at extremely small scales, it taxes the limits of current investigative tools and observational methods. The most advanced microbiology manipulation and fabrication tools/approaches are still rather primitive and very limited when considering the tasks that need to be accomplished: fabrication, assemblage, manipulation at nano scales, computing machinery fabrication, software execution, information storage, and communication. Secondly, there are many aspects of cell biology that are still beyond current understandings of the cell’s function.  Examples of the challenges that we cannot solve with current technology include

\begin{itemize}
\item building computers on a biological material basis/scale (or
understanding how the cell proteins and other organic cell elements can
be used for computation in a general way)
\item building bio-chemical manufacturing machines at a biological
scale
\item building information storage using biochemical materials
\item having software running on biological type computers
\item communicating information on a biological material basis and at a biological scale
\end{itemize}

Clearly, the conclusion is that, with the current level of
technology, it is impossible to create a design and an implementation
plan for an artificial SSR using a biochemical and a biological material
basis at a molecular scale. Other alternatives must be considered for a better
chance of successfully building an artificial SSR.
\index{self-replication!biological self-replication|)}

\subsection{Using a Macro Scale Basis for the SSR}

\index{self-replication!macro scale self-replication|(}
The other alternative for consideration is the macro scale approach, using materials
and technology that are in common use for product fabrication.
This approach must consider the minimum
dimensional scales for which there are available manufacturing
technologies for most of the parts, components, and machinery that make
up the SSR.  

The materials used to construct the SSR enclosure, SSR scaffolding, and
SSR interior should be common engineering materials used by current
fabrication technologies: metals, alloys, plastics, ceramics, silicon
or other special materials.  The scale of these artifacts to be fabricated 
as elements of the artificial SSR must be selected with care as there
are two opposing considerations which must be balanced and compromised.

The first consideration is that the 
smallest possible scale should be used in the design and
implementation of the artificial SSR parts in order to
minimize the energy consumed by the SSR during a replication cycle and minimize the size, volume and mass of the artificial SSR in order to minimize the number of materials ingested into the artificial SSR and used for fabrication of the clone within the SSR.

However, this must be balanced with the limitations of 
known engineering technologies and machinery to fabricate,
manipulate and assemble all the parts of the SSR machinery. This
means, for an illustrative example, that if the minimum size of 
semiconductor fabrication equipment that is being manufactured today is
0.5 meters, then the designed size of the mature artificial SSR cannot
be smaller than 1 meter. Therefore, a more realistic artificial
SSR design would have dimensions in the range of at least 10--100
meters.
\index{self-replication!macro scale self-replication|)}

\section{The Type and Nature of SSR Components}

\index{self-replication!design!components|(}
The conclusion in Part~1 was that the SSR must be designed and implemented as
a collection of integrated, computer-controlled and software-controlled
automatons. 
The artificial SSR must,
by necessity, contain these types of elements:

\begin{itemize}
\item Computing machinery, which implies that the following type of
elements must be present inside the artificial SSR:

\begin{itemize}
\item Printed circuit boards (PCB)
\item Microprocessors
\item Highly integrated circuits (Application Specific Integrated
Circuits = ASICs)---specialized, high density integrated circuits for
specific computing/application tasks, such as networking, numerical processing,
image processing, etc.
\item Semiconductor memories (solid state memories)
\item Magnetic memory (hard drives)
\item Electric power supplies
\item Computer connectors and wiring
\end{itemize}
\item Networking Communication Devices:

\begin{itemize}
\item Routers (wired/wireless)
\item Switches
\item Modems
\end{itemize}
\item Software
\item Robots
\item Energy generation and distribution machinery

\begin{itemize}
\item Generators
\item Transformers
\item Converters
\item Wiring
\end{itemize}
\item Batteries
\item Fabrication machinery
\item Metal machining machinery
\end{itemize}
\index{self-replication!design!components|)}

\section{Derived Design Requirements}

This section is a list of design and implementation
requirements for the artificial SSR that emerged from the previous
analysis and from the inferences presented so far. These requirements
were only implied during the discussion so far but are now made
explicit and are described in some detail.

\subsection[Each SSR Machine is Power{}-Driven]{Each SSR Machine is Power-Driven}

\index{self-replication!energy|(}
If each SSR machine is power-driven, then certain significant consequences arise in designing an artificial SSR:

\begin{itemize}
\item The SSR must have a power distribution network (e.g., an electrical
distribution network) that must reach each of the SSR's machinery. The
design of the layout and geometry of the power network must consider
the variable geometry of the SSR enclosure, scaffolding, and 
interior space and structure. Particular consideration must be made for the zones affected by
growth and shape changes.
\item Each SSR machine must be designed to use and consume power
(electricity) at a level adequate for its nature and the actions it
performs. 
\item The machinery is computer driven, which means that either
there is a parallel SSR power network for an energy level (e.g., voltage)
adequate for computing devices, or each machine must have some adequate
power converters (e.g., electrical power supplies or batteries).
\item The SSR machines that provide mechanical work or movement must be
provided with motors (rotational and/or linear) adequate for their nature.
\item Mobile machinery (e.g., transporters, moving robots) must be designed
such that their mobility is not constrained while they
are connected to the SSR power network(s).  Designing all mobile
machinery with rechargeable batteries may solve or significantly
simplify the connectivity constraints but will require additional
provisions for battery fabrication processes and fabrication and
provision of battery charging stations.
\item The design of each machine must provide specification for the average
power consumption on all power networks (normal power level and
computer power level) to which the machine is connected.
\end{itemize}
\index{self-replication!energy|)}

\subsection[Each SSR Machine is Computer{}-Driven and
Software{}-Driven]{Each SSR Machine is Computer-Driven and
Software-Driven}

\index{self-replication!process control|(}
If each SSR machine is computer-driven and software-driven, then there are several consequences to consider in designing an artificial SSR:

\begin{itemize}
\item Most SSR machines must host at least one internal computer with
the possible exception of some simpler (e.g., electro-mechanical) machines
that can be remotely controlled.
\item The SSR must have highly technical machinery and processes 
to fabricate computers and their respective parts.
\item Each SSR machine that hosts computing devices must be networked---by
wire or wirelessly---to other machines and control centers
within the SSR.
\item The SSR must have adequate machinery to not only
fabricate computers but to install them into other SSR machines,
plug them into the other machine's power network, and connect them to
the SSR's communication network.
\item The SSR must have machinery that is able to download and copy
software into any computer installed into an SSR machine, to start
(boot) that software on the machine, and to monitor its availability
and behavior.
\item The SSR must have the capability to test each piece of its machinery, to
detect malfunctions in computer and software installations as
well as in the machine hardware, to detect malfunctions in the computer
and software execution and to have adequate procedures to diagnose and
repair the identified problems.  Diagnosis and repair may be based on the availability of
fabricated spare parts.
\item The software that drives each particular piece of machinery must be
designed and written with a full understanding of the physical and
cinematic capabilities and constraints of that piece of machinery.
It must take into account all possible uses of the machine and all
its components' behaviors and interactions with external objects and
events as well as be able to handle them correctly.
\item The software that drives fabrication and assemblage machinery and
materials and fabrication processes needs to be based on a thorough
design of the machines that will be built, their cinematic capabilities, and
their specified power and energy consumption.
\item The software written for various SSR functions must carefully and
accurately coordinate and synchronize the activities of
multiple SSR machines (e.g., fabrication machines, material process machines,
manipulation and transport \mbox{robots/arms}, assemblage and construction
machines) by providing a continuous monitoring of the 3D spaces occupied
by each machine and its mobile parts to avoid collisions and to ensure
cooperative progress with both lower level and higher level tasks of
the growing SSR.
\end{itemize}

\subsection[Each Piece of SSR Machinery Is Capable of Information Communication]{Each Piece of SSR Machinery Is Capable of Information Communication}

If each piece of the SSR's machinery must have the ability to communicate information, then several features must be included in its design:

\begin{itemize}
\item The artificial SSR is a collection of automated machines and
robots. Their cooperation and coordination for achieving tasks from the
simplest (e.g., fabricating a part, or manipulating a part in a sequence of
steps for an assemblage operation) to the most complex ones (e.g., the
fabrication and assemblage of computing hardware and software
installation for a new piece of fabrication machinery) requires extensive,
continuous, multipoint, and multi-level communication of
information between machines, control functions, and software
components.
\item The SSR must have a comprehensive physical layer communication
network for information transport (wire-based and/or wireless)
with access points located on each, if not most, SSR machines/robots and
sometimes in between the subsystems of the same SSR machine.
\item The SSR might need to have adequate networking devices (e.g.,
routers, switches, modems, and codecs) to implement needed communication
patterns and topologies.
\item The SSR machine and software components engaged in communication
will need adequate networking/communication protocols with appropriate
characteristics for carrying the needed communication bandwidth, handling 
errors and retransmissions, reliable routing, and end point addressing.
\item The SSR should have the ability to deploy software on newly
constructed machines and network nodes, and to bring up the
network and verify it as part of starting up the daughter SSR system
(including its underlying communication network) as a preparatory step
in the SSR division phase.
\end{itemize}
\index{self-replication!process control|)}

\section[The Most Significant Challenges]{The Most Significant Challenges for the Design and Implementation of an Artificial SSR}

\index{self-replication!design!challenges|(}

\subsection[The energy generation and the energy closure challenge]{The
energy generation and the energy closure challenge}

\index{self-replication!closure requirements|(}
The energy generation and energy closure challenge presents multiple
hurdles which must be resolved.  
Primarily, the SSR design must select an adequate basis for energy generation.
This depends on what natural materials are available in
the SSR environment that can be used for energy generation.  
Some of the candidate material basis for energy
generation that might be considered for the design and implementation
of an artificial SSR might include biochemical or organic (e.g., vegetation used for energy generation), coal, oil/petroleum, natural gas, methanol, hydrogen, solar, wind, and/or nuclear.

The \memph{energy closure challenge} means that the amount of
energy generated by the SSR from the primary energy producing materials
extracted from the SSR environment must be sufficient to power
all machinery (e.g., fabrication, assemblage, construction, transport,
manipulators, robots, computers, and networking gear) that equip the
SSR.

Other challenges associated with energy closure include the following:

\begin{itemize}
\item The SSR must be designed with the ability to slow down or even
completely shut down during the periods when the input of energy
producing materials is reduced or null.
\item The SSR's ability to provide the means to store energy (with batteries,
accumulators or stocking energy-producing materials) may smooth out or
eliminate the need for transition to “hibernation” or shutdown states.
\item Designing SSR machines with local sources of energy (e.g., 
rechargeable batteries, accumulators, fuel reservoirs, or fuel cells)
may provide true, unconstrained mobility and
may significantly simplify the SSR design and implementation
difficulties related to keeping all mobile machines hooked to flexible
power wiring or network wiring.
\item Burning fuels or chemically generating energy leads to additional concerns in designing and implementing an artificial SSR. Particular concerns include preserving the SSR’s internal environmental parameters (e.g., temperature and humidity), avoiding hazardous materials, and providing storage transportation containers for liquid or gaseous materials.
\end{itemize}

\subsection[The material closure challenge]{The material closure challenge}

The material closure challenge
for designing and implementing the artificial SSR can be summarized as follows:
all fabrication materials that are needed for fabricating 
the parts of the SSR machinery must be available in the SSR environment or
must be extracted from raw materials available in the SSR
environment.

While this may not appear to be a daunting task, upon closer consideration,
the artificial SSR will need fabrication machinery for metal
machining, computers with semiconductor microprocessors and memories,
plastics, and ceramics.  There is potentially an extremely
long list of materials needed for SSR fabrication. For example, 
looking at a short
subset of materials needed for the macro scale artificial
SSR provides insight into the size of the challenge:

\begin{itemize}
\item Iron
\item Steel (of various varieties)
\item Copper
\item Aluminum
\item Metal alloys (of different varieties)
\item Silver
\item Gold
\item Ceramics
\item Plastics
\item Silicon
\item Polytetrafluoroethylene (Teflon) for Printed Circuit Boards
(PCBs)
\item Tin
\item Nickel
\item Germanium
\end{itemize}

Even given the partial list above, it appears that there is a very
small probability that the SSR's local environment 
will feature such a large diversity of immediately available
materials or components from which the materials in the list could somehow be
extracted. This makes the material closure requirement
appear \memph{unsolvable}, and thus any project to design and implement an
artificial macro-scale replicator may be condemned to failure. 
\index{self-replication!closure requirements|)}

\subsubsection[The match between the SSR design and the design of the
SSR environment]{The match between the SSR design and the design of the
SSR environment}

Another way to formulate the material closure challenge is that a
successful selection of materials used for energy generation and 
fabrication of internal parts, components, and machinery must be
based on a thorough knowledge of the environment in which the designed
SSR is projected to exist, including the nature of raw materials and parts in
such environment, and realistic material extraction paths
and processes. 
In other words the success requires a
perfect design of \emph{both} the SSR and its environment.

\subsection[The Fabrication Challenge]{The Fabrication Challenge}

\index{self-replication!fabrication|(}
The fabrication challenge is
the requirement that the artificial SSR must be able to
fabricate and assemble any type of parts, components and machines that
are part of the mature SSR, which implies that all fabrication and
assemblage machines should be able to fabricate exact copies of
themselves.

While the material closure challenge focuses on the difficulty of having
a wide spectrum of fabrication materials readily available, the
fabrication challenge raises several other concerns.

\begin{itemize}
\item Since the artificial SSR will have much machinery made with
metals (e.g., fabrication machinery, construction and assemblage machines,
robots, manipulator arms, networking gear,  wires, power supplies,
conduits, and scaffolding), the SSR must have a diversity of metal machining machinery.
\item The SSR must be able to fabricate the necessary machinery and enclosures for
energy generation, as specified above.
\item The SSR must be able to fabricate machinery and enclosures
to control and host a very wide set of processes
(e.g., material extraction, energy generation, possible chemical reaction
processes, electrolytic processes, and PCB etching chemical processes).
\item The SSR must be able to fabricate computers and computer parts
including microprocessors, integrated circuits, application specific
integrated circuits (ASICs), signal processing integrated circuits,
controller integrated circuits, printed circuit boards (PCBs), power
supplies, cabling, semiconductor memories, magnetic memories, and media
(hard drives, solid state drives). This also implies that the SSR must
feature highly demanding ``clean room'' spaces that robotically manipulate
materials and parts as well as perform semiconductor fabrication.
\end{itemize}
\index{self-replication!fabrication}

\subsection[The Information Closure Challenge and the Hardware/Software
Completeness Challenge]{The Information Closure Challenge and the
\mbox{Hardware/Software} Completeness Challenge}

The SSR's information closure 
is the requirement that the information contained in the SSR is
sufficient to drive its successful replication without any additional
external information. Completing the hardware and software
requirements further extends the information closure
requirement by demanding that the computing hardware and software
present in the SSR together with the information resident in the SSR
are sufficient to drive, control, and successfully complete the cloning
and division phases of SSR replication. The SSR's hardware and
software must provide full automation of the control, fabrication,
assemblage, and the handling of special situations like error detection,
error repair, and recovery after error.

The \memph{design of the information} resident in the SSR must be
appropriate for its self-replication. Its characteristics must be complete and adequate for the task. It must cover all relevant aspects that intervene during replication (e.g., materials, parts, processes, procedures, plans, spatial structures, error and recovery handling, etc.). Completeness means also that the information designed and stored in the SSR is correctly correlated with the SSR environment. That means, for example, that the SSR design should be based on an accurate and exhaustive list of raw materials and raw parts that exist in the SSR environment together with the material identification procedures and material processing/extraction procedures for those materials. Additionally, it must be adequate for the task in that it must cover all descriptive details of all entries in the information catalogs, with all relevant properties for these entries, including the correct representation of various relationships between the entries in the information catalogs.

The \mterm{computing hardware and software completeness requirement} has several implications. The designed computing hardware and software for each machine must be complete, sufficient, and adequate to control, drive, and monitor that particular machine. It must answer commands from the SSR control centers and properly communicate information, status, and control commands with other machines as needed to accomplish the higher level functions of the SSR. Additionally, the hardware and software that are used by various SSR functions and control centers are also complete, sufficient, and adequate in that they cover all possible use cases including errors and incidents.

\subsection[The Highest Challenge: The SSR Design Challenge]{The Highest
Challenge: The SSR Design Challenge}

The SSR design challenge simply
means that the SSR's design, including the design of all its subsystems
(reviewed in the previous sections), are adequate for accomplishing 
successful self-replication of the fully autonomous SSR with
preservation and without degradation of the self-replication capability
that is passed to all generations of daughter SSRs.

There are several specific aspects of the design challenge enumerated
below:

\begin{itemize}
\item The design of the SSR must be fully coordinated with the design
of the environment in which the SSR will be
placed. This means in particular that the SSR's design needs to be fully
informed about the nature, characteristics, and environmental conditions
(e.g., temperature, pressure, humidity, and aggregation status) of the medium
where it will exist, including the nature of raw materials and raw
parts that are present in this medium.
\item The analysis conducted so far reveals that the
design and construction of a fully autonomous self-replicating SSR are
\memph{extremely demanding}. The success of such a design and
construction appear to be heavily determined by the appropriate choices,
listed below, and how these choices harmonize with the SSR
environment:

\begin{itemize}
\item the material basis of the SSR (nano scale chemical basis or macro material basis)
\item the overall aggregation status of the SSR components: liquid,
solid (compact or with embedded spaces), aqueous, colloidal
\item the scale of the mature SSR 
\item the availability of energy-generation materials and processes in
the material basis of choice and at the scale of choice
\item the availability of well mastered techniques for the SSR material
basis of choice, scale of choice of fundamental engineering techniques
including energy-generation and transport, fabrication, assemblage and construction, transport and mobility, manipulation, computation, information communication, and sensing.
\end{itemize}
\end{itemize}
\index{self-replication!design!challenges|)}

\section{The Emerging Image of the Artificial SSR}

An artificial SSR is very similar to a modern city enclosed in a
dome-like structure that communicates with the outside world by
well-guarded gates used by robots to bring in construction materials
from outside the city. This modern city has two quarters: the “old
city” with its fully functional infrastructure in place including
buildings, plants, and avenues. The “new city” quarters are initially a
small, empty terrain. As the new city is being constructed and its area
extends, the dome covering gradually extends to cover both the
old, established city and the new, growing city. Both the old city and new city quarters
are pulsating with construction activity: automated machines (robots)
carry new materials, parts, and components that are used to construct
the infrastructure of the new city quarters into an exact replica of the old city and to continuously extend the dome on top of it.

The old city contains the following structures that must be replicated in the new city:

\begin{itemize}
\item material mining sub-units 
\item metallurgic plants
\item chemical plants
\item power plants
\item an electricity distribution network
\item a library with information for all city construction plans in
electronic form that is made available on the city web and used by the
city control centers, its machinery, and robots
\item a network of avenues, alleys, and conduits for robotized
transportation
\item fully automated and robotized manufacturing plants specialized in
fabrication of all parts, component assemblies, and machines that are
present in the old city
\item a fully automated semiconductor manufacturing plant with clean
rooms for fabrication of microprocessors, ASICs, memories, and other
highly integrated semiconductor circuits and controllers
\item a computer manufacturing plant
\item a network equipment manufacturing plant
\item an extended communication network connecting 
all plants and robots
\item a software manufacturing plant and software distribution and
installation of robotized agents
\item warehouses and stockrooms to store raw and fabricated materials,
parts, components, assemblies, and software on some storage media
\item a materials and parts recycling and refuse management plant
\item an army of intelligent robots for transportation, manipulation,
fabrication, and assemblage
\item an army of recycling robots that maintain clean avenues and
terrains in the city, collect debris from various plants and
reintroduce the recyclable materials and parts into the fabrication process
while the unusable parts are taken out of the city gates
\item control, command, and monitoring centers that coordinate the supply of
materials and the fabrication of an identical copy of
the original plants, avenues, factories, and stockrooms
\item a highly sophisticated, distributed, multi-layered software system
that controls all plants, robots, and communications in a cohesive
manner
\end{itemize}

Each one of the transport carriers, robots, manipulators, and
construction machinery is active and does its work without impeding the
movement of any other machine. Everything appears to be moving
seamlessly and orderly, and the construction of the new city is making
visible progress under the growing pylons of the bolting dome.

When the new city quarters are completed and they look
like the old city, the machinery and
robots of the new city become active, bringing the city to life. Something starts happening
as well: machinery and robots from both the old and the new city start
remodeling the supporting pylons and the arching dome.  
What used to be a single, super-arching dome is changing shape into two
separate domes one for each of the city quarters.

The final steps require the two teams of robots to coordinate as they complete the separation process of the old city and the new city. The new city has its own dome that has been shaped and separated from the original city. What were once two quarters of the same city is now two completely separate cities starting their own, separate destinies.
\index{self-replication!design|)}

\section[Attempts to Build Artificial Self-Replicators]{A Brief Survey of Attempts to Build Artificial Self-Replicators}

\index{self-replication!engineering attempts|(}
No successful attempt has been made so far to build a real
autonomous artificial SSR from scratch. W. M. Stevens summarizes the
situation in the abstract of his PhD thesis:

\begin{quote}
Research into autonomous constructing systems capable of constructing
duplicates of themselves has focused either on highly abstract logical models, such as
cellular automata, or on physical systems that are deliberately simplified so as to make
the problem more tractable \citep{stevens2009}.
\end{quote}

Stevens reviews some of the attempts made at building physical or
abstract self-replicating machines:

\index{Von Neumann!kinematic model}
\begin{itemize}
\item \textbf{Von Neumann’s kinematic model.} The system is comprised of a
control unit governing the actions of a constructing unit, capable of
producing any automaton according to a description provided to it on a
linear tape-like memory structure. The constructing unit picks up the
parts it needs from an unlimited pool of parts and assembles them into
the desired automaton. The project was far from being finished and
remained an abstract model when von Neumann died \citep{stevens2009}.
\index{Moses' programmable constructor}
\item \textbf{Moses' programmable constructor.} Matt Moses developed a physical constructor designed to be
capable of constructing a replica of itself under the control of a
human operator. The system is made of only eleven different types of
tailor-made plastic blocks \citep{moses2001}.
\index{self-replicating modular robots}
\item \textbf{Self-replicating modular robots.} Zykov, Mytilinaios,
Adams, and Lipson built a modular robotic system in which a
configuration of four modules can construct a replica configuration
when provided with a supply of additional modules in a location known
to the robot \citecustom{zykovetal2005}{Zykov, Mytilinaios, Adams, and Lipson}.
\index{RepRap project}
\item \textbf{The RepRap project and 3D printing}. Bowyer et al. have
developed a rapid prototyping system based around a 3D printer that is
capable of being programmed to manufacture arbitrary 3D objects. Many
of the parts of the printer that are assumed to be self-reproducing
cannot be manufactured by the system itself.  Those parts happen to be the
most complex ones (e.g., the computer controller) \citep{bowyer2007}.
\index{Drexler's assembler}
\item \textbf{Drexler's assembler.} In \textit{Engines of Creation}, K. Eric
Drexler describes a molecular assembler that is capable of operating at
the atomic scale. The molecular machine has a programmable computer, a
mobile constructing head and a set of interchangeable reaction tips
that will trigger chemical reactions designed to construct any object
at molecular scale \citep{drexler1986}.
This proposal stirred quite a controversy with some
scientists who were skeptical of the feasibility of such a project \citep{smalley2001}.
\index{Venter, Craig!synthetic bacterial cell}
\index{synthetic bacterial cell}
\item \textbf{Craig Venter's synthetic bacterial cell and synthetic
biology}. Craig Venter and the scientists at
J. Craig Venter Institute in Rockville, MD
reported in the May 20, 2010 issue of the journal
\textit{Science}
that they created a ``new species---dubbed Mycoplasma mycoides
JCVI-syn1.0---that is similar to one found in nature, except that the
chromosome that controls each cell was created from scratch'' \citep{smith2010, gibsonetal2010}. In the same ABC news report,
Mark Bedau, professor of Philosophy and
Humanities at Reed College in Portland, Oregon, 
called the new species ``a normal
bacterium with a prosthetic genome'' \citep{smith2010}.
\index{synthetic biology}
\item \textbf{Synthetic biology} is a new area of biological
research and technology that combines science and engineering. It
encompasses a variety of different approaches, methodologies, and
disciplines with a variety of definitions. The common goal is the
design and construction of new biological functions and systems not
found in nature \citep{heinemann2006}. There are interesting advances in this field with the
development of various techniques in domains like synthetic chemistry,
biotechnology, nanotechnology and gene synthesis. However, these are far
from constituting a complete, coherent, and effective set of techniques
that will allow the construction and synthesis of the large diversity
of machinery and functions that were identified in the preceding text as
the ``portrait'' of the artificial SSR.
\index{MEMS|see{micro-electro-mechanical systems}}
\index{micro-electro-mechanical systems}
\item \textbf{Micro-electro-mechanical systems (MEMS)} is the technology
of very small devices; MEMS are also referred to as
\textit{micromachines} in Japan or \textit{microsystems technology} (\textit{MST}) in Europe. 
MEMS are comprised of components between 1 to
100 micrometers in size with devices generally ranging between 20 micrometers (20 millionths of a meter)
to a millimeter (i.e. 0.02 to 1.0~mm)\citep{lyshevski2000}. The technology made significant
advances with several types of MEMS currently being used in modern
equipment.  Some examples include accelerometers, MEMS gyroscopes, MEMS microphones, pressure
sensors (used in car tire pressure sensors), disposable blood pressure
sensors, and micropower devices. This is probably one of the most
promising types of technology for implementing small-scale, artificial
SSRs. However, there are still significant hurdles some of which include implementation of
mobile elements (mini robots) and the common difficulty of fabricating
the machinery that fabricates MEMS and microprocessors.
\index{NASA REPRO study}
\index{NASA Advanced Automation for Space Missions 1980 Project|see{NASA REPRO study}}
\item \textbf{NASA Advanced Automation for Space Missions 1980 Project.}
This study, titled ``A Self-Reproducing Interstellar Probe'' (REPRO), 
is one of the most realistic explorations of the design of
an artificial ``macro'' self-replicator and is briefly analyzed in its
own section below \citep{freitas1980}.
\end{itemize}
\index{self-replication!engineering attempts|)}

\subsection[NASA Advanced Automation for Space Missions Project]{NASA
Advanced Automation for Space Missions Project}

\index{NASA REPRO study|(}
One of the missions of the NASA Advanced Automation for Space Missions Project
is described in chapter 1 of its final report:

\begin{quote}
Mission IV---Self-Replicating Lunar Factory---an
automated unmanned (or nearly so) manufacturing facility consisting of
perhaps 100 tons of the proper set of machines, tools, and teleoperated
mechanisms to permit both production of useful output and reproduction
to make more factories. \citep{freitasgilbreath1982}
\end{quote}

Later, in the same chapter, the project of making a self-replicating lunar factory is described in detail:

\begin{quote}
The
Replicating Systems Concepts Team proposed the design and construction
of an automated, multiproduct, remotely controlled or autonomous, and
reprogrammable lunar manufacturing facility able to construct
duplicates (in addition to productive output) that would be capable of
further replication. The team reviewed the extensive theoretical basis
for self-reproducing automata and examined the engineering feasibility
of replicating systems generally. The mission scenarios presented in
chapter 5 include designs that illustrate two distinct approaches---a
replication model and a growth model---with representative numerical
values for critical subsystem parameters. Possible development and
demonstration programs are suggested, the complex issue of closure
discussed, and the many applications and implications of replicating
systems are considered at length. \citep{freitasgilbreath1982}
\end{quote}

Figure \ref{fig:nasa_spirit} below is a piece of artwork that was reproduced from the
NASA study.  The description below the picture describes one of the features as a
self-replicating factory: ``{\ldots} In the lower left corner, a lunar
manufacturing facility rises from the surface of the Moon. Someday,
such a factory might replicate itself, or at least produce most of its
own components, so that the number of facilities could grow very
rapidly from a single seed'' \citep{freitasgilbreath1982}.

\migneafigure{MigneaNasaSpiritSpace.jpg}{NASA---The Spirit of Space Missions---created by Rick Guidice}{fig:nasa_spirit}


The project discusses certain details for a self-replicating lunar factory to be feasible. For one, the seed of the lunar factory that would need to be transported from Earth would likely weigh one hundred tons. Additionally, not all of the machinery could be built on the Moon. Certain items, such as computer boards, would need to be brought from Earth since parts could not be enclosed on the Earth’s surface. These additional, externally-synthesized items are similar to vitamins, which are compounds organisms need to survive but cannot usually synthesize themselves.  Scientists estimated that this project would be feasible in the 21st century.

Figure \ref{fig:lmf_parts_fabrication} illustrates the depth the project reached in considering
fabrication facilities on the Moon.

%% FIXME - cite NASA study
\migneafigurewide{MigneaPartsFabrication}{LMF Parts Fabrication Sector: Operations (Fig 5.17 in the NASA Study)}{fig:lmf_parts_fabrication}

The NASA study represents a thorough, realistic evaluation of the extent that designing a macro-scale (kilometers) self-replicator would entail. It addresses various problems that need to be solved and difficulties that would need to be overcome in order to successfully make a large self-replicator.

\subsection[The Self{}-Reproducing Interstellar Probe (REPRO) Study]{The
Self-Reproducing Interstellar Probe (REPRO) Study}

\index{Self-Reproducing Interstellar Probe Study|see{NASA REPRO study}}
\index{NASA REPRO study|(}
In 1980 Robert A. Freitas
published a study entitled ``A Self-Reproducing Interstellar Probe'' (referred to as the REPRO study) in
the \textit{Journal of the British Interplanetary Society}. Some of the
main goals of the project are summarized below.

\begin{itemize}
\item REPRO was designed to be a mammoth self-reproducing spacecraft to be built in
orbit around a gas giant such as Jupiter.
\item REPRO was a vast and ambitious project, equipped with numerous
smaller probes for planetary exploration, but its key purpose was to
reproduce. Each REPRO probe would create an automated factory that
would build a new probe every 500 years. Probe by probe, star by star,
the galaxy would be explored 
\item The total fueled mass of REPRO was projected to be 10**10 Kg = 10
**7 tons = 10 million tons for a probe mass of 100,000 tons.
\item It would take 500 years for REPRO to create a replica of itself in the
environment of a far-away planet
\item The estimated exploration time of the galaxy was 1--10 million
years
\end{itemize}
\index{NASA REPRO study|)}

\section[Simplifying Assumptions]{Simplifying Assumptions for the Design and Construction of an SSR}

\index{self-replication!design!simplified|(}
While the discussion so far has focused on building a fully-autonomous SSR, and
the difficulties encountered therein,
it is possible to reduce the complexity of its design and construction by making
some simplifying assumptions.  These might include:

\begin{enumerate}
\item Eliminate the requirement that the SSR produces its own energy.
The electrical energy (at an appropriate voltage/amperage) will be
supplied to the artificial SSR from the environment.
\item Eliminate the requirement that the SSR has the ability to select,
identify and accept through its input gateway appropriate raw
materials. All raw materials will be supplied as stock materials to the
artificial SSR. An additional, optional simplification could be that all stock
materials are labeled appropriately (e.g., with bar codes or RFIDs labels). 
However, as an illustration, the SSR will still need to use
stock copper fed through the input gateways to fabricate copper wires
of certain gauges or to use copper in the fabrication of electrical
motor parts.
\item Eliminate the requirement that the SSR fabricate the most technically
demanding parts, components, and assemblies (e.g.,
computer boards, microprocessors, semiconductor chips, memories,
etc.). These high-technology parts/components (referred to as ``vitamins'' in the
self-replication literature) will be supplied and carefully
labeled from the environment through the input gateways.
\item Eliminate the requirement that the SSR must fabricate any part,
component, assembly or machinery from basic materials. The SSR will be supplied
with pre-manufactured parts that are used in the composition of all
its machinery, scaffolding and enclosure. This simplifying assumption
means that now the SSR needs to be designed as a (sophisticated)
self-assembler that achieves self-replication by assembling exact
copies of itself using an exhaustive pool of all of the parts from the
machinery/assemblies it is composed of and are supplied by the elementary
parts coming through its input gateways.
\item Eliminate the requirement that the information repositories
(information catalogs) that drive the functions of the SSR reside
within the SSR. This requirement needs to be replaced with requirements
for the SSR to possess reliable, high speed communication capabilities
to access the information catalogs (and possibly part of the software)
residing somewhere outside the SSR. This assumption may simplify
certain elements of the SSR design but will make other
requirements, such as communication and availability, more stringent for both the
SSR and for the external information resource. 
\end{enumerate}

Even if the original requirements for the design and construction of an
autonomous, artificial SSR are relaxed and any or a combination of the
above simplifying assumptions are used as starting conditions for such
a project, there are still significant hurdles that need to be overcome
in designing and constructing an artificial SSR.
\index{self-replication!design!simplified|)}

\section{Conclusion}

As is evident, the three closure rules which must be satisfied by a
true self-replicator---energy closure, material closure, 
and the information closure---place an extraordinary burden onto
the design and implementation of self-replicating objects.  The
SSR must be able to produce and distribute energy, ingest raw
materials to fabricate parts, and contain a complete technical
description of itself, its processes, and its assembly instructions.
Additionally, the environment for the SSR must also be considered, 
and perhaps specially designed, in order to make available all
of the necessary raw materials for self-replication.

While such replicators are known to exist on a biomolecular scale within nature,
current technology does not allow for creating a self-replicator 
at such small scales.  The macro scale implementation is more suited
to present technology, though this brings its own problems regarding 
the scale of energy and materials consumption.  

While there have been many attempts to build self-replicators, none of them have
satisfied all three closure requirements.  The NASA REPRO study was the most
exhaustive attempt to design a self-replicator, which estimated the replicator
to weigh 100,000 tons (unfueled) and reproduce every 500 years.

While Part~1 and Part~2 of this paper contained an overview of the minimal technological
requirements of self-replication and foundational technological implementation considerations,
the existence of self-replication in nature comes as quite a surprise.  Therefore, Part~3
will cover speculative ideas for what the existence of the self-replication process in
nature indicates about the nature of reality.

\eandmbibliography{MigneaLibrary}

