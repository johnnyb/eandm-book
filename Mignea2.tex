\chapter[The Simplest Self-Replicator, Part 2]{Developing Insights into the Design of the Simplest Self-Replicator and its Complexity: Part 2---Evaluating the Complexity of a Concrete Implementation of an Artificial SSR}

\chapterauthor{Arminius Mignea}
\chapteraffiliation{The Lone Pine Software}

\begin{abstract}
This paper is the second in a three-part series investigating the internals 
of the simplest possible self replicator (SSR).  
This paper takes the
analysis offered by the first paper in the series, and considers various 
significant aspects that confronts the
design and construction of an artificial, concrete SSR: the material
basis of its construction, the effects of the variable geometry of the
SSR during its growth through the cloning and then division phases, and the
three closure rules that must be satisfied by the SSR---energy closure,
material closure and the information closure.

The highest technical
challenges that need to be faced by the design and construction of the
artificial SSR are discussed. The emerging complexity of the artificial
SSR is depicted using a metaphorical comparison of the replicating SSR
with a full city populated only by automated machinery and robots that
systematically and orderly construct the new city quarters identical
with the old city quarters with no help from outside but only the
construction materials entering through the city gateways. An
evaluation is made if the current level of technology is good enough
for the successful completion of a “design and construct an artificial
autonomous SSR” project either with a nano biochemical basis or a macro
material basis. 
\end{abstract}

In Part 1, the basic necessary design elements of the simplest self-replicator (SSR) 
was analyzed, including necessary components, functions, processes, and information.
Having established the minimum requirements for the design, the physical implementation
of the SSR will be discussed.

\section{The Three Closure Requirements as the Basis of an Autonomous SSR}

The SSR must be fully autonomous. This means that it can only get raw
materials and raw parts from its environment and benefit from (or
struggle because of) the environmental conditions specific to its
location.

Full autonomy means specifically\citep{freitasmerkle2004}:

\begin{enumerate}
\item The SSR must fabricate all its energy from input materials and the
generated energy must be sufficient for the SSR to produce an exact
replica of itself. This condition is called the \mterm{energy closure}. 
\item The SSR must use only materials admitted through its input gateways
and these materials must be sufficient for the SSR to grow and generate
its replica. This condition is called the \mterm{material closure}. 
\item SSR must use only information present/stored initially in the
“mature” SSR and this information must be sufficient to produce an
exact replica of the SSR. This condition is called the
\mterm{information closure}.
\end{enumerate}

\section{The Core Approach to Cloning}

In this section we try to answer the following important question for
the design of the artificial SSR: \memph{what is the core mechanism to
be used by the artificial SSR to accurately clone all of its elements?}

Below are two possible answers. And we believe that most any other
imagined answers would be similar or equivalent with one of the
two answers below:

\begin{enumerate}
\item Design and use a \mterm{universal physical copy machine} (similar to a key copy
machine but much more sophisticated) which analyzes each part or assembly and produces
a copy based on that analysis, with the goal being to simplify the design and
avoid having to maintain such detailed catalogs of information.
\item Use an exhaustive descriptive, operational and constructional SSR
information database driving an integrated set of specialized, software
and computer-controlled automatons.  In other words, have sufficient
data stored about the makeup of the SSR itself to generate a new copy
from that data.
\end{enumerate}

\subsection[Why the “universal physical copy machine” approach is not adequate]{Why
the “universal physical copy machine” approach is not adequate}

Let’s describe in more detail
what this type of solution would mean. This approach assumes that the
SSR contains a sophisticated machine that can examine and accurately
copy all other pieces and machinery making up the mature SSR. This
implies that this universal physical copy machine can even copy itself or, more
realistically, a copy of itself. In other words, the SSR contains two
universal physical copy machines.  One, Machine A, is going to do the actual copy of
all SSR machinery and do the copy of the other universal physical copy machine,
machine B. So the second copy machine, machine B, is just used as a
model for the first machine A. The emphasis on this solution is
that these copy machines A and B will alleviate the need to store
so much information about the SSR itself, to have so much software and so much
computer-controlled machinery as you will see in the second approach.

Considering this first solution leads to the following conclusions:

\begin{enumerate}
\item There will be a need to have another machinery
M\textsuperscript{disassembler}, to disassemble machine B in all its
constituent parts: b\textsubscript{1}, b\textsubscript{2},
b\textsubscript{3}, …, b\textsubscript{N }so that machine A can copy
each constituent part.
\item The machine A will copy all parts b\textsubscript{1},
b\textsubscript{2}, b\textsubscript{3}, …, b\textsubscript{N} twice (to
create all pieces needed for a Clone machine A: Copy\textsubscript{A}
and a clone machine B: Copy\textsubscript{B}) ca\textsubscript{1},
ca\textsubscript{2}, ca\textsubscript{3}, …, ca\textsubscript{N }(for
Copy\textsubscript{A }machine) and cb\textsubscript{1},
cb\textsubscript{2}, cb\textsubscript{3}, …, cb\textsubscript{N} (for
Copy\textsubscript{B} machine).
\item There will be a need for another machinery
M\textsuperscript{assembler} that will know how to take all the parts
ca\textsubscript{1}, ca\textsubscript{2}, ca\textsubscript{3}, ….,
ca\textsubscript{N} and assemble them together into the
Copy\textsubscript{A} machine and parts cb\textsubscript{1},
cb\textsubscript{2}, cb\textsubscript{3}, …, cb\textsubscript{N} and
assemble them together into the Copy\textsubscript{B}. 
\end{enumerate}

But now M\textsuperscript{assembler} must be considered.  In order
to properly construct M\textsuperscript{assembler}, a vast amount
of well-structured information of this nature must be first programmed:

\begin{itemize}
\item A catalog of all parts b\textsubscript{1}, b\textsubscript{2},
b\textsubscript{3}, …, b\textsubscript{N} and for each such part a
unique identifier and possibly physical and geometrical characteristics
(dimensions) of the part
\item A store room location (x, y, z)  from where the
M\textsuperscript{assembler} machine will pick each one of the parts
ca\textsubscript{1}, ca\textsubscript{2}, ca\textsubscript{3}, …,
ca\textsubscript{N} during assembly steps to construct the
Copy\textsubscript{A} machine
\item A \mterm{catalog of assembly instructions} (some geometrical x,
y, z instructions, what kind of assemblage step: like screwing,
inserting, welding, etc. on:

\begin{itemize}
\item how to put together part ca\textsubscript{2} to the assembly made
of parts: (ca\textsubscript{1})
\item how to add part ca\textsubscript{3} to the assembly made of parts
(ca\textsubscript{1}, ca\textsubscript{2})
\item how to add part ca\textsubscript{4} to assembly made of parts
(ca\textsubscript{1}, ca\textsubscript{2}, ca\textsubscript{3}),
\item …….
\item how to add part ca\textsubscript{N} to the assembly made of parts
(ca\textsubscript{1}, ca\textsubscript{2}, ca\textsubscript{3},
ca\textsubscript{4}, …., ca\textsubscript{N-1})
\end{itemize}
\item A manipulator machine (robot that can follow computerized
instructions) to be controlled by the M\textsuperscript{assembler}
machine in assembling the CopyA machine
\end{itemize}

So, although the original intent for Solution 1 is to avoid mountains of
information and armies of automatons and machinery, our analysis
revealed that this solution cannot do without those ingredients. 
There is no magic copy machine that can do its work, without structured
collections of information and many helper automatons (machinery) that
in turn must be information, software, and computer controlled.  
The conclusion is that there is no magic “universal physical copy machine” solution
that is significantly distinguishable from the Solution 2. 

Another problem for Solution 1 is that certain components of machine B may not
be fabricated by plain (mechanical) assemblage of parts but rather by
using more demanding assemblage processes, such as welding or 
electro-chemical processes.  These processes have no precise means of disassembly,
which would prevent the universal physical copy machine from being able to reproduce
them.

\subsection[Exhaustive information, integrated systems driving
information{}-controlled automatons]{Exhaustive information, integrated
systems driving information-controlled automatons}

Since the universal physical copy machine approach does not work, the only
other replication methodology available is to thoroughly and carefully design
the SSR as a collection of
integrated sub-systems, controlling a large variety of automatons using
a significant collection of integrated information catalogs
(databases).

We already mentioned a number of information catalogs while identifying
specific SSR functions. Any informational SSR function has both an
associated catalog and also a set of access sub-functions that provide
a set of access operations to the information catalog that can be used
by other SSR functions to execute specific action sequences.

\section{The Material Basis of the SSR}

When approaching the task of the design and implementation of an
artificial SSR a capital question surfaces rather soon. What should be
the material basis for the artificial SSR? We see here two distinct
possibilities---either use a biological basis for the SSR on a micro/nano 
scale or use more common macro scale materials and technology.

\subsection{Using a Biological Basis for the SSR}

Using a biological basis for the SSR means that the SSR must be constructed using organic materials, the
same or similar with those used by the cells, tissues and organs of the
living world.

Here are the \memph{advantages} of this approach:

\begin{itemize}
\item We know that this approach is possible, proof being the presence
of so many organisms and microorganisms in nature. The question here is
if this approach is accessible to our current engineering technologies.
\item There are low levels of energy consumption associated with this
approach and a solution for generation of enough energy for SSR
replication is one of the biggest challenges.
\item The aqueous medium of this approach may facilitate solutions for
the variable geometry problem.
\end{itemize}

Here are the \memph{problems or barriers} for this approach:

\begin{itemize}
\item The investigative tools and observation means available to us at
this scale are still quite limited
\item The most advanced micro-biology manipulation and fabrication
tools/approaches are still rather primitive and very limited when
considering the tasks that need to be accomplished: fabrication,
assemblage, manipulation at nano scales, computing machinery
fabrication, software execution, information storage, and communication.
\item We are just at the beginning of the process of understanding how
cells work. There are many areas and aspects of cell biology that still
appear to be beyond our comprehension.
\item Examples of challenges that we cannot solve with current
technology: 

\begin{itemize}
\item building computers on a biological material basis / scale (or
understanding how the cell proteins and other organic cell elements can
be used for computation in a general way)
\item building bio-chemical manufacturing machines at a biological
scale.
\item building information storage with a biochemistry material /scale
\item having software running on biological type computers.
\item communicating information on a biological material basis /scale.
\end{itemize}
\end{itemize}

The conclusion is that it is quite clear that with the current level of
technology it is impossible to create a design and an implementation
plan for an artificial SSR using biochemistry and a biological material
basis and a molecular scale. We are forced to consider other
alternatives to have better chances of success.

\subsection{Using a Macro Scale Basis for the SSR}

The other alternative to consider is the macro scale approach, using materials
and technology that is in common use for product fabrication.
This approach means that we will need to consider and use the minimum
dimensional scales for which there are available manufacturing
technologies for most of the parts, components, and machinery that make
up the SSR.  

The materials used to construct the SSR enclosure, SSR scaffolding, and
SSR interior elements should be common engineering materials used by current
fabrication technologies: metals, alloys, plastics, ceramics, silicon
or other special materials.  The scale of these artifacts to be fabricated 
as elements of the artificial SSR must be selected with care as there
are two opposing considerations which must be balanced and compromised.

The first consideration regarding the scale of the materials is that the 
smallest possible scale should be used in the design and
implementation of the artificial SSR parts in order to:

\begin{itemize}
\item minimize the energy consumed by the SSR during a replication cycle
\item minimize the size, volume and mass of the artificial SSR and thus
the amount of materials “ingested” into the artificial SSR and used for
fabrication of the clone inside the SSR.
\end{itemize}

However, this must be balanced with the consideration that the scale 
selected for the artificial SSR must be such that there
are known engineering technologies and machinery to fabricate,
manipulate and assemble all the parts of all the SSR machinery. This
means---for an illustrative example---that if the minimum size of a
semiconductor fabrication equipment that is being manufactured today is
0.5 meters then the designed size of the mature artificial SSR cannot
be smaller than 1 meter. Based solely on this example a more realistic artificial
SSR designed dimension should be in the range of at least 10-100
meters.

\section{The Type and Nature of SSR Components}

We concluded in Part 1 that SSR must be designed and implemented as
a collection of integrated, computer-controlled and software controlled
automatons. This helps us to identify the nature of some of the
elements and components that make up the SSR. The artificial SSR must,
by necessity, contain these types of elements:

\begin{itemize}
\item Computing machinery which implies that the following type of
elements must be present inside the artificial SSR:

\begin{itemize}
\item Computers
\item Printed circuit boards (PCB)
\item Microprocessors
\item Highly integrated circuits (Application Specific Integrated
Circuits = ASICs)---specialized, high density integrated circuits for
specific computing/application tasks: networking, numerical processing,
image processing, etc.
\item Semiconductor memories (solid state memories)
\item Magnetic memory (hard drives)
\item Electric power supplies
\item Computer connectors and wiring
\end{itemize}
\item Networking Communication Devices:

\begin{itemize}
\item Routers (wired/wireless)
\item Switches
\item Modems
\end{itemize}
\item Software
\item Robots
\item Energy generation and distribution machinery

\begin{itemize}
\item Generators
\item Transformers
\item Converters
\item Wiring
\end{itemize}
\item Batteries
\item Fabrication machinery
\item Metal machining machinery
\end{itemize}

\section{Derived Design Requirements}

In this section we are presenting a list of design and implementation
requirements for the artificial SSR that emerged from the previous
analysis and from the inferences presented so far. These requirements
were only implied during our discussion so far but now they are made
explicit and described in some detail.

\subsection[Each SSR Machine is Power{}-Driven]{Each SSR Machine is
Power-Driven}

This condition has the following significant consequences for the SSR design.

\begin{itemize}
\item The SSR must have a power distribution network (electrical
distribution network) that must reach each of the SSR machinery. The
design of the layout and geometry of the power network must consider
the variable geometry of the SSR enclosure, scaffolding and its
interior space and structure – in particular in the zones affected by
growth and shape changes.
\item Each SSR machine must be designed to use and consume power
(electricity) at a level adequate for its nature and the actions it
performs. 
\item Most (all) machinery is computer driven, which means that either
there is a parallel SSR power network for an energy level (voltage)
adequate for computing devices, or each machine must have some adequate
power converters (electrical power supplies, batteries).
\item The SSR machines that provide mechanical work or movement must be
provided with motors (rotational, linear) adequate for their nature.
\item Mobile machinery (transporters, moving robots) must be designed
such that their mobility is not inhibited or constrained while they
are/remain plugged into the SSR power network(s).  Designing all mobile
machinery with rechargeable batteries may solve or significantly
simplify the connectivity constraints but will require additional
provisions for battery fabrication processes and fabrication and
provision of battery charging stations.
\item The design of each machine must provide specification of average
power consumption on all power networks (normal power level and
computer power level) to which the machine is connected.
\end{itemize}

\subsection[Each SSR Machine is Computer{}-Driven and
Software{}-Driven]{Each SSR Machine is Computer-Driven and
Software-Driven}

These are the consequences for
the SSR design derived from this condition:

\begin{itemize}
\item Most SSR machines must host at least one internal computer (with
the possible exception of some simpler (electro-mechanical) machines
that can be remotely controlled).
\item The SSR must have high technology machinery and processes needed
to fabricate computers and all their building blocks (parts)
\item Each SSR machine that hosts computers must be networked---connected 
by wire or wirelessly to other machines and control centers
in the SSR.
\item The SSR must have machinery that has the ability to not only
fabricate computers but also to install them into other SSR machines,
to plug them to that machine power network, and to connect them to
the SSR communication network.
\item The SSR must have machinery that is able to download and copy
software into any computer installed into a SSR machine and to start
(boot) that software on that machine and to monitor its availability
and expected behavior.
\item The SSR must have the capability to test each of its machinery, to
detect malfunctions (errors) in computer and software installations as
well as in the machine hardware, to detect malfunction in the computer
and software execution and to have adequate procedures to diagnose and
repair the identified problems based sometimes on the availability of
fabricated spare parts.
\item The software that drives each particular machinery must be
designed and written with a full understanding of the physical and
cinematic capabilities and constraints/limitations of that machinery.
It must take into account all possible use cases of the machine and all
its components behaviors and interactions with external objects and
events and handle them correctly.
\item The software that drives fabrication and assemblage machinery and
materials and fabrication processes needs to be based on a thorough
design of the machines to be built, their cinematic capabilities and
their specified power and energy consumption.
\item The software written for various SSR functions must carefully and
accurately choreograph, coordinate and synchronize the activities of
multiple SSR machines (fabrication machines, material process machines,
manipulation and transport robots/arms, assemblage and construction
machines) providing a continuous monitoring of the 3D spaces occupied
by each machine and its mobile parts, to avoid collisions and to ensure
cooperative progress with both lower level and higher level tasks of
the growing SSR.
\end{itemize}

\subsection[Each SSR Machinery is Information Communication
Capable]{Each SSR Machinery is Information Communication Capable}

These are the consequences for
the SSR design derived from this condition:

\begin{itemize}
\item The artificial SSR is a collection of automated machines and
robots. Their cooperation and coordination for achieving tasks from the
simplest (fabricating a part, or manipulating a part in a sequence of
steps for an assemblage operation) to the complex ones (like the
fabrication, the assemblage of computing hardware and software
installation for a new fabrication machinery) requires extensive,
continuous, multipoint, and multi-level (hierarchical) communication of
information between machines, control functions, and software
components.
\item The SSR must have a comprehensive physical layer communication
network for information transport (either wire-based and/or wireless)
with access points located on each (or most) SSR machines/robots and
sometimes in between the subsystems of the same SSR machine.
\item The SSR might need to have adequate networking devices (like
routers, switches, modems, codecs) to implement needed communication
patterns and topologies.
\item The SSR machines and software components engaged in communication
will need adequate networking/communication protocols with appropriate
characteristics for the needed communication bandwidth, handling of
errors and retransmissions, reliable routing, and end point addressing.
\item The SSR should have capabilities to deploy software on newly
constructed machines and network nodes, and be able to bring up the
network, verify it as part of starting up the “daughter” SSR system
(including its underlying communication network) as a preparatory step
in the SSR division phase.
\end{itemize}

\section[The Most Significant Challenges]{The Most Significant Challenges for the Design and Implementation of an Artificial SSR}

In this section we are going to enumerate and discuss the most difficult
challenges and hurdles that the design and implementation of an
artificial SSR faces.

\subsection[The energy generation and the energy closure challenge]{The
energy generation and the energy closure challenge}

This challenge presents multiple
aspects discussed below.

\begin{itemize}
\item The \memph{selection of an adequate basis for energy generation}.
Basically this depends on what natural materials that are available in
the SSR environment can be used for energy generation
\item We enumerate below some of the candidate material basis for energy
generation that might be considered for the design and implementation
of an artificial SSR:

\begin{itemize}
\item Biochemical or organic (like vegetation used for energy
generation), biogas, biodiesel
\item Coal
\item Oil/petroleum
\item Natural gas
\item Methanol
\item Hydrogen (micropower, fuel cells)
\item Solar
\item Wind
\item Nuclear
\end{itemize}
\item The \memph{energy closure challenge} means that the amount of
energy generated by the SSR from the primary energy producing materials
accepted/extracted from the SSR environment must be sufficient to power
all machinery (fabrication, assemblage, construction, transport,
manipulators, robots, computers, and networking gear) that equip the
SSR.
\item The SSR must be designed with the ability to slow down or even
completely shut down during the periods when the input of energy
producing materials is reduced or null.
\item The SSR ability to provide means to store energy (with batteries,
accumulators or stocking energy producing materials) may smooth out or
eliminate the need for transition to “hibernation” or shutdown states
of the SSR.
\item Designing SSR machines with local sources of energy (like
rechargeable batteries, accumulators, fuel reservoirs or fuel cells)
may provide all these machines with true, unconstrained mobility and
will significantly simplify the SSR design and implementation
difficulties related to keeping all mobile machines hooked to flexible
power wiring or network wiring.
\item Burning of fuels or preparing the materials for energy generation
(chemicals) increases the concerns that the design and SSR
implementation need to consider. Examples of such concerns are: 

\begin{itemize}
\item preserving SSR internal environmental parameters: temperature,
humidity; 
\item avoiding hazardous materials or avoiding fires;
\item providing storage and transportation containers for non-solid
(liquid or gaseous) materials.
\end{itemize}
\end{itemize}
\subsection[The material closure challenge]{The material closure
challenge}
\hypertarget{RefHeading3140306210128}{}The material closure challenge
for the artificial SSR design and implementation can be formulated this
way: all fabrication materials that are needed for fabrication of all
parts of the SSR machinery must be available in the SSR environment or
must be extractable from raw materials available in the SSR
environment.

This challenge may not appear to be so daunting. However if we just
think that the artificial SSR will have fabrication machinery for metal
machining, computers with semiconductor microprocessors and memories,
plastics, ceramics, we realize that there is potentially an extremely
large list of materials needed for SSR fabrication. Only a very short
subset of materials needed for the “clunky” (macro scale) artificial
SSR can give us an idea of how much a challenge this aspect can become:

\begin{itemize}
\item Iron,
\item Steel (of various varieties)
\item Copper,
\item Aluminum
\item Metal alloys (of different varieties)
\item Silver,
\item Gold,
\item Ceramics,
\item Plastics
\item Silicon
\item Polytetrafluoroethylene (Teflon) for Printed Circuit Boards
(PCBs),
\item Tin,
\item Nickel,
\item Germanium
\end{itemize}

Even if the list above is only partial it appears that there is a very
small probability that the environment where the artificial SSR may be
placed may feature such a large diversity of immediately available
materials or materials from which the materials in the list can be
somehow extracted. This makes the \memph{material closure requirement
to appear unsolvable} and thus any project to design and implement an
artificial “clunky” replicator may be condemned to failure. 

\subsubsection[The match between the SSR design and the design of the
SSR environment]{The match between the SSR design and the design of the
SSR environment}

Another way to formulate the challenge of the material closure is that a
successful selection of materials used for energy generation and for
fabrication of SSR internal parts, components and machinery must be
based on a thorough knowledge of the environment in which the designed
SSR is projected to exits, the nature of raw materials and parts in
such environment and if there are realistic material extraction paths
and processes starting with those materials in the environment. For us
there is no exaggeration to state that the chances of a successful
design and implementation of an artificial SSR depends on the design of
its projected environment. In other words the success requires a
perfect design both of the SSR and of its environment.

\subsection[The Fabrication Challenge]{The Fabrication Challenge}

\hypertarget{RefHeading3144306210128}{}The fabrication challenge can be
stated as the requirement that the artificial SSR must be able to
fabricate and assemble any type of parts, components and machines that
are part of the “mature” SSR which implies that all fabrication and
assemblage machines should be able to fabricate exact copies of
themselves.

While the material closure challenge focuses on the difficulty of having
available a very wide spectrum of fabrication materials, the
fabrication challenge also raises a wide spectrum of concerns:

\begin{itemize}
\item Since the artificial SSR will have a lot of machinery made with
metals (fabrication machinery, construction and assemblage machines,
robots, manipulator arms, networking gear,  wires, power supplies,
conduits, scaffolding), the SSR must have metal machining machinery –
probably of a large diversity. 
\item The SSR must be able to fabricate machinery and enclosures for
energy generation.
\item The SSR must be able to fabricate machinery and enclosures
(recipients) to control and host an undefined set of processes
(material extraction, energy generation, possible chemical reaction
processes, electrolytic processes, PCB etching chemical processes)
\item The SSR must be able to fabricate computers and computer parts
including microprocessors, integrated circuits, application specific
integrated circuits (ASICs), signal processing integrated circuits,
controller integrated circuits, printed circuit boards (PCBs), power
supplies, cabling, semiconductor memories, magnetic memories and media
(hard drives, solid state drives). This also implies that the SSR must
feature highly demanding “clean room” spaces with robotic manipulation
of materials and parts and semiconductor fabrication equipment.
\end{itemize}

\subsection[The Information Closure Challenge and the Hardware/Software
Completeness Challenge]{The Information Closure Challenge and the
Hardware/Software Completeness Challenge}

\hypertarget{RefHeading3146306210128}{}The information closure for the
SSR is the requirement that the information resident on the SSR is
sufficient to drive its successful replication without any additional
information coming from outside the SSR. The hardware and software
completeness requirement further extends the information closure
requirement with the demand that the computing hardware and software
present in the SSR together with the information resident in the SSR
are sufficient to drive, control and successfully complete the cloning
and division phases of the replication of the SSR. The SSR hardware and
software must provide full automation of the control, fabrication,
assemblage and the handling of special situations like error detection,
error repair and recovery after error.

The \memph{design of the information} resident in the SSR must be
appropriate for its self-replication. It must be:

\begin{itemize}
\item \textbf{Complete (exhaustive):} it must cover all relevant aspects
(materials, parts, processes, procedures, plans, spatial structures,
error and recovery handling, etc.) that intervene during replication.
Completeness means also that the information designed and stored in the
SSR is correctly correlated with the design of the SSR environment.
That means, for example that the SSR design should be based on an
accurate and exhaustive list of raw materials and raw parts that exist
in the SSR environment together with the material identification
procedures and material processing/extraction procedures for those
materials.
\item \textbf{Adequate:} must cover all descriptive details of all
entries in the information catalogs, with all relevant properties for
these entries, with correct representation of various relationships
between the entries in the information catalogs.
\end{itemize}

The \mterm{computing hardware and software completeness requirement}
means the following:

\begin{itemize}
\item The designed computing hardware and software for each machine is
complete, sufficient and adequate to control, drive and monitor that
machine and to answer commands from the SSR control centers and to
properly communicate and exchange information, status and control
commands with other machines as needed to accomplish the higher level
functions of the SSR..
\item The hardware and software that are used by various SSR functions
and control centers are also complete, sufficient and adequate: they
cover all possible use cases including all possible errors and
incidents.
\end{itemize}

\subsection[The Highest Challenge: the SSR Design Challenge]{The Highest
Challenge: the SSR Design Challenge}

\hypertarget{RefHeading3148306210128}{}The SSR design challenge simply
means that the design of the SSR and the design of all its subsystems
(reviewed in the previous sections) are adequate for accomplishing a
successful self-replication of the fully autonomous SSR with
preservation and no degradation of the self-replication capability
passed to all generations of daughter SSRs.

There are several specific aspects of the design challenge enumerated
below.

\begin{itemize}
\item The design of the SSR must be fully coordinated with the design
(and thus the knowledge) of the environment in which the SSR will be
placed. This means in particular that the SSR design need to be fully
informed about the nature, characteristics and environmental conditions
(temperature, pressure, humidity, aggregation status) of the medium
where it will exist, including the nature of raw materials and raw
parts that are present in this medium.
\item The analysis and investigation conducted so far revealed that the
design and construction of a fully autonomous self-replicating SSR are
\memph{extremely demanding}. The success of such a design and
construction appear to be heavily determined by appropriate choices
listed below and how these choices harmonize or not with the SSR
environment and its nature:

\begin{itemize}
\item The material basis of the SSR. Is it a carbon-based chemistry
(organic material basis) or not?  Is it a macro (“clunky”) material
basis or not?
\item The overall aggregation status of the SSR components: liquid,
solid (compact or with embedded spaces), aqueous, colloidal.
\item The scale of the mature SSR. Is it in the nanometer, micrometer,
millimeter, meter, kilometer scale?
\item The availability of energy generation materials and processes in
the material basis of choice and at the scale of choice.
\item The availability of well mastered techniques for the SSR material
basis of choice, scale of choice of fundamental engineering techniques
like:

\begin{itemize}
\item Energy generation and transport
\item Fabrication
\item Assemblage and construction
\item Transport and mobility
\item Manipulation
\item Computation
\item Information communication
\item Sensing
\end{itemize}
\end{itemize}
\end{itemize}

\section{The Emerging Image of the Artificial SSR}

An artificial SSR is very similar to a modern city enclosed in a
dome-like structure that communicates with the outside world by
well-guarded gates used by robots to bring in construction materials
from outside the city. This modern city has two quarters: the “old
city” with its infrastructure in place and fully functional with
buildings, plants and avenues. The “new city” quarters are initially a
small empty terrain. As the new city is being constructed, its area
extends together with gradual extension of the dome covering both the
old and the growing new city. Both the old city and new city quarters
are pulsating with construction activity: automated machines (robots)
carry new materials, parts and components that are used to construct
the infrastructure of the new city quarters, to continuously extend the
dome on top of it and to construct in the new quarters an exact replica
of the old city.

Before the construction of the new city quarters even began, here is
what can be seen in the old city:

\begin{itemize}
\item material mining sub-units 
\item metallurgic plants
\item chemical plants
\item power plants
\item an electricity distribution network
\item a library with information for all city construction plans in
electronic form that is made available on the city web and used by the
city control centers, its machinery and robots to achieve their tasks. 
\item a network of avenues, alleys and conduits for robotized
transportation
\item fully automated and robotized manufacturing plants specialized in
fabrication of all parts, components assemblies and machines that are
present in the old city.
\item a fully automated semiconductor manufacturing plant with clean
rooms for fabrication of microprocessors, ASICs, memories and other
highly integrated semiconductor circuits and controllers.
\item a computer manufacturing plant
\item a network equipment manufacturing plant
\item an extended communication network connecting by wire or rather
wirelessly all plants and robots
\item a software manufacturing plant and software distribution and
installation robotized agents.
\item warehouses and stockrooms to store raw and fabricated materials,
parts, components, assemblies and software on some storage media.
\item a materials and parts recycling and refuse management plant
\item an army of intelligent robots for transportation, manipulation,
fabrication and assemblage.
\item an army of recycling robots that maintain clean avenues and
terrains in the city, collect debris from various plants and
reintroduce the recyclable materials and parts in the fabrication and
construction while the bad parts are taken out of the city gates.
\item control, command and monitor centers that coordinate the supply of
materials, the fabrication and the construction of an identical copy of
the original plants, avenues, factories and stockrooms
\item a highly sophisticated, distributed, multi-layered software system
that controls all plants, robots and communications in a cohesive
manner.
\end{itemize}

Each one of the transport carriers, robots, manipulators and
construction machinery is active and does its work without impeding the
movement of any other machine. Everything appears to be moving
seamlessly, orderly and the construction of the new city is making
visible progress under the growing pylons of the bolting dome.

When the new city quarters are completed and they are looking exactly
like the old city, you can see some activity and the machinery and
robots of the new city coming out “to life”. Something starts happening
as well: machinery and robots from both the old and the new city start
a re-modeling of the supporting pylons and of the arching dome. Little
by little their plans start making sense and it can be seen now that
what used to be a single, super-arching dome is changing shape into two
separate domes one for each of the city quarters.

The last robot activity is as coordinated as it ever was: you see the
two teams of the robots, finishing the re-shaping of the domes and the
complete separation of the old city and the new city with their own
round shaped roof. The separating wall is completed and what was once –
not long time ago - the dome hosting the two quarters is now two,
totally separate “old cities” starting their own, separate destinies.

\section{A Brief Survey of Attempts to Build Artificial Self Replicators}

No successful attempt has been made so far for building a real
autonomous artificial SSR from scratch. W. M. Stevens summarizes the
situation in the abstract of his PhD thesis:

\begin{quote}
Research into autonomous constructing systems capable of constructing
duplicates of themselves has focused either on highly abstract logical models, such as
cellular automata, or on physical systems that are deliberately simplified so as to make
the problem more tractable\citep{stevens2009}
\end{quote}

Stevens then reviews some of the attempts made at building physical or
abstract self-replicating machines:

%% FIXME - definitely need to add references here

\begin{itemize}
\item \textbf{Von Neumann’s kinematic model.} The system is made up of a
control unit governing the actions of a constructing unit, capable of
producing any automaton according to a description provided to it on a
linear tape-like memory structure. The constructing unit picks up the
parts it needs from an unlimited pool of parts and assembles them into
the desired automaton. The project was far from being finished and
remained an abstract model when von Neumann died.
\item \textbf{Moses{\textquotesingle} programmable constructor.} Matt Moses developed a physical constructor designed to be
capable of constructing a replica of itself under the control of a
human operator. The system is made of only 11 different types of
tailor-made plastic blocks.\citep{moses2001}
\item \textbf{Self-replicating modular robots.} Zykov, Mytilinaios,
Adams and Lipson built a modular robotic system in which a
configuration of four modules can construct a replica configuration
when provided with a supply of additional modules in a location known
to the robot.\citep{zykovetal2005}
\item \textbf{The RepRap project and 3D printing}. Bowyer et al.  have
developed a rapid prototyping system based around a 3D printer that is
capable of being programmed to manufacture arbitrary 3D objects. Many
of the parts of the printer that is assumed to be self-reproducing
cannot be manufactured by the system and those parts happen to be the
most complex ones (the computer controller for example).\citep{bowyer2007}
\item \textbf{Drexler’s assembler.} In \textit{Engines of Creation}, K. Eric
Drexler describes a molecular assembler that is capable of operating at
the atomic scale. The molecular machine has a programmable computer, a
mobile constructing head and a set of interchangeable reaction tips
that will trigger chemical reactions controlled to construct any object
at molecular scale.  \citep{drexler1986}
This proposal stirred quite a controversy with some
scientists being skeptical of the feasibility of such a project.\citep{smalley2001}
\item \textbf{Craig Venter’s synthetic bacterial cell and synthetic
biology}. Craig Venter and the scientists at
J. Craig Venter Institute in Rockville, MD
reported in the May 20, 2010 issue of journal
\textit{Science}
that they created a “new species---dubbed Mycoplasma mycoides
JCVI-syn1.0---that is similar to one found in nature, except that the
chromosome that controls each cell was created from scratch.”\citep{smith2010}\citep{gibsonetal2010} In the same ABC news report,
Mark Bedau, professor of Philosophy and
Humanities at Reed College in Portland, Ore., also writing in the
Nature commentary, called the new species ``a normal
bacterium with a prosthetic genome.''\citep{smith2010}
\item \textbf{Synthetic
biology} is a new area of biological
research and technology that combines science and engineering. It
encompasses a variety of different approaches, methodologies, and
disciplines with a variety of definitions. The common goal is the
design and construction of new biological functions and systems not
found in nature.\citep{heinemann2006} There are interesting advances in this field with
development of various techniques in domains like synthetic chemistry,
biotechnology, nanotechnology and gene synthesis. However these are far
from constituting a complete, coherent and effective set of techniques
that will allow the construction and synthesis of the large diversity
of machinery and functions that we identified in the preceding text as
the “portrait” of the artificial SSR.
\item \textbf{Micro-electro-mechanical systems (MEMS)} is the technology
of very small devices; MEMS are also referred to as
\textit{micromachines} (in Japan), or \textit{micro systems technology}
– \textit{MST} (in Europe). MEMS are made up of components between 1 to
100 micrometers in size, with devices generally ranging between 20 micrometers (20 millionths of a meter)
to a millimeter (i.e. 0.02 to 1.0~mm)\citep{lyshevski2000}. The technology made significant
advances with several types of MEMS being currently used in modern
equipment:  accelerometers, MEMS gyroscopes, MEMS microphones, pressure
sensors (used in car tire pressure sensors), disposable blood pressure
sensors and micropower devices. This is probably one of the most
promising types of technology for implementing small scale artificial
SSRs. However, there are still significant hurdles: implementation of
mobile elements (mini robots) and the common difficulty of fabricating
the machinery that fabricates MEMS and microprocessors.
\item \textbf{NASA Advanced Automation for Space Missions 1980 Project}
. This study, titled ``A Self-Reproducing Interstellar Probe'' (REPRO), 
is one of the most realistic exploration of the design of
an artificial ``macro'' self-replicator and is briefly analyzed in its
own section below.\citep{freitas1980} 
\end{itemize}

\subsection[NASA Advanced Automation for Space Missions Project]{NASA
Advanced Automation for Space Missions Project}

\hypertarget{RefHeading3154306210128}{}One of the missions of this
project is described in its Chapter 1:

\begin{quote}
Mission IV - Self-Replicating Lunar Factory - an
automated unmanned (or nearly so) manufacturing facility consisting of
perhaps 100 tons of the proper set of machines, tools, and teleoperated
mechanisms to permit both production of useful output and reproduction
to make more factories.
\end{quote}

Then, later, it is described in more details:

\begin{quote}
(d) Replicating Systems Concepts Team. The
Replicating Systems Concepts Team proposed the design and construction
of an automated, multiproduct, remotely controlled or autonomous, and
reprogrammable lunar manufacturing facility able to construct
duplicates (in addition to productive output) that would be capable of
further replication. The team reviewed the extensive theoretical basis
for self-reproducing automata and examined the engineering feasibility
of replicating systems generally. The mission scenarios presented in
chapter 5 include designs that illustrate two distinct approaches - a
replication model and a growth model - with representative numerical
values for critical subsystem parameters. Possible development and
demonstration programs are suggested, the complex issue of closure
discussed, and the many applications and implications of replicating
systems are considered at length.
\end{quote}

Figure \ref{fig:nasa_spirit} below is reproduced from the NASA study and, among other things it
depicts “…In the lower left corner, \textit{a lunar
manufacturing facility rises from the surface of the Moon. Someday,
such a factory might replicate itself, or at least produce most of its
own components, so that the number of facilities could grow very
rapidly from a single seed” }


%% FIXME - add image
\migneafigure{MigneaNasaSpiritSpace.jpg}{NASA---The Spirit of Space Missions---created by Rick Guidice}{fig:nasa_spirit}

Among other things the project specifies:

\begin{itemize}
\item The seed of the lunar factory – transported from Earth – would
weigh 100 tons
\item Not all machinery could be built on the Moon; computer boards will
be brought up from earth as “vitamins”, since parts closure will not be
achieved
\item It was estimated that the project will be feasible in the
21\textsuperscript{st} century
\end{itemize}

Figure \ref{fig:lmf_parts_fabrication} below illustrates the depth the project reached in considering
fabrication facilities on the Moon.

%% FIXME - cite NASA study
\migneafigure{MigneaPartsFabrication}{LMF Parts Fabrication Sector: Operations (Fig 5.17 in the NASA Study)}{fig:lmf_parts_fabrication}

Being quite thorough the NASA study represents a realistic evaluation of
the extent, the problems and the difficulties that need to be addressed
by the design of a macro (kilometers scale in this case) self
replicator.

\subsection[The Self{}-Reproducing Interstellar Probe (REPRO) Study]{The
Self-Reproducing Interstellar Probe (REPRO) Study}

\hypertarget{RefHeading3156306210128}{}In 1980 Robert A. Freitas
publishes  “A Self-Reproducing Interstellar Probe” (REPRO) study  in
the \textit{Journal of the British Interplanetary Society}. Some of the
main goals of the project are summarized below.

\begin{itemize}
\item REPRO was a mammoth self-reproducing spacecraft to be built in
orbit around Jupiter.
\item REPRO was a vast and ambitious project, equipped with numerous
smaller probes for planetary exploration, but its key purpose was to
reproduce. Each REPRO probe would create an automated factory that
would build a new probe every 500 years. Probe by probe, star by star,
the galaxy would be explored 
\item The total fueled mass of REPRO was projected to be 10**10 Kg = 10
**7 tons = 10 million tons for a probe mass of 100,000 tons.
\item It takes 500 years for REPRO to create a replica of itself in the
relatively hospitable environment of a far-away planet
\item The estimated exploration time of the galaxy was 1– 10 million
years
\end{itemize}

\bibliographystyle{eandm}
\bibliography{MigneaLibrary}

