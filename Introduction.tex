\chapter{Introduction}

\section{Philosophy and Pragmatism, Science and Engineering}

When people think about engineering, they usually think about bridges and computers, math and science, and a thorough practicality.  Rarely are philosophy, theology, worldviews, or metaphysics considered as being anywhere close to connected with engineering.  However, the truth is that these ultimate things are thoroughly embedded within engineering, but they are rarely if ever reflected upon.  
The act of building relies on certain assumptions about the nature and limitations of the world, as well as the nature and limitations of the builders.  
The \textit{purpose} of a building or software program is just as important to its design as the construction materials or design patterns. 
These considerations are closer to philosophy or theology than they are math and science.  However, the reasoning involved in such considerations is rarely
formalized or even made explicit.

On the flip side of the coin, we tend to make engineering the ultimate test of knowledge.  In America, at least, something isn't really considered true unless you can build a better product with it.  This is the philosophical heritage of William James, the father of the pragmatic school of philosophy.  In pragmatism, the utlimate test of ideas is their ``cash value''---what you can \textit{do} with the ideas.  For pragmatists, the search for \textit{truth} for its own sake is somewhat misguided, because we will never know that something is the truth until we can apply it to real-world problems.  For better or worse, this tends to be the philosophy of most modern people.  Therefore, the ultimate test of knowledge is whether or not we can build something out of it.  Or, as I like to tell people, ``in America, we don't consider anything to be really true unless you can build a better phone out of it.''  It is this philosophy that makes \textit{testability} so fundamental to the practice of science.  \textit{Testability} is the cash value of scientific ideas.  Testability evaluates one or more scientific models based on their \textit{practical} consequences.  This is why quantum mechanics (which is testable) holds a much firmer place in science than string theory (which is not). 

Such an emphasis of practicality has caused modern humanity to all but abandon speculative pursuits such as philosophy.  
Some recent scientists, most notably Stephen Hawking and to a lesser extent Lawrence Krauss, have openly rejected philosophy.\citep{warman2011}\citep{andersen2012}  
However, such an attitude disregards the great contributions that such speculative endeavors have had on even the most practical concerns.  Modern science, in fact, is based on the rigorous and unremitting application of ideas which originated in philosophy.  The conservation laws in physics are really nothing except a practical expression of the philosophical idea of \foreign{ex nihilo nihil fit} (out of nothing, nothing comes), also called the ``principle of sufficient reason,'' which dates back to before Aristotle.\citep{psr2011}

Pragmatism itself is an outgrowth of philosophy.  Pragmatism often represents itself as being outside of speculative philosophy.  However, the very nature of pragmatism comes as the result of speculative reasoning.  It is the outgrowth of Liebniz's concept of ``the identity of indiscernables,'' which was first put forth in his \btitle{Discourse on Metaphysics}.\citep{ident2012}  The identity of indiscernables states that if two things are distinct from each other, there will be some property or group of properties which will also be distinct.  Pragmatism is merely the act of looking for properties that make two theories distinct from each other, or which makes reality identical or distinct from some theory.  If my theory is identical with reality, then the properties which hold in my theory should also hold in reality.  Testing is merely probing reality to determine, on the basis of the identity of indiscernables, whether or not the theory is identical with reality.

So, as it turns out, science, testability, and the fundamentals of physics all draw directly from speculative philosophy.  Rather than science or pragmatism getting rid of the need for speculative philosophy, they prove its importance.  What modern scientific thought has actually demonstrated is that sound philosophy, over time, generates rock-solid principles.  Principles that are so solid, that most practicing scientists simply assume their truth as self-evident, unaware that these ideas have had a history within philosophy, and are the result of centuries of reflection, debate, and development.  It is true that not all in philosophy is as sound or solid as these principles.  However, to reject philosophy just because newer ideas are not as solid as the more well-developed ones is simply to reject the general advance of knowledge because we have not achieved certainty yet.  Rather than dismissal, what we need is more active engagement and development.

This rejection of philosophy is evident not just in many modern scientists, but also in the treatment of higher ideals by the mainstream culture.  Everything that is not directly rooted in pragmatism is viewed as a mere matter of opinion.  Thus, discussions about society and social governance, rather than being rational discourses on the nature, purpose, and ideals of humanity, tend to degrade into ramblings about how each person has their own values.  There may be some trivial sense in which this is true, but when taken as a societal statement, it is as irrational as saying the principle of sufficient reason doesn't apply to me.  It may be true that we don't know or agree on the ideals of humanity, but this should not be taken as an excuse to reject them as targets of public discussion.

\section{Reintegrating Philosophy into Science and Engineering}

Science's great progress has been the result of the continual and unrelenting application of sound philosophy, such as the principle of sufficient reason and the identity of indiscernables, into all areas of inquiry.  It seems reasonable, then, that additional progress can be made by explicitly recognizing the link between these fields, and encouraging more cross-discipline dialog.  While some progress in this has been made, more is desperately needed.

In 2000, Baylor University held a conference called \textit{The Nature of Nature}.  Its goal was to bring scientists, philosophers, and other academic disciplines together to talk about the ultimate nature of reality.  Specifically, the question was whether \textit{naturalism} was a valid presumption in the pursuit of science.\citep{gordondembski2011}  What is naturalism?  Naturalism is the idea that all of reality is a self-contained physical system.  If naturalism is true, then any phenomena must be, at least in theory, describable by references to physics.  Therefore, any idea which is not reducible to physics should not be considered a valid explanation.  In such a view, for instance, \textit{design} might be a description, but it cannot be a cause.  A soul might be a useful fiction, but it cannot be a reality.  Free choice is an illusion.  

The \textit{Nature of Nature} conference included many of the very people who built many modern scientific fields, including Francis Crick (discoverer of DNA and the genetic code), Roger Penrose (whose contributions to physics is similar to Stephen Hawking's), Guillermo Gonzalez (who pioneered work on galactic habitability zones), and many other experts and professionals in science and philosophy.  While the conference did not come to any particular conclusion, it was successful in moving the question from the periphery to a more central question in the philosophy of science.  Plantinga's contribution to the conference, for instance, eventually culminated in his book \btitle{Where the Conflict Really Lies: Science, Religion, and Naturalism}, published by Oxford University Press.\citep{plantinga2011}\citep{plantinga2011b}

The conference eventually resulted in a book named after the conference, \btitle{The Nature of Nature}.\citep{natnat2011}  One thing, however, was markedly absent from the list of topics - any discussion of the practical consequences of any of the theories of reality offerred by the conference attendees.  In other words - lots of interesting ideas, but nothing concrete enough that would allow you to build a better phone.

Although I don't know if they were influenced by the \textit{Nature of Nature} conference, since the time of that conference there have been at least two conferences which dealt with the relationship between the nature of nature, science, and engineering.
The first was the Royal Academy of Engineering's \textit{Engineering and Metaphysics} seminar in 2007.  The focus of this conference was the relationship between ontology (philosophy of being) and process engineering.  % FIXME - more discussion here

In addition, a further conference in 2009 titled \textit{Parallels and Convergences} tackled a variety of questions focusing around the large-scale goals of engineering, including space exploration and transhumanism, and how it integrates with the purpose of humanity. %% FIXME - more discussion here

In 2011, a conference was planned to address two major areas of integration between engineering, philosophy, and theology.  The first was to examine how philosophical and theological ideas can be directly integrated into the practice of engineering.  The second was to investigate how the tools of engineering can be retrofitted to analyze philosophical and theological questions.  This resulted in the 2012 Conference on Engineering and Metaphysics, whose proceedings is contained in the present volume.

\section{The Engineering and Metaphysics 2012 Conference}

The papers in this volume, for the most part, follow the talks given in the Engineering and Metaphysics 2012 Conference.\footnote{If you want to see the original talks, they are all available online at the Blyth Institute website: \url{http://www.blythinstitute.org/eandm2012}}  We were pleased to have, between the participants, presenters, and authors, some overlap between our conference and the original \textit{Nature of Nature} conference which inspired it.  The authors come from a variety of backgrounds---academically, spiritually, and socially.  The subtitle of this book is ``an interdisciplinary investigation into order and design in nature and craft.''  Our group included philosophers, theologians, engineers, computer scientists, and liberal arts professors.  It was truly an interdisciplinary group.  

It was also, quite assuredly, an investigation.  We had no goal but to look deeper.  We went a number of different directions, as is evident from our table of contents.  We investigated nature from an engineering perspective, and engineering from various perspectives on nature.  We even investigated the investigation.  At the end, what
we found were not finished masterpieces but sprouts of new growth---the beginnings of new ways of analyzing problems in engineering, in philosophy, and in theology.  Some surely will be more successful in the long run than others, but each one benefits us by asking us to look at old questions in new and unfamiliar ways.

Can you ask theological questions mathematically?  Can nature be rigorously analyzed in terms of purpose as well as by matter in motion?  Can the role of the human spirit be integrated into science?  Can it be used to analyze engineering outputs?  Does it leave a distinctive mark on nature?  How does theology change the goals and processes of engineering?  Which of these questions are ill-posed and unworkable, and which have lasting value?

These questions are all ground-floor questions.  My hope is not that this volume be a finishing point for these types of investigations, but rather that it will inspire others to ask even better questions along the same lines.  My hope is for entirely new fields to emerge at the boundaries of theology, philosophy, science, and engineering, which will ask new questions, develop new methodologies, and learn not only new answers, but also entire new ways of understanding.  In short, my goal for the conference and this volume is not a final answer, but an initial inspiration.  I hope that the reader is challenged to reexamine the borders and boundaries of disciplines, and to think about the world in new and exciting ways.

\bibliographystyle{eandm}
\bibliography{IntroductionLibrary}
