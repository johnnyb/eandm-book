\eandmchapter{Introduction}{Introduction}{Jonathan Bartlett}{The Blyth Institute}
\setcounter{page}{1}
\pagenumbering{arabic}

\section{Philosophy and Pragmatism, Science and Engineering}

When engineering comes to mind, people usually think about bridges and computers, math and science---in other words, a practical application of technology to life.  Seldom do they connect engineering with philosophers and theologians. However, the truth is that these ultimate things are thoroughly embedded within engineering, even if they are rarely reflected upon.  
The act of construction relies on certain assumptions about the nature and limitations of the world as well as the nature and limitations of the builders.  
The \textit{purpose} of a building or software program is just as important to its structure as its materials or design. 
These considerations are closer to philosophy or theology than they are to math and science.  However, the reasoning involved in such considerations is hardly ever
formalized or even made explicit.

On the flip side of the coin, engineering is often used as the ultimate test of knowledge. 
In America, at least, a thing is not really considered true or authentic or of value unless someone can use it to build a better product, e.g., a better phone.  
This concept reflects the philosophical heritage of William James, the father of the pragmatic school of philosophy.  In pragmatism, the ultimate test of ideas is their ``cash value''---what people can \textit{do} with the ideas.  For pragmatists, the search for \textit{truth} for its own sake is somewhat misguided because truth cannot be apprehended until it is applied to real-world problems.  For better or worse, this tends to be the philosophical framework by which most modern people live.  Therefore, the ultimate test of knowledge is whether or not we can build something out of it.  This pragmatic philosophy makes \textit{testability}, the cash value of scientific ideas, so fundamental to the practice of science.  It evaluates one or more scientific models based on their \textit{practical} consequences.  This is why quantum mechanics (which is testable) holds a much firmer place in science than string theory (which is not). 

Such an emphasis on practicality has caused modern humanity to all but abandon speculative pursuits such as philosophy.  
Some recent scientists, most notably Stephen Hawking and to a lesser extent Lawrence Krauss, have openly rejected philosophy \citep{warman2011, andersen2012}.
However, such an attitude disregards the great contributions that such speculative endeavors have had on even the most practical concerns.  Modern science, in fact, is based on the rigorous and unremitting application of ideas which originated in philosophy.  The conservation laws in physics are really nothing except a practical expression of the philosophical idea of \foreign{ex nihilo nihil fit} (out of nothing, nothing comes), also called the ``principle of sufficient reason,'' which dates back to before Aristotle \citep{psr2011}.

Pragmatism itself is an outgrowth of philosophy.  Although pragmatism often represents itself as being outside of speculative philosophy,  its very nature arises out of speculative reasoning.  Pragmatism is the outgrowth of Liebniz's concept of ``the identity of indiscernibles,'' which was first put forth in his \btitle{Discourse on Metaphysics} \citep{ident2012}.  The identity of indiscernibles states that if two things are distinct from each other, there will be some property or group of properties which will also be distinct.  Pragmatism is merely the act of looking for properties that make two theories distinct from each other, or makes reality identical or distinct from some theory.  If a theory is identical with reality, then the properties which hold in the theory should also hold in reality.  Testing is merely probing reality to determine, on the basis of the identity of indiscernibles, whether or not the theory is identical with reality.

So, as it turns out, science, testability, and the fundamentals of physics all draw directly from speculative philosophy.  Rather than science or pragmatism getting rid of the need for speculative philosophy, they prove its importance.  What modern scientific thought has actually demonstrated is that sound philosophy, over time, generates rock-solid principles, principles that are so solid that most practicing scientists simply assume their truth as self-evident, often unaware that these ideas have had a history within philosophy and are the result of centuries of reflection, debate, and development.  It is true that not everything in philosophy is as sound or solid as these principles.  However, to reject philosophy just because newer ideas have not achieved the status as the more well-developed ones is simply to reject the general advance of knowledge because philosophers have not achieved certainty yet.  Rather than dismissal, what is needed is more active engagement and development.

Not only do many modern scientists reject philosophy, but the manner in which these higher ideals are treated by the mainstream culture demonstrates the belief that everything not directly rooted in pragmatism is a mere matter of opinion.  
Thus, in discussions about society and social governance, any attempt to include reasonings based on the nature, purpose, and ideals of humanity, tend to be met with ramblings about how ideas such as these are private, personal matters, and not suitable for public inquiry and discourse.  Rejecting such philosophical reasoning in public discourse and deeming such knowledge as a matter of personal preference is as irrational as saying, ``The principle of sufficient reason doesn't apply to me.''  It may be true that people don't know or agree on the ideals of humanity, however, this should not be taken as an excuse to reject them as targets of public discussion, but rather a reason to explore them further and deeper.

\section{Reintegrating Philosophy into Science and Engineering}

As has been previously demonstrated, science's great progress has been the result of the continual and unrelenting application of sound philosophy, such as the principle of sufficient reason and the identity of indiscernibles, into all areas of inquiry.  It seems reasonable, then, that additional progress can be made by explicitly recognizing the link between these fields and encouraging more cross-disciplinary dialog.  While some progress in this has been made, more is desperately needed.

In 2000, Baylor University held a conference called \textit{The Nature of Nature}.  Its goal was to bring scientists, philosophers, and other academic disciplines together to talk about the ultimate nature of reality.  Specifically, the question was whether \textit{naturalism}, the idea that all of reality is a self-contained physical system, was a valid presupposition in the pursuit of science \citep{gordondembski2011}.
If naturalism is true, then any phenomena must be, at least in theory, describable by references to physics.  Therefore, any idea not reducible to physics should not be considered a valid explanation.  In such a view, for instance, \textit{design} might be a description, but it cannot be a cause.  A soul might be a useful fiction, but it cannot be a reality.  Free choice is an illusion.  

The \textit{Nature of Nature} conference included the very people who built many modern scientific fields, including Francis Crick (who discovered DNA and the genetic code), Roger Penrose (whose contributions to physics are similar to Stephen Hawking's), Guillermo Gonzalez (who pioneered work on galactic habitability zones), and many other experts and professionals in science and philosophy.  While the conference did not come to any particular conclusion, it was successful in moving the question concerning the ultimate nature of reality from the periphery to a more central position.  Alvin Plantinga's contribution to the conference, for instance, eventually culminated in his book, \btitle{Where the Conflict Really Lies: Science, Religion, and Naturalism}, published by Oxford University Press \citep{plantinga2011, plantinga2011b}.

The conference eventually resulted in a book named after the conference, \btitle{The Nature of Nature} \citep{natnat2011}.  One thing, however, was markedly absent from the list of topics---any discussion of the practical consequences of any of the theories of reality offered by the conference attendees. In other words, many interesting ideas were put forth, but nothing concrete enough to result in the building of a better phone.

Since the \textit{Nature of Nature} conference, at least two conferences which dealt with the relationship between the nature of nature, science, and engineering have been held.
The first was the Royal Academy of Engineering's \textit{Engineering and Metaphysics} seminar in 2007.  The focus of this conference was the relationship between ontology (philosophy of being) and process engineering (the act of doing).  In addition,  a conference in 2009 titled \textit{Parallels and Convergences} at Claremont Graduate University tackled a variety of questions focused around the large-scale goals of engineering---including space exploration and transhumanism---and their integration with the purposes of humanity.  

In 2011, a conference was planned to address two major areas of integration between engineering, philosophy, and theology.  The first was to examine how philosophical and theological ideas can be directly integrated into the practice of engineering.  The second was to investigate how the tools of engineering can be retrofitted to analyze philosophical and theological questions.  This resulted in the 2012 Conference on Engineering and Metaphysics, whose proceedings is contained in this present volume.

\section{The Engineering and Metaphysics 2012 Conference}

The papers in this volume, for the most part, follow the talks given in the Engineering and Metaphysics 2012 Conference.\footnote{If you want to see the original talks, they are all available online at the Blyth Institute website: \url{http://www.blythinstitute.org/eandm2012}}  Between the participants, presenters, and authors, there exists some overlap between this conference and the original \textit{Nature of Nature} conference which inspired it.  The authors come from a variety of backgrounds---academically, spiritually, and socially.  Since the participants included philosophers, theologians, engineers, computer scientists, and liberal arts professors---a truly interdisciplinary group---the book has been subtitled ``an interdisciplinary investigation of order and design in nature and craft.''  

The conference was indeed focused on investigation, an intense desire to look deeper, to search in a number of different directions as evidenced in the Table of Contents.  Nature was investigated from an engineering perspective, and engineering from various perspectives on nature.  Even the investigation was investigated.  At the conclusion of the conference, what
was produced were not finished masterpieces but new ways of holding the brush and painting the canvas, analyzing problems in engineering, philosophy, and theology through new lenses.  Some ideas and approaches surely will be more successful in the long run than others, but each one benefits the discussion by looking at old questions in new and unfamiliar ways.

Some questions asked include the following: Can theological questions be asked mathematically?  Can nature be rigorously analyzed in terms of purpose as well as by matter in motion?  Can the human spirit be integrated into science?  Can it be used to analyze engineering outputs?  Does it leave a distinctive mark on nature?  How does theology change the goals and processes of engineering?  Which of these questions are ill-posed and unworkable, and which have lasting value?

These questions are all foundational questions.  This volume is not designed to be a finishing point for these types of investigations, but rather to serve as an inspiration to others to ask even better questions along the same lines.  The hope is that entirely new fields will emerge at the boundaries of theology, philosophy, science, and engineering, which will ask new questions, develop new methodologies, and learn not only new answers, but also entire new ways of understanding.  In short, the goal for the conference and this volume is not a final answer, but an initial inspiration.  After perusing these proceedings, the reader will be challenged to reexamine the borders and boundaries of disciplines, and to think about the world in new and exciting ways.

\section{Articles in this Volume}

The papers in this volume are divided into four parts---Engineering, Philosophy, and Worldview; Architecture and the Ultimate; Software Engineering and Human Agency; and The Engineering of Life.  Below is a short preview of each paper and its importance.  While they
approach very different subjects from very different perspectives, each one investigates the way in which usable knowledge can be increased by looking beyond the strict functional materialism which often dominates engineering discussions.

\subsection*{Reversible Universe: Implications of Affordance-based Reverse Engineering of Complex Natural Systems}

This volume begins with a paper examining how scientists look at nature and suggests that reverse engineering is a fruitful methodology for natural investigations.  It suggests that \emph{purpose} is just as much of a discoverable fact of nature as is mechanism and suggests a methodology based on affordance-based reverse engineering for discovering nature's purpose as well as nature's mechanisms.  

\subsection*{The Independence and Proper Roles of Engineering and Metaphysics in Support of an Integrated Understanding of God's Creation}

The next paper analyzes the boundaries of various disciplines and shows the kinds of problems that arise from misunderstanding the proper roles and boundaries of various disciplines, including mathematics, science, engineering, and philosophy.  It looks at how various spheres of knowledge do and do not interrelate, with the goal of producing a symphonic arrangement of knowledge and action.

\subsection*{Truth, Beauty, and the Reflection of God: John Ruskin's \textit{Seven Lamps of Architecture} and \textit{The Stones of Venice} as Palimpsests for Contemporary Architecture}

Much of modern engineering is functional.  If it works, then what more is there to do?  
In this paper, additional foundational considerations besides function are suggested
for the practice of architecture, including moral, ethical, philosophical, and religious
principles.  Using John Ruskin as a plumbline, the paper provides examples of modern 
architecture which embody these principles and suggests ways in which these principles
can be incorporated into future architectural projects.

\subsection*{Using Turing Oracles in Cognitive Models of Problem-Solving}

Problem-solving plays a fundamental role in engineering, as one of the main tasks of an engineer
is to generate creative solutions to technical problems.  As such, this paper examines 
the question of whether humans are entirely physical or if they have
a spiritual component and the impact that this has on cognitive models of problem-solving.  The
paper suggests Alan Turing's \emph{oracle} concept as a way of integrating non-mechanistic human
abilities into models of human insight.

\subsection*{Calculating Software Complexity Using the Halting Problem}

Building on the previous paper, this paper gives a practical application of non-mechanistic
models of problem-solving by developing a software complexity metric which is based on 
supra-computational abilities of humans when solving problems requiring insight.  It uses
the computational insolubility of the halting problem to find and measure the amount of 
insight required to understand a computer program.

\subsection*{Algorithmic Specified Complexity}

The next paper in the volume considers the question of 
what it means for something to be engineered.
Is there any property of an \emph{engineered} system which separates
it from things which are not engineered?  This paper makes a technical examination of 
algorithmic information theory to derive a metric that the authors term
\emph{Algorithmic Specified Complexity}, which uses compressibility and context to 
measure the likelihood that a particular sequence is the result of intentional engineering rather
than happenstance.

\subsection*{Complex Specified Information (CSI) Collecting}

If one assumes that humans are non-mechanical and are capable of supra-computational abilities, then
it may be possible to reliably harness this ability in certain applications.  This paper looks at
how this might be measured, tested, and harnessed programmatically.  The paper includes an experimental
design which, though it was not successful in this attempt, can provide a starting point for future
experiments and investigations.

\subsection*{Developing Insights into the Design of the Simplest Self-Replicator and its Complexity}

This final paper is an extended consideration of the minimal requirements for true self-replication, divided
into three parts.  Part 1 considers the abstract design required to allow self-replication.  It analyzes
what sorts of processes, components, and information is needed for any self-replication to occur.  
Part 2 analyzes the potential physical implementation possibilities and the various design considerations
when choosing implementation materials.  Part 3 compares the minimal artificial self-replicator to 
the self-replicators found in nature---namely cell biology.  This part examines possible origin-of-life
scenarios based on the analysis of the design requirements of self-replication.

% Uncomment for extended references on introduction
%
%\nocite{halsmer}
%\nocite{sich}
%\nocite{hall}
%\nocite{bartlett1}
%\nocite{bartlett2}
%\nocite{ewert}
%\nocite{holloway}
%\nocite{mignea1}
%\nocite{mignea2}
%\nocite{mignea3}

\eandmbibliography{IntroductionLibrary}
