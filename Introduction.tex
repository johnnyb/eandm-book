\chapter*{Introduction}
\addcontentsline{toc}{chapter}{Introduction}
\markboth{INTRODUCTION}{}

\section{Background}
American epistemology, as generally practiced by normal americans, for better or worse, is based on the philosophy of pragmatism.  William James defined pragmatism as the evaluating the "cash value" of ideas.  In other words, it isn't enough for something to be true, or to know that it is true, but it is only worthwhile to find a truth which helps us accomplish something.  Even more than that - the only way we can know that a truth really is a truth is if we can use that truth in a concrete, practical sense.

With such an epistemology, science becomes rather speculative by comparison.  Science can develop a model of an atom, but it only becomes valid knowledge when someone can take that model and build a better energy source.  Science itself has adopted a very pragmatic approach to truth as well.  Testability is one of the cornerstones of scientific inquiry.  But what is testability except the pragmatic ability to discern what \emph{practical} difference two models have?

Such an emphasis of practicality has caused modern humanity to all but abandon speculative pursuits such as philosophy.  However, such an attitude disregards the great contributions that such speculative endeavors have had on even the most practical concerns.  Some recent scientists, most notably Lawrence Krauss and Stephen Hawking, have openly rejected philosophy, all the while ignoring the fact that modern science is based on the rigorous application of ideas which were originated in philosophy.  What are the conservation laws except the practical expression of the philosophical concept of \foreign{ex nihilo nihil fit}\footnote{out of nothing, nothing comes}?  What is pragmatism except the expression of the ontological concept of the identity of indiscernibles?

It turns out that, far from excluding philosophy, modern scientific thought has showed that sound philosophy generates rock-solid principles which are essential to its practice.  If scientists disclaim philosophy, it is simply because the philosophical facts they work with are so dependable that the fact that they are not self-evident has been lost.  These philosophical principles all had a history, and had to undergo considerable development and debate to get into their current form.  To reject philosophy just because newer ideas are not as solid as the more well-developed ones is simply to reject the general advance of knowledge because we have not achieved certainty yet.

These scientists who reject philosophy are far from alone.  The entire culture has stopped seeking ultimate things and instead pursues practical things.  Modern America only views ideas as true if it allows them to build a better phone.  If it doesn't, it is considered a superfluous idea.  Such a mode of thought, however, stretches even pragmatism to a breaking point, as it cuts down reality to the size of material objects, and ignores the larger encompassment of reality.

In 2000, Baylor University held a conference called \btitle{The Nature of Nature}.  Its goal was to bring scientists, philosophers, and other academic disciplines together to talk about the ultimate nature of reality.  The conference included the very people who built modern science, including Francis Crick (discoverer of DNA and the genetic code), Roger Penrose (whose contributions to physics is similar to Stephen Hawking's), Alvin Plantinga (one of the major thinkers in epistemology), Guillermo Gonzalez (who pioneered work on galactic habitability zones), and many others.  

While there was no final agreement between the participants on the ultimate nature of reality, the conference itself sparked and highlighted a variety of research projects in which the question of reality's nature is brought to the center.  Plantinga, for instance, showed that a necessary requirement for the rationality of science is a picture of reality that is larger than what naturalism usually gives.  This research eventually culminated in his book \btitle{Where the Conflict Really Lies: Science, Religion, and Naturalism}, published by Oxford University Press.  

The conference resulted in a book named after the conference, \btitle{The Nature of Nature}.  However, one thing was markedly absent from the list of topics - any discussion of the practical consequences of any of the theories of reality offerred by the conference attendees.  In other words - lots of interesting ideas, but nothing that would allow you to build a better phone.

\section{Previous Attempts}


\section{Our Present Work}


%% It is often said that we live in a scientific culture, but that is not altogether true.  Our culture, while it gives lip service to science, is actually more culturally indebted to engineering than to science.  Few people know anything about the latest in chemical, physical, or biological theory, but \emph{everyone} knows the features of the latest mobile gadget. 
%% In fact, much of American epistemology is built on engineering. 
