\title{About the Authors}
\maketitle

\setlength{\intextsep}{0pt}%
% environment to standardize the bios
\newenvironment{authorbio}[3]{
	\subsection*{#1}

	\ifstrequal{#2}{}{}  % skip the image if one wasn't provided
	{
			\includegraphics[width=1.5in]{#2} %
			\ifstrequal{#3}{}{%
			}{\vspace{#3}%
			}%
	}

	}{ 
	
		\mbox{}
}

\unnumberedsection{Editors and Organizers}{Editors and Organizers}
\sectionmark{Editors and Organizers}

% Adapted from - http://www.oru.edu/academics/college_of_science_and_engineering/dominic_halsmer.php
\begin{authorbio}{Dominic Halsmer}{halsmer_photo.jpg}{}
Dominic Halsmer's love for science and engineering began from the time he was a child playing at the small airport his father and uncles owned.  He and his siblings would work on projects to defy gravity, once even successfully building a kite large enough to lift a person off the ground using an old parachute and rope.  This natural curiosity eventually blossomed into the work he does today.  

Dominic went on to earn his B.S. and M.S. degrees in Aeronautical and Astronautical Engineering from Purdue University, and his Ph.D. in Mechanical Engineering from UCLA in 1992.  College was a time of spiritual searching, but ultimately he returned to his faith.  ``The love and Christ-like example of my parents made the difference,'' he said.  He also notes that the evidence from science for the truth of the Christian worldview also helped draw him back to his faith.  Today, he has a heart for reaching out with the gospel to skeptical scientists and engineers.

Dominic joined the Engineering and Physics Department at Oral Roberts University shortly after graduating from UCLA.  His interest in flight continues, including participation in the NASA/ASEE Summer Faculty Fellowship Program at NASA Goddard Space Flight Center, and working with undergraduates to test the stability of spinning aircraft under thrust.  His current research focuses on studying how the universe is engineered to reveal the glory of God and accomplish His purposes.  In 2013 he earned his M.A. degree in Biblical Literature from Oral Roberts University.  Dominic is married to his high school sweetheart Kate, and they are the parents of four children.  From 2007 to 2012, he served as Dean of the College of Science and Engineering at ORU, and he now directs the ORU Center for Faith and Learning.

% FROM ORU Website
%Third from the youngest of 13 children, Dr. Dominic Halsmer grew up in rural Indiana where his family's large acreage provided ample room for a swimming pool, ball fields and trampoline. His godly parents and large extended family that lived near by gave him a strong foundation of faith, athleticism and adventure that endures today.
%
%Dominic's father and uncle owned and operated a small airport and the Halsmer kids spent many happy hours trying to defy gravity at the airfield. Once, under older brother David's direction, they developed a kite large enough to lift a person off the ground using an old parachute and some rope. Dominic's natural interest in physics and engineering blossomed in this fertile ground that fostered a child's exploration and curiosity.
%
%Dominic earned his B.S. and M.S. degrees in Aeronautical and Astronautical Engineering from Purdue University. College was a time of spiritual searching, but ultimately he returned to his faith. "The love and Christ-like example of my parents made the difference," he said. He also notes that the evidence from science for the truth of the Christian worldview also helped draw him back to his faith. Today, he has a heart for reaching out with the gospel to skeptical scientists and engineers.
%
%In 1992, Dominic earned his Ph.D. in Mechanical Engineering from UCLA. That fall, he joined the Engineering and Physics Department at Oral Roberts University. In 1996 and 1997, he participated in the NASA/ASEE Summer Faculty Fellowship Program at NASA Goddard Space Flight Center. His research activities have involved undergraduate engineering students in developing an apparatus to test the stability of spinning spacecraft under thrust. Currently, he is studying how the universe is engineered to reveal the glory of God and accomplish His purposes.
%
%Today, Dominic is a virtual poster boy for the ORU Whole Person concept. An accomplished athlete, he often earns a top-three slot in Tulsa area races and in 2007 placed third in the 3000 meter steeplechase at the National Master's Outdoor Track and Field Championships in Maine. But first and foremost, he is a committed Christian husband and father in addition to being an accomplished engineer with a thirst for knowledge and a love of teaching.  Dominic can often be seen running with students during lunch time.
%
%Dominic is married to his high school sweetheart Kate, and they are the parents of four children ages 14 to 22 years.
%
%B.S. and M.S. in Aeronautical and Astronautical engineering from Purdue University
%Ph.D. in Mechanical Engineering from UCLA
\end{authorbio}

% Adapted from http://webapps.oru.edu/new_php/academics/faculty_profile.php?id=99&k=
\newpage
\begin{authorbio}{Mark R. Hall}{hall_photo.jpg}{}
Mark Hall is the Dean of the College of Arts and Cultural Studies at Oral Roberts University, where he currently teaches courses on the intersection of science and the humanities, including \textit{Science and the Imagination} and \textit{C. S. Lewis and the Inklings}, as well as other courses in literature.  He was recently co-organizer of the "When Worlds Collide" conference on science and science fiction featuring plenary speakers Paul Davies and Joan Slonczewski.

Mark Hall’s extensive list of degrees include a B.S.E. in English Education, an M.S.E. in English, and a Specialist degree in Higher Education with an emphasis in English, all from the University of Central Missouri.  He has also completed three Masters' degrees from Oral Roberts University including an M.A. in Biblical Literature, an M.A. in Theological and Historical Studies, and an M.A. in  Biblical Literature (Advanced Languages concentration).  Mark received his Ph.D. in English from the University of Tulsa.  

Mark is an ordained minister and a former church pastor.  He has been married for over 22 years to his wife, Rachel, who is the Music Director at Jenks United Methodist Church. Dr. Hall and his wife have two children, Jonathan and Kathryne.   Mark is active in his community through politics and the theater and is an avid reader, with one of his favorite authors being C. S. Lewis.
\end{authorbio}

\begin{authorbio}{Jonathan Bartlett}{bartlett_photo2.jpg}{}
Jonathan Bartlett is the Director of The \mbox{Blyth} \mbox{Institute} in Tulsa, Oklahoma.  The \mbox{Blyth} \mbox{Institute} is a non-profit research and education organization focusing on pioneering non-reductionistic approaches to biology.  Jonathan's research focuses on the origin of novelty---both the origin of biological novelty in adaptation as well as the origin of insight in the human creative process. 

Jonathan's other roles include managing a team of software developers as the Director of Technology at New Medio, tutoring homeschool students in chemistry and calculus at Classical Conversations, and being part of the Classical Conversations Writer's Circle, where he has a monthly column discussing issues of science, faith, and education.  Jonathan is the author of the book \textit{Programming from the Ground Up}, which has been used in Universities from Princeton University to Oklahoma State for teaching undergraduate assembly language.

Jonathan received a B.S. in computer science and a B.A. in religion from Oklahoma Baptist University, and an M.T.S. from Phillips Theological Seminary.  Jonathan and his wife, Christa, have had five boys---three living and two deceased.  Jonathan spends his free time practicing Taekwondo with his boys, tending to his garden, and exploring bookstores with his wife.
\end{authorbio}

\unnumberedsection{Primary Authors}{Primary Authors}
\sectionmark{Primary Authors}

\begin{authorbio}{Alexander Sich}{SichPhoto.jpg}{}
Alexander Sich is Associate Professor of Physics at Franciscan University of Steubenville. He holds a B.S. in Nuclear Engineering from Rensselaer Polytechnic Institute (with a minor in physics), an M.A. in Soviet Studies from Harvard University, an M.A. in Philosophy from Holy Apostles College \& Seminary, and a Ph.D. in Nuclear Engineering from MIT.  Professor Sich conducted his Ph.D. research in the Ukraine at the Chernobyl site and worked in the former Soviet Union in the nuclear safety and nonproliferation effort regarding weapons of mass destruction for over thirteen years. In addition to technical articles, Professor Sich has published opinion pieces on nuclear safety (including the Iranian Bushehr issue) in \emph{The Bulletin of the Atomic Scientists}, \emph{The Boston Globe}, The \emph{Wall Street Journal}, \emph{The Diplomat}, and \emph{Newsday}. Professor Sich is married with seven children, speaks near native fluent Ukrainian and fluent Russian. His current research interests lie primarily in the philosophy of nature and in teaching.
\end{authorbio}

%% Received via email
\begin{authorbio}{Arminius Mignea}{arminius_photo.jpg}{}
Arminius Mignea received his Engineer Diploma from the Polytechnic University of Bucharest, School of Computers. He worked as a software engineer and researcher at the Institute for Computing Technique for 14 years. Employed in the operating systems laboratory, he wrote system software in macro assembly language, Pascal, and C, and published some 12 research papers alone or as part of the team. After immigrating to the United States and settling in California's Silicon Valley, he worked for numerous technology startups developing both server and front-end software for network, enterprise, and security management products. More recently he was involved in the specification, architecture design, and development of software systems for enterprise network traffic monitoring, software build management, and TV server systems. About ten years ago Arminius started being interested in the designs, architectures, and machinery of the biological systems. Arminius is fascinated with the beauty, intricacies, coordination, and incredible levels of organization in living things, and marvels at the skills of their architect.
\end{authorbio}

\begin{authorbio}{Eric Holloway}{holloway_photo.jpg}{}
Mr. Holloway is currently an officer in the Air Force and has been on
active duty for eight years (including a deployment to Afghanistan in
2010). He has a B.S. from Biola and an M.S. from the Air Force Institute
of Technology, both in computer science. The M.S. was focused on Artificial Intelligence
with an emphasis on evolutionary algorithms and was funded by an \$80k
grant from the Air Force Research Lab. The work was published in
two conference proceedings, IEEE and ACM.
\end{authorbio}

\begin{authorbio}{Winston Ewert}{ewert_photo.jpg}{}
Winston Ewert hails from Canada where he earned a Bachelor's degree in Computer Science from Trinity Western University. He continued his graduate career at Baylor University where he earned a master's degree and is currently a Ph.D. candidate. At Baylor he is currently working for the Evolutionary Informatics Lab, with Robert J. Marks II and William Dembski, dedicated to understanding the role of information in evolution. He has a number of publications in the areas of search, conservation of information, artificial life, swarm intelligence, and evolutionary modelling. 
\end{authorbio}

\unnumberedsection{Additional Authors}{Additional Authors}
\sectionmark{Additional Authors}
% Facebook page - https://www.facebook.com/tyler.todd.315/about
% dynomite212005@oru.edu
% todd.tyler.n@gmail.com
\begin{authorbio}{Tyler Todd}{}{}
Tyler Todd graduated from Oral Roberts University in 2009 with a degree in Engineering Physics.  He currently works for Valmont Industries as a Manufacturing Engineer.
\end{authorbio}

% LI - http://www.linkedin.com/profile/view?id=23238440&authType=NAME_SEARCH&authToken=SbS5&locale=en_US&srchid=200467151381097959048&srchindex=5&srchtotal=15&trk=vsrp_people_res_name&trkInfo=VSRPsearchId%3A200467151381097959048%2CVSRPtargetId%3A23238440%2CVSRPcmpt%3Aprimary
% nathaniel.p.roman@boeing.com
\begin{authorbio}{Nate Roman}{}{}
Nate Roman graduated from Oral Roberts University in 2009 with a B.S. in Engineering Physics.  He currently works for Boeing as an Electrophysicist.
\end{authorbio}

\begin{authorbio}{Robert J. Marks II}{marks_photo.jpg}{1pt}
Robert J. Marks II is currently the Distinguished Professor of Electrical and Computer Engineering 
at Baylor University, Waco, TX. He is a Fellow of both IEEE and the Optical Society
of America. He served for 17 years as the Faculty Advisor with the Campus Crusade for Christ,
University of Washington chapter. His consulting activities include Microsoft Corporation, Pacific
Gas \& Electric, and Boeing Computer Services. Eleven of his papers have been republished in collections 
of seminal works. He is the author of \textit{Introduction to Shannon Sampling and Interpolation
Theory} (Springer-Verlag), \textit{Handbook of Fourier Analysis and Its Applications} (Oxford University
Press) and is a co-author of \textit{Neural Smithing} (MIT Press). His research has been funded by organizations 
such as the National Science Foundation, General Electric, Southern California Edison,
Electric Power Research Institute, the Air Force Office of Scientific Research, the Office of Naval
Research, the Whitaker Foundation, Boeing Defense, the National Institutes of Health, The Jet
Propulsion Laboratory, the Army Research Office, and the National Aeronautics and Space Administration 
(NASA). 
His Erd\"{o}s-Bacon
number is five.
%He is a former Editor-in-Chief of the IEEE TRANSACTIONS ON NEURAL
%NETWORKS \& LEARNING SYSTEMS. He was the recipient of numerous professional awards,
%including a NASA Tech Brief Award and a Best Paper Award from the American Brachytherapy
%Society for prostate-cancer research. He was the recipient of the IEEE Outstanding Branch Councilor 
%Award, the IEEE Centennial Medal, the IEEE Computational Intelligence Society Meritorious
%Service Award, the IEEE Circuits and Systems Society Golden Jubilee Award, and, for 2007, the
%IEEE CIS Chapter of the IEEE Dallas Section Volunteer of the Year Award. He was named a
%Distinguished Young Alumnus of the Rose-Hulman Institute of Technology and is an inductee into
%the Texas Tech Electrical Engineering Academy. He was the recipient of the Banned Item of the
%Year Award from the Discovery Institute and a recognition crystal from the International Joint
%Conference on Neural Networks for contributions to the field of neural networks. 
\end{authorbio}

% Taken from http://geweckespot.com
% rachelle.gewecke@gmail.com
\begin{authorbio}{Rachelle Gewecke}{rgewecke_photo.jpg}{}
Rachelle was born and raised in rural Iowa and completed her undergraduate studies in psychology at Oral Roberts University (ORU).  During her time at ORU, Rachelle participated in a research assistantship with Dr. Dominic Halsmer, an opportunity that produced conference presentations and ultimately a paper examining the concept of reverse engineering nature to discover intelligent design.  Rachelle currently resides in Spirit Lake, IA, with her husband, Michael, and daughter, Emily.
\end{authorbio}


% Taken from http://geweckespot.com
% michaelgwke@gmail.com
\newpage
\begin{authorbio}{Michael Gewecke}{mgewecke_photo.jpg}{}
Michael Gewecke earned a Bachelor of Arts at Oral Roberts University, a private liberal arts college in Tulsa, OK where he studied Theological/Historical Studies.  Most recently he received a Master of Divinity degree from Princeton Theological Seminary in Princeton, NJ.  In addition to his academic theological study, Michael has been a technology consultant and online website designer for the last five years.  Most recently, Michael served as the CEO of Worship Times, a website design and hosting company that is dedicated to serving small non-profit religious organizations across the United States.  These integrative experiences of theology and technology lend towards Michael's interest in the cross section between science and religion with particular emphasis upon their mutuality.   
\end{authorbio}

% Facebook page - https://www.facebook.com/jessica.fitzgerald.58
% Adapted from - http://www.linkedin.com/pub/jessica-fitzgerald/61/39/ab8
% concita1092@oru.edu
\begin{authorbio}{Jessica Fitzgerald}{fitzgerald_photo.jpg}{}
Jessica Fitzgerald is a student at Oral Roberts University pursuing a Bachelor of Science Degree in Engineering Physics with a minor in Mathematics. She plans to attend graduate school for physics or engineering, with a particular interest in nanoscience. Her strengths include data analysis, computation, and attention to detail.
\end{authorbio}

~

\begin{authorbio}{William A. Dembski}{dembski_photo.jpg}{1pt}
William A. Dembski received the B.A. degree in psychology, the M.S. degree in statistics, the
Ph.D. degree in philosophy, and the Ph.D. degree in mathematics in 1988 from the University of
Chicago, Chicago, IL, and the M.Div. degree from Princeton Theological Seminary, Princeton,
NJ, in 1996. He was an Associate Research Professor with the Conceptual Foundations of Science, 
Baylor University, Waco, TX, where he also headed the first Intelligent Design think-tank
at a major research university: The Michael Polanyi Center. He was the Carl F. H. Henry Professor 
in theology and science with The Southern Baptist Theological Seminary, Louisville, KY,
where he founded its Center for Theology and Science. He has taught at Northwestern University, 
Evanston, IL; the University of Notre Dame, Notre Dame, IN; and the University of Dallas,
Irving, TX. He has done postdoctoral work in mathematics with the Massachusetts Institute of
Technology, Cambridge, in physics with the University of Chicago, and in computer science with
Princeton University, Princeton, NJ.  %He is a Mathematician and Philosopher. 
He is currently a
Research Professor in philosophy with the Department of Philosophy, Southwestern Baptist Theological 
Seminary, Fort Worth, TX. He is currently also a Senior Fellow with the Center for Science
and Culture, Discovery Institute, Seattle, WA. He has held National Science Foundation graduate
and postdoctoral fellowships. He has published articles in mathematics, philosophy, and theology
journals and is the author/editor of more than a dozen books. 
%In The Design Inference: Eliminating 
%Chance Through Small Probabilities (Cambridge University Press, 1998), he examines the
%design argument in a post-Darwinian context and analyzes the connections linking chance, proba-
%bility, and intelligent causation. The sequel to The Design Inference (Rowman \& Littlefield, 2002)
%critiques Darwinian and other naturalistic accounts of evolution. It is titled No Free Lunch: Why
%Specified Complexity Cannot Be Purchased Without Intelligence. He has edited several influential
%anthologies, including Uncommon Dissent: Intellectuals Who Find Darwinism Unconvincing (ISI,
%2004) and Debating Design: From Darwin to DNA (Cambridge University Press, 2004, coedited
%with M. Ruse). His newest book, coauthored with J.Wells, is titled The Design of Life: Discovering
%Signs of Intelligence in Biological Systems (Foundation for Thought and Ethics, 2007). His area
%of interest in intelligent design has grown in the wider culture; he has assumed the role of public
%intellectual. In addition to lecturing around the world at colleges and universities, he is frequently
%interviewed on the radio and television. His work has been cited in numerous newspaper and
%magazine articles, including three front-page stories in The New York Times as well as the August
%15, 2005 Time Magazine cover story on intelligent design. He has appeared on the BBC, NPR
%(Diane Rehm, etc.), PBS (Inside the Law with Jack Ford and Uncommon Knowledge with Peter
%Robinson), CSPAN2, CNN, Fox News, ABC Nightline, and the Daily Show with Jon Stewart.
\end{authorbio}


