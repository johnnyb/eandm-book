\chapter{About the Authors}

\section*{Alexander Sich}

Alexander Sich is Associate Professor of Physics at Franciscan University of Steubenville. He holds a B.S. in Nuclear Engineering from Rensselaer Polytechnic Institute (with a minor in physics), an M.A. in Soviet Studies from Harvard University, an M.A. in Philosophy from Holy Apostles College \& Seminary, and a Ph.D. in Nuclear Engineering from MIT. Professor Sich conducted his Ph.D. research in Ukraine at the Chernobyl site, and worked in the former Soviet Union in nuclear safety and nonproliferation effort regarding weapons of mass destruction for over thirteen years. In addition to technical articles, Professor Sich has published opinion pieces on nuclear safety (including the Iranian Bushehr issue) in \emph{The Bulletin of the Atomic Scientists}, \emph{The Boston Globe}, The \emph{Wall Street Journal}, \emph{The Diplomat}, and \emph{Newsday}. Professor Sich is married with seven children, speaks near native fluent Ukrainian and fluent Russian. His current research interests lie primarily in the philosophy of nature, and in teaching.

\biosection{Eric Holloway}{holloway_photo.jpg}

Mr. Holloway is currently an officer in the Air Force and has been on
active duty for eight years (including a deployment to Afghanistan in
2010). He has a BSc from Biola and an MSc from the Air Force Institute
of Technology, both in computer science. The MSc was focused on AI
with an emphasis on evolutionary algorithms and was funded by an \$80k
grant from the Air Force Research Lab. The work was was published in
two conference proceedings, IEEE and ACM.

\biosection{Winston Ewert}{ewert_photo.jpg}

Winston Ewert hails from Canada where he earned a Bachelor's degree in Computer Science from Trinity Western University. He continued his graduate career at Baylor University where he earned a master's degree and is currently a Ph.D. candidate. While at Baylor he is working for the Evolutionary Informatics Lab, with Robert J. Marks II and William Dembski, dedicated to understanding the role of information in evolution. He has a number of publications in the areas of search, conservation of information, artificial life, swarm intelligence, and evolutionary modelling. 

\biosection{William A. Dembski}{dembski_photo.jpg}

William A. Dembski received the B.A. degree in psychology, the M.S. degree in statistics, the
Ph.D. degree in philosophy, and the Ph.D. degree in mathematics in 1988 from the University of
Chicago, Chicago, IL, and the M.Div. degree from Princeton Theological Seminary, Princeton,
NJ, in 1996. He was an Associate Research Professor with the Conceptual Foundations of Science, 
Baylor University, Waco, TX, where he also headed the First intelligent design think-tank
at a major research university: The Michael Polanyi Center. He was the Carl F. H. Henry Professor 
in theology and science with The Southern Baptist Theological Seminary, Louisville, KY,
where he founded its Center for Theology and Science. He has taught at Northwestern University, 
Evanston, IL, the University of Notre Dame, Notre Dame, IN, and the University of Dallas,
Irving, TX. He has done postdoctoral work in mathematics with the Massachusetts Institute of
Technology, Cambridge, in physics with the University of Chicago, and in computer science with
Princeton University, Princeton, NJ. He is a Mathematician and Philosopher. He is currently a
Research Professor in philosophy with the Department of Philosophy, Southwestern Baptist Theological 
Seminary, Fort Worth, TX. He is currently also a Senior Fellow with the Center for Science
and Culture, Discovery Institute, Seattle, WA. He has held National Science Foundation graduate
and postdoctoral fellowships. He has published articles in mathematics, philosophy, and theology
journals and is the author/editor of more than a dozen books. In The Design Inference: Eliminating 
Chance Through Small Probabilities (Cambridge University Press, 1998), he examines the
design argument in a post-Darwinian context and analyzes the connections linking chance, proba-
bility, and intelligent causation. The sequel to The Design Inference (Rowman \& Littlefield, 2002)
critiques Darwinian and other naturalistic accounts of evolution. It is titled No Free Lunch: Why
Specified Complexity Cannot Be Purchased Without Intelligence. He has edited several influential
anthologies, including Uncommon Dissent: Intellectuals Who Find Darwinism Unconvincing (ISI,
2004) and Debating Design: From Darwin to DNA (Cambridge University Press, 2004, coedited
with M. Ruse). His newest book, coauthored with J.Wells, is titled The Design of Life: Discovering
Signs of Intelligence in Biological Systems (Foundation for Thought and Ethics, 2007). His area
of interest in intelligent design has grown in the wider culture; he has assumed the role of public
intellectual. In addition to lecturing around the world at colleges and universities, he is frequently
interviewed on the radio and television. His work has been cited in numerous newspaper and
magazine articles, including three front-page stories in The New York Times as well as the August
15, 2005 Time Magazine cover story on intelligent design. He has appeared on the BBC, NPR
(Diane Rehm, etc.), PBS (Inside the Law with Jack Ford and Uncommon Knowledge with Peter
Robinson), CSPAN2, CNN, Fox News, ABC Nightline, and the Daily Show with Jon Stewart.

\biosection{Robert J. Marks II}{marks_photo.jpg}

Robert J. Marks II is currently the Distinguished Professor of Electrical and Computer Engineering 
at Baylor University, Waco, TX. He is a Fellow of both IEEE and the Optical Society
of America. He served for 17 years as the Faculty Advisor with the Campus Crusade for Christ,
University of Washington chapter. His consulting activities include Microsoft Corporation, Pacific
Gas \& Electric, and Boeing Computer Services. Eleven of his papers have been republished in collections 
of seminal works. He is the author of Introduction to Shannon Sampling and Interpolation
Theory (Springer-Verlag), Handbook of Fourier Analysis and Its Applications (Oxford University
Press) and is a co-author of Neural Smithing (MIT Press). His research has been funded by organizations 
such as the National Science Foundation, General Electric, Southern California Edison,
Electric Power Research Institute, the Air Force Office of Scientific Research, the Office of Naval
Research, the Whitaker Foundation, Boeing Defense, the National Institutes of Health, The Jet
Propulsion Laboratory, the Army Research Office, and the National Aeronautics and Space Administration 
(NASA). He is a former Editor-in-Chief of the IEEE TRANSACTIONS ON NEURAL
NETWORKS \& LEARNING SYSTEMS. He was the recipient of numerous professional awards,
including a NASA Tech Brief Award and a Best Paper Award from the American Brachytherapy
Society for prostate-cancer research. He was the recipient of the IEEE Outstanding Branch Councilor 
Award, the IEEE Centennial Medal, the IEEE Computational Intelligence Society Meritorious
Service Award, the IEEE Circuits and Systems Society Golden Jubilee Award, and, for 2007, the
IEEE CIS Chapter of the IEEE Dallas Section Volunteer of the Year Award. He was named a
Distinguished Young Alumnus of the Rose-Hulman Institute of Technology and is an inductee into
the Texas Tech Electrical Engineering Academy. He was the recipient of the Banned Item of the
Year Award from the Discovery Institute and a recognition crystal from the International Joint
Conference on Neural Networks for contributions to the field of neural networks. His Erd\"{o}s-Bacon
number is five.
