\chapter*{About the Authors}
\addcontentsline{toc}{chapter}{About the Authors}
\markboth{ABOUT THE AUTHORS}{}

% environment to standardize the bios
\newenvironment{authorbio}[2]{
	\subsection*{#1}

	\ifstrequal{#2}{}{}  % skip the image if one wasn't provided
	{
		\begin{wrapfigure}{l}{1in}
		 	%\begin{minipage}{1in}
				\includegraphics[width=1in]{#2}
			%\end{minipage}
		\end{wrapfigure}
	}

	}{ % the actual content inserted between here
	
		\mbox{}
}

\section*{Editors and Organizers}

\begin{authorbio}{Jonathan Bartlett}{bartlett_photo.jpg}
Jonathan Bartlett is the Director of The Blyth Institute in Tulsa, Oklahoma.  Jonathan's research focuses on the origin of novelty---both the origin of biological novelty in adaptation as well as the origin of insight in the human creative process.  Jonathan's other roles include managing a team of developers as the Director of Technology at New Medio, tutoring homeschool students in chemistry and calculus at Classical Conversations, and being part of the Classical Conversations Writer's Circle, where he has a monthly column discussing issues of science, faith, and education.  Jonathan received an M.T.S. from Phillips Theological Seminary, and is the author of the book \textit{Programming from the Ground Up}, which has been used in Universities from Princeton University to Oklahoma State for teaching undergraduate assembly language.
\end{authorbio}

\begin{authorbio}{Dominic Halsmer}{halsmer_photo.jpg}
Third from the youngest of 13 children, Dr. Dominic Halsmer grew up in rural Indiana where his family's large acreage provided ample room for a swimming pool, ball fields and trampoline. His godly parents and large extended family that lived near by gave him a strong foundation of faith, athleticism and adventure that endures today.

Dominic's father and uncle owned and operated a small airport and the Halsmer kids spent many happy hours trying to defy gravity at the airfield. Once, under older brother David's direction, they developed a kite large enough to lift a person off the ground using an old parachute and some rope. Dominic's natural interest in physics and engineering blossomed in this fertile ground that fostered a child's exploration and curiosity.

Dominic earned his B.S. and M.S. degrees in Aeronautical and Astronautical Engineering from Purdue University. College was a time of spiritual searching, but ultimately he returned to his faith. "The love and Christ-like example of my parents made the difference," he said. He also notes that the evidence from science for the truth of the Christian worldview also helped draw him back to his faith. Today, he has a heart for reaching out with the gospel to skeptical scientists and engineers.

In 1992, Dominic earned his Ph.D. in Mechanical Engineering from UCLA. That fall, he joined the Engineering and Physics Department at Oral Roberts University. In 1996 and 1997, he participated in the NASA/ASEE Summer Faculty Fellowship Program at NASA Goddard Space Flight Center. His research activities have involved undergraduate engineering students in developing an apparatus to test the stability of spinning spacecraft under thrust. Currently, he is studying how the universe is engineered to reveal the glory of God and accomplish His purposes.

Today, Dominic is a virtual poster boy for the ORU Whole Person concept. An accomplished athlete, he often earns a top-three slot in Tulsa area races and in 2007 placed third in the 3000 meter steeplechase at the National Master's Outdoor Track and Field Championships in Maine. But first and foremost, he is a committed Christian husband and father in addition to being an accomplished engineer with a thirst for knowledge and a love of teaching.  Dominic can often be seen running with students during lunch time.

Dominic is married to his high school sweetheart Kate, and they are the parents of four children ages 14 to 22 years.

B.S. and M.S. in Aeronautical and Astronautical engineering from Purdue University
Ph.D. in Mechanical Engineering from UCLA
\end{authorbio}

\begin{authorbio}{Mark Hall}{hall_photo.jpg}
Dr. Mark Hall, an English Professor at Oral Roberts University, was born and raised in Missouri. He currently teaches Science and the Imagination, C. S. Lewis and the Inklings, English Medieval Period, English Romantic Period, English Victorian Period, British Literature I, along with other classes. This professor, who describes ORU students as being “driven, enthusiastic, and focused,” has been teaching at ORU for over 20 years.

Dr. Hall’s extensive list of degrees include the following: B.S.E. in English Education, an M.S.E. in English, and a Specialist degree in Higher Education with an emphasis in English, all from the University of Central Missouri.  He has also completed three Masters' degrees from Oral Roberts University including an M.A. in Biblical Literature, an M.A. in Theological and Historical Studies, and an M.A. in  Biblical Literature (Advanced Languages concentration).  Dr. Hall received his Ph.D. in English from the University of Tulsa.

Before coming to teach at ORU, Dr. Hall taught as an English Teacher’s Assistant and then an English Instructor at the University of Central Missouri.  In 1987 he came to ORU to attend seminary and taught as a Teacher’s Assistant in the College of Theology and Ministry and an English Instructor in the Department of English and Modern Languages.  He then became an Assistant Professor in English, an English Associate Professor, and now he teaches as a Full Professor in English.  Dr. Hall served as the Chair for the School of LifeLong Education at ORU from 2003-2008 and has been the Coordinator of General Education at ORU since 2008.

An ordained minister and a former church pastor, he has been married for over 22 years to his wife, Rachel, who is the Music Director at Jenks United Methodist Church. Dr. Hall and his wife have two children, Jonathan and Kathryne. Jonathan is an ORU graduate and currently working on his MBA, and Kathryne is currently an ORU student majoring in Writing.  Dr. Hall is active in his community through politics and the theater and is an avid reader, with one of his favorite authors being C. S. Lewis.

Dr. Hall says, “I teach to make a difference in the lives of my students so that they can make an impact for Christ in their world.”
\end{authorbio}

\section*{Primary Authors}

\begin{authorbio}{Alexander Sich}{}
Alexander Sich is Associate Professor of Physics at Franciscan University of Steubenville. He holds a B.S. in Nuclear Engineering from Rensselaer Polytechnic Institute (with a minor in physics), an M.A. in Soviet Studies from Harvard University, an M.A. in Philosophy from Holy Apostles College \& Seminary, and a Ph.D. in Nuclear Engineering from MIT. Professor Sich conducted his Ph.D. research in Ukraine at the Chernobyl site, and worked in the former Soviet Union in nuclear safety and nonproliferation effort regarding weapons of mass destruction for over thirteen years. In addition to technical articles, Professor Sich has published opinion pieces on nuclear safety (including the Iranian Bushehr issue) in \emph{The Bulletin of the Atomic Scientists}, \emph{The Boston Globe}, The \emph{Wall Street Journal}, \emph{The Diplomat}, and \emph{Newsday}. Professor Sich is married with seven children, speaks near native fluent Ukrainian and fluent Russian. His current research interests lie primarily in the philosophy of nature, and in teaching.
\end{authorbio}

\begin{authorbio}{Eric Holloway}{holloway_photo.jpg}
Mr. Holloway is currently an officer in the Air Force and has been on
active duty for eight years (including a deployment to Afghanistan in
2010). He has a BSc from Biola and an MSc from the Air Force Institute
of Technology, both in computer science. The MSc was focused on AI
with an emphasis on evolutionary algorithms and was funded by an \$80k
grant from the Air Force Research Lab. The work was was published in
two conference proceedings, IEEE and ACM.
\end{authorbio}

\begin{authorbio}{Winston Ewert}{ewert_photo.jpg}

Winston Ewert hails from Canada where he earned a Bachelor's degree in Computer Science from Trinity Western University. He continued his graduate career at Baylor University where he earned a master's degree and is currently a Ph.D. candidate. While at Baylor he is working for the Evolutionary Informatics Lab, with Robert J. Marks II and William Dembski, dedicated to understanding the role of information in evolution. He has a number of publications in the areas of search, conservation of information, artificial life, swarm intelligence, and evolutionary modelling. 
\end{authorbio}


\begin{authorbio}{Arminius Mignea}{}

\end{authorbio}


\section*{Additional Authors}



\begin{authorbio}{Michael Gewecke}{mgewecke_photo.jpg}
Michael grew up in Saint Cloud, MN, where he attended a small private Christian school from kindergarten through high school.   Michael was born and later baptized in the Pentecostal church although his family later moved into a non-denominational church.   Since high school, Michael has been actively involved in church leadership and has volunteered within church media and technology.  During his senior year of high school, Michael directed a full length feature film that was shown over multiple days in the movie theater in Saint Cloud.  That same year, Michael graduated valedictorian of his class and was accepted as an honors fellow at Oral Roberts University.  While studying theology and church history Michael met his future wife, Rachelle, became Presbyterian, and actively participated in the life and ministry of First Presbyterian Church in Tulsa, OK.  Upon graduating from Oral Roberts University with honors, Michael attended Princeton Theological Seminary where he has ministered in two local congregations and a state psychiatric hospital while also serving as CEO of a company that serves churches by helping them establish an effective online presence for their ministries.  
\end{authorbio}

\begin{authorbio}{Rachelle Gewecke}{rgewecke_photo.jpg}
Rachelle grew up in rural Iowa where she was baptized in the Reformed Church of America.  Rachelle begged her mother (a music professor) to teach her to play the piano when she was four years old, and she has been playing ever since.  Within her twenty plus years with the instrument, Rachelle has performed in a broad range of venues, including state youth symphonies, university orchestras, faculty studios, and, most recently, as accompanist and choir director at Community Presbyterian Church of the Sand Hills, NJ.  When Rachelle is not playing piano (or viola) she is often at home with Emily, where she has taken on the task of redirecting her toddler’s seemingly boundless energy.
\end{authorbio}

\begin{authorbio}{William A. Dembski}{dembski_photo.jpg}

William A. Dembski received the B.A. degree in psychology, the M.S. degree in statistics, the
Ph.D. degree in philosophy, and the Ph.D. degree in mathematics in 1988 from the University of
Chicago, Chicago, IL, and the M.Div. degree from Princeton Theological Seminary, Princeton,
NJ, in 1996. He was an Associate Research Professor with the Conceptual Foundations of Science, 
Baylor University, Waco, TX, where he also headed the First intelligent design think-tank
at a major research university: The Michael Polanyi Center. He was the Carl F. H. Henry Professor 
in theology and science with The Southern Baptist Theological Seminary, Louisville, KY,
where he founded its Center for Theology and Science. He has taught at Northwestern University, 
Evanston, IL, the University of Notre Dame, Notre Dame, IN, and the University of Dallas,
Irving, TX. He has done postdoctoral work in mathematics with the Massachusetts Institute of
Technology, Cambridge, in physics with the University of Chicago, and in computer science with
Princeton University, Princeton, NJ. He is a Mathematician and Philosopher. He is currently a
Research Professor in philosophy with the Department of Philosophy, Southwestern Baptist Theological 
Seminary, Fort Worth, TX. He is currently also a Senior Fellow with the Center for Science
and Culture, Discovery Institute, Seattle, WA. He has held National Science Foundation graduate
and postdoctoral fellowships. He has published articles in mathematics, philosophy, and theology
journals and is the author/editor of more than a dozen books. In The Design Inference: Eliminating 
Chance Through Small Probabilities (Cambridge University Press, 1998), he examines the
design argument in a post-Darwinian context and analyzes the connections linking chance, proba-
bility, and intelligent causation. The sequel to The Design Inference (Rowman \& Littlefield, 2002)
critiques Darwinian and other naturalistic accounts of evolution. It is titled No Free Lunch: Why
Specified Complexity Cannot Be Purchased Without Intelligence. He has edited several influential
anthologies, including Uncommon Dissent: Intellectuals Who Find Darwinism Unconvincing (ISI,
2004) and Debating Design: From Darwin to DNA (Cambridge University Press, 2004, coedited
with M. Ruse). His newest book, coauthored with J.Wells, is titled The Design of Life: Discovering
Signs of Intelligence in Biological Systems (Foundation for Thought and Ethics, 2007). His area
of interest in intelligent design has grown in the wider culture; he has assumed the role of public
intellectual. In addition to lecturing around the world at colleges and universities, he is frequently
interviewed on the radio and television. His work has been cited in numerous newspaper and
magazine articles, including three front-page stories in The New York Times as well as the August
15, 2005 Time Magazine cover story on intelligent design. He has appeared on the BBC, NPR
(Diane Rehm, etc.), PBS (Inside the Law with Jack Ford and Uncommon Knowledge with Peter
Robinson), CSPAN2, CNN, Fox News, ABC Nightline, and the Daily Show with Jon Stewart.
\end{authorbio}

\begin{authorbio}{Robert J. Marks II}{marks_photo.jpg}

Robert J. Marks II is currently the Distinguished Professor of Electrical and Computer Engineering 
at Baylor University, Waco, TX. He is a Fellow of both IEEE and the Optical Society
of America. He served for 17 years as the Faculty Advisor with the Campus Crusade for Christ,
University of Washington chapter. His consulting activities include Microsoft Corporation, Pacific
Gas \& Electric, and Boeing Computer Services. Eleven of his papers have been republished in collections 
of seminal works. He is the author of Introduction to Shannon Sampling and Interpolation
Theory (Springer-Verlag), Handbook of Fourier Analysis and Its Applications (Oxford University
Press) and is a co-author of Neural Smithing (MIT Press). His research has been funded by organizations 
such as the National Science Foundation, General Electric, Southern California Edison,
Electric Power Research Institute, the Air Force Office of Scientific Research, the Office of Naval
Research, the Whitaker Foundation, Boeing Defense, the National Institutes of Health, The Jet
Propulsion Laboratory, the Army Research Office, and the National Aeronautics and Space Administration 
(NASA). He is a former Editor-in-Chief of the IEEE TRANSACTIONS ON NEURAL
NETWORKS \& LEARNING SYSTEMS. He was the recipient of numerous professional awards,
including a NASA Tech Brief Award and a Best Paper Award from the American Brachytherapy
Society for prostate-cancer research. He was the recipient of the IEEE Outstanding Branch Councilor 
Award, the IEEE Centennial Medal, the IEEE Computational Intelligence Society Meritorious
Service Award, the IEEE Circuits and Systems Society Golden Jubilee Award, and, for 2007, the
IEEE CIS Chapter of the IEEE Dallas Section Volunteer of the Year Award. He was named a
Distinguished Young Alumnus of the Rose-Hulman Institute of Technology and is an inductee into
the Texas Tech Electrical Engineering Academy. He was the recipient of the Banned Item of the
Year Award from the Discovery Institute and a recognition crystal from the International Joint
Conference on Neural Networks for contributions to the field of neural networks. His Erd\"{o}s-Bacon
number is five.
\end{authorbio}

\begin{authorbio}{Nate Roman}{}
FIXME - need info
\end{authorbio}

\begin{authorbio}{Tyler Todd}{}
FIXME - need info
\end{authorbio}

\begin{authorbio}{Jessica Fitzgerald}{fitzgerald_photo.jpg}
Jessica Fitzgerald is a student at Oral Roberts University studying Engineering Physics.  She is pursuing a Bachelor of Science Degree in Engineering Physics with a minor in Mathematics. She plans to attend graduate school for physics or engineering, with a particular interest in nanoscience. Her strengths include data analysis, computation, and attention to detail.
\end{authorbio}
